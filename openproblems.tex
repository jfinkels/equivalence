%%%%
%% openproblems.tex
%%
%% Copyright 2011, 2012 Jeffrey Finkelstein
%%
%% Except where otherwise noted, this work is made available under the terms of
%% the Creative Commons Attribution-ShareAlike 3.0 license,
%% http://creativecommons.org/licenses/by-sa/3.0/.
%%
%% You are free:
%%    * to Share — to copy, distribute and transmit the work
%%    * to Remix — to adapt the work
%% Under the following conditions:
%%    * Attribution — You must attribute the work in the manner specified by
%%    the author or licensor (but not in any way that suggests that they
%%    endorse you or your use of the work).
%%    * Share Alike — If you alter, transform, or build upon this work, you may
%%    distribute the resulting work only under the same, similar or a
%%    compatible license.
%%    * For any reuse or distribution, you must make clear to others the
%%    license terms of this work. The best way to do this is with a link to the
%%    web page http://creativecommons.org/licenses/by-sa/3.0/.
%%    * Any of the above conditions can be waived if you get permission from
%%    the copyright holder.
%%    * Nothing in this license impairs or restricts the author's moral rights.
%%%%
\section{Open problems}
\label{sec:openproblems}

Besides the open problems listed throughout the paper, we consider the following questions to be worth exploring.
\begin{itemize}
\item There are many problems of \emph{inequivalence} in \cite{gj79} which are listed as \NP-complete or \PSPACE-complete.
  What do these problems have to do with \NPEq-completeness, \coNPEq-completeness, and \PSPACEEq-completeness?
\item Do the complexity results from, for example, \cite{at96} or \cite{rs11}, which study isomorphisms and congruences of boolean formulae, boolean circuits, polynomials, and other structures, translate to the setting of kernel reductions?
\end{itemize}
