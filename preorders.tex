%%%%
%% preorders.tex
%%
%% Copyright 2011 Jeffrey Finkelstein
%%
%% Except where otherwise noted, this work is made available under the terms of
%% the Creative Commons Attribution-ShareAlike 3.0 license,
%% http://creativecommons.org/licenses/by-sa/3.0/.
%%
%% You are free:
%%    * to Share — to copy, distribute and transmit the work
%%    * to Remix — to adapt the work
%% Under the following conditions:
%%    * Attribution — You must attribute the work in the manner specified by
%%    the author or licensor (but not in any way that suggests that they
%%    endorse you or your use of the work).
%%    * Share Alike — If you alter, transform, or build upon this work, you may
%%    distribute the resulting work only under the same, similar or a 
%%    compatible license.
%%    * For any reuse or distribution, you must make clear to others the 
%%    license terms of this work. The best way to do this is with a link to the
%%    web page http://creativecommons.org/licenses/by-sa/3.0/.
%%    * Any of the above conditions can be waived if you get permission from
%%    the copyright holder.
%%    * Nothing in this license impairs or restricts the author's moral rights.
%%%%
\section{Preorder reductions}

The definition of kernel reduction can be extended to other types of binary relations.
In this section we examine computational problems concerning preorders.

Given a universe $U$, $P\subseteq U\times U$ is a \defn{preorder on $U$} if $P$ is
\begin{enumerate}
\item reflexive: for all $x\in U$, $(x,x)\in P$
\item transitive: for all $x,y,z\in U$, $(x,y)\in P$ and $(y,z)\in P$ implies $(x,z)\in P$
\end{enumerate}

If $P$ and $Q$ are preorders on $\Sigma^*$, we say $P$ \defn{preorder reduces to} $Q$ if there exists a computable $f\colon\Sigma^*\to\Sigma^*$ such that $\forall x,y\in\Sigma^*$, $\pair{x}{y}\in P\iff \pair{f(x)}{f(y)}\in Q$.
We denote this by $R\prnt S$.
If $f$ is computable in polynomial time, then we say $P$ \defn{polynomial time preorder reduces to} $Q$ and use the notation $P\pr Q$.

Note that an equivalence relation is a preorder which is also symmetric.
Therefore a kernel reduction is in fact a preorder reduction.

The subgraph isomorphism problem, defined below, is a preorder (and is also \NP-complete\cite{gj79}).
\begin{langdef}{SGI}
\begin{instance}
  undirected graphs $G$ and $H$
\end{instance}
\begin{question}
  Does $H$ contain a subgraph that is isomorphic to $G$?
\end{question}
\end{langdef}

\begin{proposition}
  \lang{SGI} is a preorder.
\end{proposition}
\begin{proof}
  \lang{SGI} is reflexive since a graph $G$ always has a subgraph that is isomorphic to $G$ (namely itself).

  For transitivity, suppose that $\pair{G}{H}\in\lang{SGI}$ and $\pair{H}{J}\in\lang{SGI}$, so $H$ has a subgraph, call it $H'$, that is isomorphic to $G$ by the function $\phi_1$ and $J$ has a subgraph, call it $J'$, that is isomorphic to $H$ by the function $\phi_2$.
  Since $H$ has a subgraph that is isomorphic to $G$ and $H$ is isomorphic to $J'$, it follows that $J'$ has a subgraph that is isomorphic to $G$ (namely the subgraph induced by the image of $H'$ under the isomorphism $\phi_2$).
\end{proof}

We know that $\GI\mor\lang{SGI}$, since $\GI\in\NP$ and $\lang{SGI}$ is \NP-complete, but is it also true that $\GI\pr\lang{SGI}$?
Is there another natural example of a preorder reduction, would such a reduction (being inherently asymmetric) require information about both elements of the pair?
