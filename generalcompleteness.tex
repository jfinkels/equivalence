%% generalcompleteness.tex - completeness under kernel reductions
%%
%% Copyright 2010, 2011, 2012, 2014 Jeffrey Finkelstein.
%%
%% This LaTeX markup document is made available under the terms of the Creative
%% Commons Attribution-ShareAlike 4.0 International License,
%% https://creativecommons.org/licenses/by-sa/4.0/.
\section
    [Conditions for complete problems under polynomial time kernel reductions]
    {Conditions for complete problems \\ under polynomial time kernel reductions}
\label{sec:generalcompleteness}
% Foreword
%
% context (focus on anyone) why now? - current situation, and why the need is so important
Most well-behaved complexity classes contain problems that are complete under many-one reductions.
Do the corresponding classes of equivalence problems contain problems that are complete under kernel reductions?
% need (focus on readers) why you? - why this is relevant to the reader, and why something needed to be done
Having access to a complete problem offers many benefits to a complexity theorist, and improves our understanding of equivalence problems in general.
In \autocite[Theorem~8.7]{bcffm}, the authors constructed a complete problem with respect to polynomial time kernel reductions for $\NPEq$ under the assumption that $\NP = \coNP$.
% task (focus on author) why me? - what was undertaken to address the need
Since we consider that assumption unlikely, we determined sufficient conditions for having a complete problem under polynomial time kernel reductions.
% object (focus on document) why this document - what the document covers
This section presents a more general theorem that implies as a corollary a complete problem for $\NPEq$ under the assumption $\NP = \coNP$.

% Summary
%
% findings (focus on author) what? - what the work revealed when performing the task
By extending the technique of \autocite[Theorem~8.7]{bcffm}, we found that $\PSPACEEq$ has a complete problem under polynomial time kernel reductions unconditionally.
We also show that each level of the polynomial time hierarchy contains an equivalence problem that is hard for the lower levels under these reductions.
% conclusion (focus on readers) so what? - what the findings mean for the audience
This means that some well-known classes do have complete problems, and the existence for complete problems in other classes, like $\NP$ and even $\P$, remains possible.
% perspective (focus on anyone) what now? - what should be done next
Since the equivalence problem we constructed is highly artificial, we were unable to find any natural problems that are hard or complete.

We need one additional definition in order to describe the complexity classes that contain a hard problem under kernel reductions.
If $\mathcal{C}$ is a complexity class then the class $\forall\mathcal{C}$ is the set of languages $A$ such that there exists a language $B\in\mathcal{C}$ and a polynomial $p$ satisfying $x\in A$ if and only if $\forall w\in\Sigma^{\leq p(|x|)} \pair{x}{w}\in B$.
$\forall\mathcal{C}$ is called the \defn{closure of $\mathcal{C}$ under polynomially bounded universal quantification}.

\begin{theorem}\label{thm:generalcompleteness}
  Let $\mathcal{C}$ be a subset of $\PSPACE$ which contains the problem of deciding whether two strings are equal.
  Then there exists an equivalence relation in $\CRAZYEq$ which is hard for $\CEq$ under $\kr$ reductions.
\end{theorem}

Before proving this theorem, we will provide some immediate corollaries of this general result.

\begin{corollary}\label{cor:sufficient}
  If $\mathcal{C}$ is a subset of $\PSPACE$ and $\mathcal{C}=\CRAZY$, then $\CEq$ has a complete problem under $\kr$ reductions.
\end{corollary}

\begin{corollary}\label{cor:hardproblems}
  Under polynomial time kernel reductions,
  \begin{enumerate}
  %\item $\EXPEq$ has a complete problem
  \item $\PSPACEEq$ has a complete problem
  \item $\PKPEq$ contains a problem that is hard for $\DKPEq$, for all $k \geq 1$
  %\item\label{itm:hardforsigma} $\PKPOPEq$ contains a problem which is hard for $\SKPEq$, for all $k\geq 0$
  %\item $\PKPOPEq$ contains a problem which is hard for $\PKPEq$, for all $k\geq 0$
  %\item $\coNPEq$ contains a problem which is hard for $\PEq$
  \end{enumerate}
\end{corollary}
\begin{proof}\mbox{}
  \begin{enumerate}
  %\item $\EXP$ is closed under complement (because it is a deterministic complexity class) and polynomially bounded universal quantification (because we can simulate the universal guess deterministically in exponential time).
  \item $\PSPACE$ is closed under complement (because it is a deterministic complexity class) and polynomially bounded universal quantification (because we can simulate the universal guess deterministically in polynomial space).
  \item
    First, $(\forall(\DKP \cup \mathsf{co}\DKP))\mathsf{Eq} = (\forall \DKP)\mathsf{Eq}$, since $\DKP$ is closed under complement.  % \autocite[Proposition~8.4 (a)]{bdg95}
    Next, $(\forall \DKP)\mathsf{Eq} = \PKPEq$, since $\forall \DKP = \PKP$.  % \autocite[Proposition~8.3 (g)]{bdg95}
    Now if we choose $\mathcal{C} = \DKP$ in \autoref{thm:generalcompleteness}, then $\PKPEq$ has a problem that is hard for $\DKPEq$ under $\kr$ reductions.
    \qedhere
  %\item If $\mathcal{C}=\SKP$, then the $\kr$-hard problem is in $(\forall(\SKP\cup\mathsf{co}\SKP))\mathsf{Eq}=(\forall(\SKP\cup\PKP))\mathsf{Eq}=\PKPOPEq$.
  %\item Same as the previous justification, but starting with $\mathcal{C}=\PKP$.\qedhere
%  \item Same as the previous justification, but starting with $\mathcal{C}=\P$.
  \end{enumerate}
\end{proof}

More specifically, this means that $\coNPEq$ (which equals $\POPEq$) has a problem that is $\kr$-hard for $\PEq$ (which equals $\DOPEq$).
This leads to \autocite[Theorem~8.7, part~1]{bcffm}, which is restated here.

\begin{corollary}[{\autocite[Theorem~8.7, part~1]{bcffm}}]
  If $\NP = \coNP$ then $\NPEq$ has a complete problem under polynomial time kernel reductions.
\end{corollary}
\begin{proof}
  If $\NP = \coNP$, then the polynomial hierarchy collapses to $\POP$, and specifically $\PTP = \DTP = \POP = \coNP = \NP$.
  From \autoref{cor:hardproblems} we conclude that $\NPEq$ has a $\kr$-hard problem for $\NPEq$.
  Such a problem is by definition $\NPEq$-complete.
\end{proof}

We now return to the proof of \autoref{thm:generalcompleteness} by first providing some motivating ideas.
Recall the canonical complete problem (sometimes called the ``universal'' problem) for $\NP$ (and indeed for various other complexity classes):
\begin{displaymath}
  K = \lb\triple{M}{x}{1^t} \st M\plain{accepts} x \plain{within} t \textnormal{ steps}\rb
\end{displaymath}
The idea of this proof is to adapt this into an equivalence relation $R_K$ consisting of pairs of triples of the form $\pair{\triple{M}{x}{1^{t_x}}}{\triple{M}{y}{1^{t_y}}}$, where $M$ accepts $\pair{x}{y}$, as in the reduction from an arbitrary $\NP$ language to $K$.
The problem we encounter here is that $R_K$ is not necessarily an equivalence relation.
Consider, for example, transitivity, which must be satisfied for all possible pairs of the form $\triple{M}{w}{1^{t_w}}$.
For \emph{arbitrary machines} $M$, just because $M$ accepts $\pair{x}{y}$ and $\pair{y}{z}$ does not necessarily mean that $M$ accepts $\pair{x}{z}$.
The solution is to encode into $R_K$ the requirement that the language which $M$ accepts, $L(M)$, is itself an equivalence relation.
The three properties required of $R_K$ then follow from the properties of $L(M)$.
%
%As a technical consideration for this proof, we point out languages in $\PSPACE$ may be decided by alternating Turing machines which run in polynomial time, so it is permissible to consider polynomially clocked Turing machines.
%
\begin{proof}[Proof of \autoref{thm:generalcompleteness}]
  First we will define a helper algorithm which decides whether a given machine accepts an equivalence relation on strings up to a given length.
  Define the algorithm $A$ as follows on input $\pair{M}{n}$, where $M$ is a polynomially clocked Turing machine of type $\mathcal{C}$ and $n\in\mathbb{N}$:
  \begin{enumerate}
  \item universally guess $a,b,$ and $c\in\Sigma^{\leq n}$
  \item simulate $M$ on $\pair{a}{a}$; if it rejects, reject
  \item simulate $M$ on $\pair{a}{b}$, then on $\pair{b}{a}$; if the former accepts and the latter rejects, reject
  \item simulate $M$ on $\pair{a}{b}$, then on $\pair{b}{c}$, then on $\pair{a}{c}$; if the first two accept and the last one rejects, reject
  \item if execution reaches this point, accept
  \end{enumerate}
  These simulations check that $L(M)$ satisfies reflexivity, symmetry, and transitivity on strings of length at most $n$.
  If $A$ accepts, then the three properties are satisfied, and if it rejects then one of the three properties is violated.
  Since $M$ is a machine of type $\mathcal{C}$, checking if $M$ accepts on some input and if $M$ rejects on some input is in $\mathcal{C}\cup\mathsf{co}\mathcal{C}$.
  The universal guesses of $a,b,$ and $c$ (of length at most $n$) followed by checks of whether the six simulations of $M$ accept or reject place $L(A)$ in the class $\CRAZY$.
  If $p$ is the polynomial which bounds the running time of $M$, then the running time of this algorithm is $6p\left(\left|\pair{1^n}{1^n}\right|\right)+c$, where $c$ is a constant which represents the time needed to account for the implementation of $A$ (the control of the simulations of $M$, performing logical conjunctions, etc.).
  Hence the running time of $A$ is polynomial in $n$.

  Now we can define the set $R_K$ by
  \begin{align*}
    R_K = {} & \lb\pair{u}{v} \st u=v\rb \\
    & \cup \lb\pair{\triple{M}{x}{1^{t_x}}}{\triple{M}{y}{1^{t_y}}} \st \textnormal{1 through 4 below are satisfied}\rb
  \end{align*}
  where the conditions are
  \begin{enumerate}
  \item\label{itm:machine} $M$ is a polynomially clocked Turing machine of type $\mathcal{C}$
  \item\label{itm:emx} $A$ accepts $\pair{M}{|x|}$ within $t_x$ steps
  \item\label{itm:emy} $A$ accepts $\pair{M}{|y|}$ within $t_y$ steps
  \item\label{itm:accepts} $M$ accepts $\pair{x}{y}$
  \end{enumerate}
  We claim that $R_K$ is in $\CRAZYEq$ and $\CEq$-hard.

  First we show that $R_K\in\CRAZY$.
  By the argument above, $A$ is a $\CRAZY$ algorithm.
  %Since $M$ is a polynomially clocked $\mathcal{C}$ machine by \autoref{itm:machine}, then the simulation of $M$ on $\pair{x}{y}$ in \autoref{itm:accepts} can be performed by a $\mathcal{C}$ algorithm.
  Assuming without loss of generality that $|x|\geq |y|$, if $A$ accepts $\pair{M}{|x|}$ within $t_x$ steps then we know that there is a polynomial time bound on the running time of $M$ on input $\pair{x}{y}$, so simulating it is certainly in $\CRAZY$.
  Finally, testing for equality is in $\mathcal{C}$ by hypothesis so deciding $R_K$ overall can be performed by a $\CRAZY$ algorithm.

  Next we show that $R_K$ is an equivalence relation.
  Reflexivity follows from the reflexivity of the equality relation.
  For symmetry, suppose that the pair $\pair{\triple{M}{x}{1^{t_x}}}{\triple{M}{y}{1^{t_y}}}$ is in $R_K$.
  Since \autoref{itm:emx} and \autoref{itm:emy} are true by hypothesis, we know that symmetry on strings of length at most $\max(|x|, |y|)$ in $L(M)$ is satisfied, and that includes the strings $x$ and $y$.
  So since $M$ accepts $\pair{x}{y}$ it must follow that $M$ accepts $\pair{y}{x}$.
  Furthermore, \autoref{itm:machine}, \autoref{itm:emx}, and \autoref{itm:emy} are the same up to symmetry of $x$ and $y$, so we have $\pair{\triple{M}{y}{1^{t_y}}}{\triple{M}{x}{1^{t_x}}}\in R_K$.
  For transitivity, suppose that both $\pair{\triple{M}{x}{1^{t_x}}}{\triple{M}{y}{1^{t_y}}}\in R_K$ and $\pair{\triple{M}{y}{1^{t_y}}}{\triple{M}{z}{1^{t_z}}}\in R_K$.
  Since transitivity is true on strings of length at most $\max(|x|, |y|, |z|)$ by the transitivity propositions checked by \autoref{itm:emx} and \autoref{itm:emy}, and since $M$ accepts both $\pair{x}{y}$ and $\pair{y}{z}$ by hypothesis, it must follow that $M$ accepts $\pair{x}{z}$.
  Again the conditions in \autoref{itm:machine}, \autoref{itm:emx}, and \autoref{itm:emy} are the same.
  We have shown that $R_K$ is reflexive, symmetric, and transitive, so it is an equivalence relation.
  At this point, we have proven that $R_K\in\CRAZYEq$.

  Now we need to show that $R_K$ is $\CEq$-hard.
  Let $S\in\CEq$.
  Suppose $M$ is the polynomially clocked $\mathcal{C}$ machine which decides $S$, and $p$ is the polynomial which bounds the running time of $M$.
  Then the kernel reduction from $S$ to $R_K$ is $w\mapsto\triple{M}{w}{1^{6p(|\pair{w}{w}|)+c}}$, where $p$ and $c$ are the polynomial and constant described in the first paragraph of this proof.
  Call this reduction $f$.
  The reduction is obviously computable in time polynomial in $|w|$.
  It remains to show that this reduction is correct.

  Suppose $\pair{x}{y}\in S$.
  Now $f(x)=\triple{M}{x}{1^{6p(|\pair{x}{x}|)+c}}$ and, similarly, $f(y)=\triple{M}{y}{1^{6p(|\pair{y}{y}|)+c}}$.
  \autoref{itm:machine} is true by construction, and \autoref{itm:accepts} is true since $M$ is the machine which decides $S$.
  Assume \autoref{itm:emx} is false.
  Then $M$ does not accept an equivalence relation on strings of length at most $|x|$.
  This is a contradiction, since $M$ decides $S$, an equivalence relation, by hypothesis.
  Therefore \autoref{itm:emx} must be satisfied.
  The same argument applies to \autoref{itm:emy}.
  Hence $\pair{f(x)}{f(y)}\in R_K$.

  If $\pair{x}{y}\notin S$ then $M$ does not accept $\pair{x}{y}$, since otherwise $\pair{x}{y}$ would be a member of $S$.
  Hence $\pair{x}{y}\notin R_K$.
  Therefore we have shown that $R_K$ is $\CEq$-hard.
\end{proof}

\begin{openproblem}
  Is there a more general characterization of complexity classes which have a $\kr$-hard problem?
\end{openproblem}

\begin{openproblem}
  Under what conditions does a complexity class have a complete problem?
  Can we adapt this idea to create a complete problem for $\PEq$ or $\NPEq$?
\end{openproblem}

\begin{openproblem}
  Is the converse of \autoref{cor:sufficient}, or perhaps a partial converse, true?
  In other words, is it true that the existence of a $\CEq$-complete problem problem implies closure under complement or universal quantification (or both)?
  If so, this would be evidence that no $\NPEq$-complete problem exists, since this would imply $\NP = \coNP$.
\end{openproblem}

\begin{openproblem}
  Can this theorem be used to construct $\krnt$-hard problems for smaller complexity classes such as $\NLEq$ under the appropriate time-bounded reduction?
  Larger classes such as $\EXPEq$?
\end{openproblem}

\begin{openproblem}
  To what other equivalence relations does our $\kr$-hard problem reduce?
  Are there ``natural'' $\kr$-hard problems in complexity classes which satisfy the conditions in \autoref{thm:generalcompleteness}?
\end{openproblem}

%% As an additional corollary, we show that the equivalence relation $R_K$ is necessarily hard given a known hard equivalence relation under $\kr$ reductions.

%% \begin{corollary}
%%   Let $\mathcal{C}_1$ be a complexity class and $\mathcal{C}_2$ be a subset of $\PSPACE$ which contains the problem of deciding whether two strings are equal.
%%   If there exists an equivalence relation $S$ in $\CTEq$ which is hard for $\COEq$ under $\kr$ reductions, then there is an equivalence relation in $\CRAZIEREq$ which is hard for $\COEq$ under $\kr$ reductions.
%% \end{corollary}
%% \begin{proof}
%%   $S\kr R_K$ by the reduction described in the proof of \autoref{thm:generalcompleteness}.
%%   A similar analysis shows that $R_K\in\CRAZIEREq$.
%%   Since $S$ is hard for $\COEq$ and polynomial time kernel reductions compose, $R_K$ is also hard for $\COEq$.
%% \end{proof}

%% This last question leads us to briefly note that equivalence of true quantified boolean formulas is a \PSPACE-complete problem; perhaps it is also a \PSPACEEq-complete problem.
%% \begin{proposition}
%%   Define $\QBFEq$ by
%%   \begin{displaymath}
%%     \QBFEq = \{\pair{\phi}{\psi} | \phi \iff \psi\},
%%   \end{displaymath}
%%   where $\phi$ and $\psi$ are fully quantified boolean formulae.
%%   Then $\QBFEq$ is \PSPACE-complete.
%% \end{proposition}
%% \begin{proof}
%%   Suppose $\phi$ denotes $\overline{Q}\tau$, where $\overline{Q}$ represents the sequence of quantified variables and $\tau$ is the boolean formula over those variables.
%%   Then the many-one reduction from $\QBF$ is $\phi\mapsto\pair{\phi}{\exists z\colon\overline{Q}(\tau\land z)}$, where $z$ is a variable not already in $\phi$.
%%   This reduction can obviously be computed in time polynomial in the length of the input, $\phi$.
%%   If $\phi$ is not satisfiable, then no assignment of $z$ makes $\overline{Q}(\tau\land z)$ satisfiable, because $\tau$ will always be false.
%%   Hence $\exists z\colon\overline{Q}(\tau\land z)$ must be false.
%%   If $\phi$ is satisfiable, then choosing $z$ to be true makes $\overline{Q}(\tau\land z)$ true.
%%   Thus $\phi$ is satisfiable if and only if $\exists z\colon\overline{Q}(\tau\land z)$ is satisfiable, so $\QBF\mor\QBFEq$.
%%   Since $\QBFEq$ is clearly decidable in $\PSPACE$, we conclude that $\QBFEq$ is \PSPACE-complete.
%% \end{proof}
