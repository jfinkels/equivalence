Problems of determining equivalence of objects abound in theoretical computer
science and have widespread applications in a more practical setting. Indeed,
one of the most well-studied open problems in computer science is determining
the computational complexity of the graph isomorphism problem ($GI$). If
$\P\neq\NP$, then $GI$ is a candidate for $\P\backslash\NP$.

A recent paper by Fortnow and Grochow \cite{fg09} defines a new kind of
reduction among equivalence relations, a \defn{kernel reduction}, which may be
more natural than the usual many-one reduction for problems of determining
equivalence. They suggest considering polynomial time kernel reductions for
equivalence problems in \P, but in \autoref{sec:p} we will provide some
evidence that allowing a polynomial time kernel reduction among equivalence
relations for which membership can be decided in polynomial time may be too
powerful. The main result to this effect is \autoref{thm:reps}, in which we
show that the problem of computing a polynomial time kernel reduction among
feasible equivalence problems can, under certain conditions, be reduced to the
problem of computing representatives of equivalence classes for each respective
equivalence relation.

In \autoref{sec:np} we attempt to determine whether there exist any equivalence
problems which are complete under polynomial time kernel reductions, and what
they look like. We also analyze the class of equivalence problems which 
