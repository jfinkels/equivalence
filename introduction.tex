%%%%
%% introduction.tex
%%
%% Copyright 2011 Jeffrey Finkelstein
%%
%% Except where otherwise noted, this work is made available under the terms of
%% the Creative Commons Attribution-ShareAlike 3.0 license,
%% http://creativecommons.org/licenses/by-sa/3.0/.
%%
%% You are free:
%%    * to Share — to copy, distribute and transmit the work
%%    * to Remix — to adapt the work
%% Under the following conditions:
%%    * Attribution — You must attribute the work in the manner specified by
%%    the author or licensor (but not in any way that suggests that they
%%    endorse you or your use of the work).
%%    * Share Alike — If you alter, transform, or build upon this work, you may
%%    distribute the resulting work only under the same, similar or a 
%%    compatible license.
%%    * For any reuse or distribution, you must make clear to others the 
%%    license terms of this work. The best way to do this is with a link to the
%%    web page http://creativecommons.org/licenses/by-sa/3.0/.
%%    * Any of the above conditions can be waived if you get permission from
%%    the copyright holder.
%%    * Nothing in this license impairs or restricts the author's moral rights.
%%%%
\section{Introduction}

In this paper we examine the power of ``kernel reductions'' on languages induced by equivalence relations, specifically for which membership can be decided by a non-deterministic Turing machine running in time polynomial in the length of the input.
Given two equivalence relations $R$ and $S$, each of which can be expressed as a set of pairs of strings, a \defn{kernel reduction} from $R$ to $S$ is a function $f$ for which $\pair{x}{y}\in R\iff\pair{f(x)}{f(y)}\in S$.
This function maps each \emph{member} of the pair in $R$ to an element of a pair in the relation $S$, instead of mapping the \emph{entire} pair to another pair.
The full definition of ``kernel reduction'' is given by Fortnow and Grochow \cite{fg11} (and is repeated below), though the idea has existed before then.
In fact, all polynomial time many-one reductions to and from the graph isomorphism problem are, to the best of our knowledge, in fact kernel reductions, though they have not before been called by this name.
The same seems to go for polynomial time many-one reductions to and from other equivalence problems.
This kind of reduction is more natural than the usual many-one reduction for problems of equivalence.
It is therefore important to study the power of these reductions and how they can help further classify currently known and newly discovered complexity classes.

In \autoref{sec:preliminaries} we provide the definitions necessary for the study of polynomial time kernel reductions and effectively computable equivalence problems.
In \autoref{sec:definitions} we provide some possible definitions for $\NPEq$, the class of equivalence problems for which membership can be decided in nondeterministic polynomial time.
In \autoref{sec:basicfacts} we examine some basic facts about polynomial time kernel reductions which are useful for developing an understanding about the power of these reductions with respect to polynomial time many-one reductions.
In \autoref{sec:generalcompleteness} we provide general sufficient conditions under which a complexity class consisting of equivalence relations contains a problem which is hard under polynomial time kernel reductions.
In \autoref{sec:npeqcompleteness} we explore the relationship between completeness in $\NP$ under polynomial time many-one reductions and completeness in $\NPEq$ under polynomial time kernel reductions.
In \autoref{sec:intermediary} we adapt a proof by Sch\"{o}ning's ``uniform diagonalization theorem'' to the setting of equivalence relations and polynomial time kernel reductions in order to prove results about intermediary problems.
In \autoref{sec:openproblems} we provide a few open problems, although some open problems are posed throughout this work.
