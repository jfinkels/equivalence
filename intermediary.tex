%% intermediary.tex - NP-intermediary equivalence relations
%%
%% Copyright 2010, 2011, 2012, 2014, 2015 Jeffrey Finkelstein.
%%
%% This LaTeX markup document is made available under the terms of the Creative
%% Commons Attribution-ShareAlike 4.0 International License,
%% https://creativecommons.org/licenses/by-sa/4.0/.
\newcommand{\floor}[1]{\lfloor #1 \rfloor}
\section{Existence of intermediary problems}
\label{sec:intermediary}
% Foreword
%
% context (focus on anyone) why now? - current situation, and why the need is so important
According to the seminal theorem by Ladner \autocite{ladner75}, if $\P \neq \NP$, then there are problems of intermediate complexity, in the sense that these problems are neither in $\P$ nor $\NP$-complete.
The theorem does not immediately imply a similar result for equivalence problems, since $\PEq$ is different from $\P$ and $\NPEq$ is different from $\NP$ (specifically, in each case, the latter contains problems that are not equivalence problems).
% need (focus on readers) why you? - why this is relevant to the reader, and why something needed to be done
Do kernel reductions induce the same rich structure between $\PEq$ and $\NPEq$ as do many-one reductions between $\P$ and $\NP$?
% task (focus on author) why me? - what was undertaken to address the need
We adapt a proof of Ladner's theorem from \autocite{df03} (which has been attributed to Russell~Impagliazzo) to classes of equivalence problems;
% object (focus on document) why this document - what the document covers
this section details that adaptation.

% Summary
%
% findings (focus on author) what? - what the work revealed when performing the task
The main theorem of this section is the existence of $\NPEq$-intermediary problems under the assumption that $\PEq \neq \NPEq$ (which is equivalent to the assumption $\P \neq \NP$ by \autoref{thm:pnppeqnpeq}).
%% Although this theorem relies on the assumption that there is an $\NPEq$-complete problems, no such assumption is required to show the existence of intermediary problems between, say, $\PHEq$ and $\PSPACEEq$ (\autoref{cor:pspace}).
% conclusion (focus on readers) so what? - what the findings mean for the audience
We conclude that even though kernel reductions are strictly weaker than many-one reductions, they still preserve the hierarchies of problems of various computational complexities we expect from our understanding of traditional complexity classes.
% perspective (focus on anyone) what now? - what should be done next
The graph isomorphism problem, as one of the few candidates for an $\NP$-intermediary problem, may be the best candidate for an $\NPEq$-intermediary problem as well.

This proof of Ladner's theorem for equivalence relations is a delayed diagonalization via progressive padding.
First we define the equivalence relation performing the diagonalization and the corresponding padding function, then we show that this problem is neither in $\PEq$ nor $\NPEq$-complete.

\begin{construction}\label{con:diag}
  Let $K$ be an $\NP$-complete equivalence relation.
  We know that such equivalence relations exist by \autoref{thm:npceqrel}.
  Define the equivalence relation $R$ by
  \begin{equation*}
    R = \left\{\left\langle x 0 1^{p(n) - n - 1}, y 0 1^{p(n) - n - 1}\right\rangle \, \middle| \, \pair{x}{y} \in K \text{ and } |x| = |y| = n\right\},
  \end{equation*}
  where $p$ is a padding function that will be defined below.
  The equivalence relation $R$ is a padded version of $K$.

  Our goal is to define the function $p$ so that $R$ is not too hard and not too easy: it's output should be large enough that $R$ is not $\NPEq$-complete but not so large that $R$ is in $\P$.
  For this we need an enumeration of each polynomially clocked Turing machine, $\{M_i\}_i$, where machine $M_i$ halts within time $i n^i$ on inputs of length $n$.
  For any pair of strings $x$ and $y$, we say a Turing machine $M$ \emph{disagrees with $R$ on $\pair{x}{y}$} if
  \begin{itemize}
  \item $M(\pair{x}{y})$ accepts and $\pair{x}{y} \notin R$, or
  \item $M(\pair{x}{y})$ rejects and $\pair{x}{y} \in R$.
  \end{itemize}
  We define $p$ for each positive integer $n$ by the following iterative process (and thus we implicitly define $R$ iteratively as well).
  Initially, let $i = 1$, then perform the following steps for each $n$ in order.
  \begin{itemize}
  \item Define $p(n)$ to be $n^i$.
  \item
    Check if there is any pair of strings $x$ and $y$, each of length at most $\log \log n$, such that $M_i$ disagrees with $R$ on $\pair{x}{y}$.
    If any such pair exists \emph{and} $\floor{\log \log n}$ is an integer not already seen, then increment $i$.\qedhere
  \end{itemize}
\end{construction}

The following three lemmas prove that this problem is of intermediate complexity if $\PEq \neq \NPEq$.

\begin{lemma}
  The function $p$ in \autoref{con:diag} is computable in time polynomial in $n$.
\end{lemma}
\begin{proof}
  Computing $p(n)$ requires computing $p(1)$, $p(2)$, $\dotsc$, $p(n - 1)$; if each of these $n - 1$ computations takes a polynomial amount of time, the total time required to compute $f(n)$ remains polynomial in $n$, by induction.
  Since the strings $x$ and $y$ are of length at most $\log \log n$, the total number of iterations required to test all pairs of strings is polynomial in $n$.
  The simulation of $M_i$ is computable in time $i (\log \log n)^i$, but $i$ is at most $\log \log n$, since $i$ can only be incremented at most $\log \log n$ times.
  Using the fact that a polynomial in $\log \log n$ is bounded above by $O(\log n)$,
  \begin{align*}
    i (\log \log n)^i & = \log \log n (\log \log n)^{\log \log n} \\
    & = 2^{(\log \log \log n)^2 \log \log n } \\
    & \leq 2^{\log^3 \log n} \\
    & = 2^{O(\log n)} \\
    & = \poly(n),
  \end{align*}
  so the machine $M_i$ runs in time polynomial in $n$.
  The language $R$ is in $\NPEq$ because it is a padded version of the language $K$, which is in $\NPEq$.
  Since $\NP \subseteq \EXP$ and the inputs $x$ and $y$ are each of length $\log \log n$, membership in $R$ can be determined in time polynomial in $n$.
  (Even though the definition of $R$ requires $p$ to be defined, $p$ is already defined for strings of length less than $n$, including the strings $x$ and $y$.)
  Since each step can be performed in polynomial time and there are at most $n$ iterations required when defining $p(n)$, we conclude that $p$ is computable in time polynomial in $n$.
\end{proof}

\begin{lemma}
  Suppose $R$ is the equivalence relation in \autoref{con:diag}.
  If $\PEq \neq \NPEq$, then $R \notin \PEq$.
\end{lemma}
\begin{proof}
  Assume with the intention of producing a contradiction that $R \in \PEq$.
  Thus there is a natural number $i$ such that $M_i$ decides $R$.
  For sufficiently large $n$, the machine $M_i$ never disagrees with $R$, so $p(n) = n^i$ for all sufficiently large $n$.
  Assuming without loss of generality that the string $x$ and $y$ are each of length $n$, this yields a polynomial-time kernel reduction from $K$ to $R$ via the function $\pair{x}{y} \mapsto \left\langle x 0 1^{p(n) - n - 1}, y 0 1^{p(n) - n - 1}\right\rangle$.
  This mapping is polynomial-time computable because $p(n) = n^i$ for all sufficiently large $n$, and $i$ does not depend on $n$.
  Since $\PEq$ is closed under polynomial-time kernel reductions, $K$ is in $\PEq$, and hence $\PEq = \NPEq$, since $K$ is $\NPEq$-complete.
  This is a contradiction with the assumption that $\PEq \neq \NPEq$, hence $R \notin \PEq$.
\end{proof}

\begin{lemma}
  Suppose $R$ is the equivalence relation in \autoref{con:diag}.
  If $\PEq \neq \NPEq$, then $R$ is not $\NPEq$-complete.
\end{lemma}
\begin{proof}
  Assume with the intention of producing a contradiction that $R$ is $\NPEq$-complete.
  Thus $K \kr R$, so there is a function $f$ such that $f$ halts within $n^j$ steps and for each string $x$ and $y$, we have $\pair{x}{y} \in K$ if and only if $\pair{f(x)}{f(y)} \in R$.
  If the image of $f$ were finite, then $R$ would have a constant number of equivalence classes.
  In this case, $R$ would be in $\PEq$, and since $\PEq$ is closed under polynomial-time kernel reductions, $K$ would be in $\PEq$ as well, a contradiction with the hypothesis that $\PEq \neq \NPEq$.

  Suppose the image of $f$ is infinite.
  We know $f(w)$ must be of the form $w01^{p(|w|) - n - 1}$ for each string $w$ of length $n$, so the length of $f(w)$ is $p(|w|)$.
  There is a natural number $n_0$ such that for each $n \geq n_0$, there is a positive integer $k$ such that $k$ is greater than $j$ and for each string $w$ of length $n$, we have $|f(w)| = p(n) = n^k$.
  (The integer $k$ is strictly greater than $j$, since if it were less than or equal to $j$, the image of $f$ would be finite.)
  Now we can construct a polynomial-time algorithm for $K$.
  Assume without loss of generality that all inputs are pairs of strings of equal length.
  On inputs of the form $\pair{x}{y}$, proceed as follows.
  \begin{itemize}
  \item If $|x| < n_0$ (or equivalently $|y| < n_0$), decide whether $\pair{x}{y} \in K$ by examining a hardcoded lookup table for strings of length less than $n_0$.
  \item Compute $f(x)$ and $f(y)$.
  \item If either $|f(x)|$ or $|f(y)|$ is not in the range of $p$, reject.
  \item
    Suppose $f(x) = x' 0 1^{p(m) - m - 1}$ and $f(y) = y' 0 1^{p(m) - m - 1}$, where $|x'| = |y'| = m$.
    Invoke this algorithm recursively on input $\pair{x'}{y'}$.
  \end{itemize}

  Assuming for now that $x'$ and $y'$ are shorter than $x$ and $y$.
  Then the correctness of this algorithm follows from the fact that
  \begin{equation*}
    \pair{x}{y} \in K \iff \pair{f(x)}{f(y)} \in R \iff \pair{x'}{y'} \in K.
  \end{equation*}
  Since the length of the inputs to the algorithm decrease on each recursive invocation, there are at most $n$ recursive calls on inputs of pairs of strings of length $n$.
  Eventually the solution can be found in the hardcoded lookup table (the base case of the recursion).
  Each recursive invocation of the algorithm other than the base case requires computing $f$ on an input of length $n$ (twice), which can be done in polynomial time.
  Thus the overall time required for this algorithm is polynomial in $n$.
  This proves that $K \in \P$ and thus $\P = \NP$.
  Since $\P = \NP$ if and only if $\PEq = \NPEq$, we have a contradiction.

  Finally, we prove that $|x'| < |x|$ (the proof that $|y'| < |y|$ is the same), which we postponed from the previous paragraph.
  Due to its time bound, $|f(x)| \leq |x|^j$ for any string $x$.
  By assumption, $p(|x'|) = |x'|^k$.
  By construction, $p(|x'|) = |f(x)|$
  Combining these three relations yields the inequality
  \begin{equation*}
    |x'|^k = p(|x'|) = |f(x)| \leq |x|^j,
  \end{equation*}
  so $|x'| \leq |x|^{j / k} < |x|$, since $k > j$ and lengths must be natural numbers.
\end{proof}

Combining the preceding three lemmas yields Ladner's theorem for classes of equivalence relations.

\begin{theorem}\label{thm:intermediary}
  If $\PEq \neq \NPEq$, then there is an equivalence relation in $\NPEq$ that is neither in $\PEq$ nor $\NPEq$-complete.
\end{theorem}

This technique can be generalized to other classes of equivalence relations, as long as the underlying machines for the smaller class can be enumerated and the larger class has an equivalence relation that is complete under many-one reductions.
For example, we can produce equivalence relations between the polynomial hierarchy and $\PSPACE$.

\begin{corollary}\label{cor:pspace}
  If $\PHEq \neq \PSPACEEq$, then there is an equivalence relation in $\PSPACEEq$ that is neither in $\PHEq$ nor $\PSPACEEq$-complete.
\end{corollary}
