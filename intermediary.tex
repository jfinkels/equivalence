%% intermediary.tex - NP-intermediary equivalence relations
%%
%% Copyright 2010, 2011, 2012, 2014, 2015 Jeffrey Finkelstein.
%%
%% This LaTeX markup document is made available under the terms of the Creative
%% Commons Attribution-ShareAlike 4.0 International License,
%% https://creativecommons.org/licenses/by-sa/4.0/.
\section{Existence of intermediary problems}
\label{sec:intermediary}
% Foreword
%
% context (focus on anyone) why now? - current situation, and why the need is so important
According to the seminal theorem by Ladner \autocite{ladner}, if $\P \neq \NP$, then there are problems of intermediate complexity, in the sense that these problems are neither in $\P$ nor $\NP$-complete.
The theorem does not immediately imply a similar result for equivalence problems, since $\PEq$ is different from $\P$ and $\NPEq$ is different from $\NP$ (specifically, in each case, the latter contains problems that are not equivalence problems).
% need (focus on readers) why you? - why this is relevant to the reader, and why something needed to be done
Do kernel reductions induce the same rich structure between $\PEq$ and $\NPEq$ as do many-one reductions between $\P$ and $\NP$?
% task (focus on author) why me? - what was undertaken to address the need
We adapt Schöning's Uniform Diagonalization Theorem \autocite{schoning}, a generalized version of Ladner's Theorem, to classes of equivalence problems.
The specific presentation we follow is from \autocite{bdg95}.
% object (focus on document) why this document - what the document covers
This section details the proof of that adaptation.

% Summary
%
% findings (focus on author) what? - what the work revealed when performing the task
The main theorem of this section, following as a corollary from the Uniform Diagonalization Theorem for equivalence problems, is as follows (where $\NPEqC$ denotes the set of $\NPEq$-complete problems).
\begin{restatable*}{theorem}{intermediary}\label{thm:intermediary}
  If $\PEq \neq \NPEq$ and $\NPEqC$ is non-empty, then there is an equivalence relation in $\NPEq$ which is in neither $\PEq$ nor $\NPEqC$.
\end{restatable*}
Although this theorem relies on the assumption that $\NPEqC$-complete problems exist, no such assumption is required to show the existence of intermediary problems between, say, $\PHEq$ and $\PSPACEEq$ (\autoref{cor:pspace}).
% conclusion (focus on readers) so what? - what the findings mean for the audience
We conclude that even though kernel reductions are strictly weaker than many-one reductions, they still preserve the hierarchies of problems of various computational complexities we expect from our understanding of traditional complexity classes.
% perspective (focus on anyone) what now? - what should be done next
The graph isomorphism problem, as one of the few candidates for an $\NP$-intermediary problem, may be the best candidate for an $\NPEq$-intermediary problem as well.

First we need to provide some technical definitions.

\begin{definition}
  A class of languages $\mathcal{C}$ is \defn{closed under finite variations} if and only if $A\in \mathcal{C}$ and $A\symdiff B$ (the symmetric difference of $A$ and $B$) is finite implies $B\in \mathcal{C}$ for all $B$.
\end{definition}

\begin{definition}
  The \defn{kernel join} of two equivalence relations $R$ and $S$, denoted $R \kj S$, is defined by
  \begin{equation*}
    R \kj S = \lb \pair{x0}{y0} \st \pair{x}{y} \in R \rb \cup \lb \pair{x1}{y1} \st \pair{x}{y} \in S \rb \cup \{(\lambda, \lambda)\}.
  \end{equation*}
\end{definition}

The pair of empty strings $(\lambda, \lambda)$ is required in the above definition so that the equivalence classes of $R \kj S$ partition the set of all finite strings $\Sigma^*$.

\begin{lemma}\label{lem:join}
  If $R$ and $S$ are equivalence relations, then $R \kj S$ is an equivalence relation.
\end{lemma}
\begin{proof}
  The proof follows immediately from the definitions.
  %% Since the equivalence class of the empty string, $[\lambda]$, is a singleton set, each of the three properties required of an equivalence relation hold for that set.
  %% Thus we need only consider nonempty strings.
  %% Let $x$, $y$ and $z$ be nonempty strings in $\Sigma^*$.
  %%
  %% Since $x$ is non-empty, $x = x'0$ or $x = x'1$.
  %% If $x = x'0$, then $\pair{x'}{x'} \in R$ since $R$ is reflexive.
  %% If $x = x'1$, then $\pair{x'}{x'} \in S$ since $R$ is reflexive.
  %% Hence $\pair{x}{x}\in R\kj S$ regardless of the last bit of $x$.
  %% Therefore $R \kj S$ is reflexive.
  %%
  %% Suppose $\pair{x}{y}\in R\kj S$.
  %% Then either $x=x'0$, $y=y'0$ and $\pair{x'}{y'}\in R$ or $x=x'1$, $y=y'1$ and $\pair{x'}{y'}\in S$.
  %% In either case, symmetry follows from the symmetry of $R$ or $S$.
  %%
  %% Suppose $\pair{x}{y}$ and $\pair{y}{z}$ are both in $R\kj S$.
  %% In the case that $x=x'0$, $y=y'0$ and $\pair{x'}{y'}\in R$, and that $z=z'0$ and $\pair{y'}{z'}\in R$, then by the transitivity of $R$, $\pair{x'}{z'}\in R$, so $\pair{x}{z}\in R\kj S$.
  %% The argument is similar in the case that $x=x'1$, $y=y'1$ and $z=z'1$.
  %% It is a contradiction for the other two cases to exist, since $y$ cannot be equal to both $y'0$ and $y'1$.
  %%
  %% Since $R\kj S$ is reflexive, symmetric and transitive, $R\kj S$ is an equivalence relation.
\end{proof}

\begin{proposition}\label{prop:symdiff}
  Symmetric difference of equivalence relations preserves symmetry.
\end{proposition}
\begin{proof}
  Let $R$ and $S$ be equivalence relations.
  Let $(x,y)\in(R\symdiff S)$.
  In the case that $(x,y)\in R$ and $(x,y)\notin S$, then $(y,x)\in R$.
  If $(y,x)$ were in $S$, then $(x,y)$ would also be in $S$, by symmetry, but this is a contradiction.
  Hence $(y,x)\in R$ and $(y,x)\notin S$.
  The argument for the other case is symmetric.
  Therefore $(x,y)\in(R\symdiff S)\implies (y,x)\in(R\symdiff S)$.
\end{proof}

\begin{definition}
  Let $r\colon\mathbb{N}\to\mathbb{N}$ be a computable function such that $r(m)>m$ for all $m$.
  Define the set $G[r]$ as
  \begin{displaymath}
    G[r]=\lb x\in\Sigma^* \st r^n(0)\leq|x|<r^{n+1}(0) \plain{for some even} n\rb
  \end{displaymath}
  where $r^n(m)$ denotes the $n$-fold application of $r$ to $m$:
  \begin{displaymath}
    \overbrace{r\circ r\circ r\circ\cdots\circ r}^{n \plain{times}}(m)
  \end{displaymath}
  $G[r]$ is called the \defn{gap language} generated by $r$.
\end{definition}

\begin{lemma}\label{lem:gap_p}
  If $r$ is time constructible, then $G[r]\in\P$.
\end{lemma}
\begin{proof}
  %% TODO add theorem number in this citation
  The proof can be found in \autocite{bdg95}.
\end{proof}

We will denote the Cartesian product $G[r]\times G[r]$ by the slightly more succinct ${G[r]}^2$, and $\overline{G[r]}\times\overline{G[r]}$ by $\overline{G[r]}^2$.
Elements of ${G[r]}^2$ are pairs of strings whose lengths are in the ``even gaps'' of $r$, while elements of $\overline{G[r]}^2$ are pairs of strings whose lengths are in the ``odd gaps'' of $r$.

\begin{lemma}
  ${G[r]}^2$ and $\overline{G[r]}^2$ are \defn{partial equivalence relations} (that is, they are symmetric and transitive).
\end{lemma}
\begin{proof}
  The proof follows immediately from the definitions.
  %% We will prove the theorem for ${G[r]}^2$; a symmetric argument proves the theorem for $\overline{G[r]}^2$.
  %%
  %% Let $x,y\in\Sigma^*$.
  %% Suppose $\pair{x}{y}\in {G[r]}^2$, so the lengths of $x$ and $y$ are both in an even gap of $r$.
  %% Then $\pair{y}{x}\in{G[r]}^2$, so ${G[r]}^2$ is symmetric.
  %% Now let $z\in\Sigma^*$ and suppose also that $\pair{y}{z}\in {G[r]}^2$.
  %% Then $y$ and $z$ are both in an even gap of $r$, so $x$, $y$ and $z$ are all in some even gap of $r$, and hence $\pair{x}{z}\in {G[r]}^2$.
  %% Therefore ${G[r]}^2$ is transitive.
\end{proof}

We are now prepared to prove the main technical theorem which will allow us to construct an equivalence relation which is ``between'' two complexity classes.

\begin{theorem}\label{thm:diag}
  Let $R_1$ and $R_2$ be decidable equivalence relations, and let $\mathcal{C}_1$ and $\mathcal{C}_2$ be classes of decidable equivalence relations such that
  \begin{enumerate}
  \item $R_1\notin\mathcal{C}_1$,
  \item $R_2\notin\mathcal{C}_2$,
  \item $\mathcal{C}_1$ and $\mathcal{C}_2$ are computably enumerable,
  \item $\mathcal{C}_1$ and $\mathcal{C}_2$ are closed under finite variations.
  \end{enumerate}
  Then there exists a decidable equivalence relation $R$ such that
  \begin{enumerate}
  \item $R\notin \mathcal{C}_1$,
  \item $R\notin \mathcal{C}_2$,
  \item $R\kr R_1\kj R_2$.
  \end{enumerate}
\end{theorem}
\begin{proof}
  Let $P_1, P_2, \ldots$ and $Q_1, Q_2, \ldots$ be enumerations of Turing machines deciding the languages in $\mathcal{C}_1$ and $\mathcal{C}_2$ respectively.
  Define the functions
  \begin{eqnarray*}
    r_1(n)=\underset{i\leq n}{max}\{|z_{i,n}|\}+1 & \text{and} &
    r_2(n)=\underset{i\leq n}{max}\{|z'_{i,n}|\}+1
  \end{eqnarray*}
  where $z_{i,n}$ is the smallest word in $\Sigma^*$ such that there exists an $x\in\Sigma^*$, with $n<|x|\leq|z_{i,n}|$, such that $\pair{z_{i,n}}{x}\in(L(P_i)\symdiff R_1)$, and $z'_{i,n}$ is the smallest word in $\Sigma^*$ such that there exists an $x'\in\Sigma^*$, with $n<|x'|\leq|z'_{i,n}|$, such that $\pair{z'_{i,n}}{x'}\in(L(Q_i)\symdiff R_2)$.
  Note that it also suffices to find an $x$ and $x'$ such that $\pair{x}{z_{i,n}}\in(L(P_i)\symdiff R_1)$ and $\pair{x'}{z'_{i,n}}\in(L(Q_i)\symdiff R_2)$, since symmetric difference on equivalence relations preserves symmetry by \autoref{prop:symdiff}.
  The more important requirement is that $|x|\leq|z_{i,n}|$, since we will require below that both $x$ and $z_{i,n}$ are in the same gap of a specific function.

  We claim that $z_{i,n}$ and $z'_{i,n}$ always exist.
  Assume that no such $z_{i,n}$ exists, so there are no words such that there exists an $x\in\Sigma^*$, with $n<|x|\leq|z_{i,n}|$, such that $\pair{z_{i,n}}{x}\in(L(P_i)\symdiff R_1)$.
  Therefore, there are no pairs in $(L(P_i)\symdiff R_1)$ with both elements of length greater than $n$.
  Then there are a finite number of pairs in $L(P_i)\symdiff R_1$, so $R_1$ is a finite variation of $L(P_i)$.
  Since $\mathcal{C}_1$ is closed under finite variations, $R_1\in\mathcal{C}_1$.
  This is a contradiction with the hypothesis that $R_1\notin\mathcal{C}_1$.
  Therefore such a $z_{i,n}$ always exists.
  The argument that $z'_{i,n}$ always exists is similar.

  Since $L(P_i)$ and $L(Q_i)$ are decidable for all $i$, and since $R_1$ and $R_2$ are decidable, so are $L(P_i)\symdiff R_1$ and $L(Q_i)\symdiff R_2$.
  For each $n$, there is a procedure which always halts and which computes $z_{i,n}$.
  A similar procedure computes $z'_{i,n}$.
  The procedure which computes the maximum of a finite set of numbers and which adds one to that value always halts as well, so $r_1$ and $r_2$ are total computable functions.

  Let $r\ge \max(r_1,r_2)$ be a non-decreasing time constructible function (the proof that such a function exists is left as an exercise to the reader).
  % The proof can be found in Lemma 2.3 on p. 46 of [BDG95].
  Now for all $n$ and all $i\leq n$, each element of the pair $\pair{z_{i,n}}{x}$ has length between $n$ and $r_1(n)$, by construction.
  The same is true for $\pair{z'_{i,n}}{x'}$ between $n$ and $r_2(n)$.
  Notice that $\pair{z_{i,n}}{x}$ and $\pair{z'_{i,n}}{x'}$ are ``witnesses'' that $R_1\neq L(P_i)$ and $R_2\neq L(Q_i)$ respectively.
  Hence for all $n$, there are some witnesses between $n$ and $r(n)$ that for all $i\leq n$, $R_1\neq L(P_i)$ and $R_2\neq L(Q_i)$.

  Define $R=({G[r]}^2\cap R_1)\cup(\overline{G[r]}^2\cap R_2)$, so $R$ is equal to pairs of $R_1$ in the ``even gaps'' of $r$ and $R$ is equal to pairs of $R_2$ in the ``odd gaps'' of $r$.
  It remains to show that $R$ is an equivalence relation which satisfies the properties stated in the theorem.

  First we show that $R$ is indeed an equivalence relation.
  Symmetry and transitivity follow from the symmetry and transitivity of $R_1$, $R_2$, ${G[r]}^2$ and $\overline{G[r]}^2$.
  To show reflexivity, suppose $x\in G[r]$.
  Hence $\pair{x}{x}\in {G[r]}^2$.
  Since $R_1$ is an equivalence relation, $\pair{x}{x}\in R_1$.
  Therefore $\pair{x}{x}\in({G[r]}^2\cap R_1)$, so $\pair{x}{x}\in R$.
  The argument for the case that $x\in\overline{G[r]}$ is similar.
  Therefore $R$ is reflexive, symmetric and transitive.

  Next we show that $R\notin\mathcal{C}_1$.
  The argument which proves $R\notin\mathcal{C}_2$ is symmetric.
  Assume $R\in\mathcal{C}_1$ in order to produce a contradiction.
  Then there exists an $i$ such that $R=L(P_i)$.
  Let $m$ be an even integer such that $r^m(0)\geq i$.
  By construction, there exists a pair $\pair{x}{z}$ such that $r^m(0)\leq|x|\leq|z|<r^{m+1}(0)$ and $\pair{x}{z}\in(L(P_i)\symdiff R_1)$.
  Since $m$ is even, $\pair{x}{z}\in {G[r]}^2$.
  Since $R$ is equal to $R_1$ where it coincides with ${G[r]}^2$, then $\pair{x}{z}\in(L(P_i)\symdiff R)$.
  This is a contradiction with the hypothesis that $R=L(P_i)$.
  Therefore $R\notin\mathcal{C}_1$.

  Finally, we show that $R\kr R_1\kj R_2$.
  First, we note that $R_1\kj R_2$ is an equivalence relation by \autoref{lem:join}, so a kernel reduction here is syntactically possible.
  Since $G[r]\in\P$ by \autoref{lem:gap_p}, the function defined by
  \begin{displaymath}
    f(x)=
    \begin{cases}
      x0 & \text{if}\, x\in G[r]\\
      x1 & \text{if}\, x\notin G[r]\\
    \end{cases}
  \end{displaymath}
  is a polynomial-time computable function.
  
  To show that $f$ computes the reduction from $R$ to $R_1\kj R_2$ correctly, suppose first that $\pair{x}{y}\in R$, so $\pair{x}{y}$ is in either $({G[r]}^2\cap R_1)$ or $(\overline{G[r]}^2\cap R_2)$.
  In the former case, both $x$ and $y$ are in $G[r]$, so $f(x)=x0$ and $f(y)=y0$, and both $x$ and $y$ are in $R_1$, so $\pair{x0}{y0}=\pair{f(x)}{f(y)}\in R_1$.
  The argument for the latter case is symmetric.

  For the converse, suppose $\pair{f(x)}{f(y)}\in R_1\kj R_2$.
  Then $f(x)$ and $f(y)$ either both end with $0$ or both end with $1$.
  In the case that both end with $0$, then there exist some strings $w_x$ and $w_y$ such that $f(x)=w_x0$, $f(y)=w_y0$ and $\pair{w_x}{w_y}\in R_1$.
  By construction of $f$, $w_x$ must equal the input $x$ and $w_y$ must equal the input $y$, so $\pair{x}{y}\in R_1$.
  Also by construction, $f(x)=x0$ if and only if $x\in G[x]$ and $f(y)=y0$ if and only if $y\in G[x]$, so $\pair{x}{y}\in{G[r]}^2$.
  Hence $\pair{x}{y}\in({G[r]}^2\cap R_1)\subseteq R$.
  The argument for the case that both $f(x)$ and $f(y)$ end with $1$ is symmetric, and shows that $\pair{x}{y}\in(\overline{G[r]}^2\cap R_2)\subseteq R$.
  Therefore $\pair{x}{y}\in R$ if and only if $\pair{f(x)}{f(y)}\in R_1\kj R_2$, so $f$ correctly computes the reduction from $R$ to $R_1\kj R_2$.

  Since we have shown that the equivalence relation $R$ satisfies the properties in the statement of the theorem, this concludes the proof.
\end{proof}

We would now like to show that the result of this theorem holds when $\mathcal{C}_1=\PEq$ and $\mathcal{C}_2=\NPEqC$.

\begin{proposition}\label{prop:npeqc}
  If $\PEq \neq \NPEq$, then $\PEq \cap \NPEqC = \emptyset$.
\end{proposition}
\begin{proof}
  If $\NPEqC$ is empty, then $\PEq \cap \NPEqC = \emptyset$ unconditionally.
  Assume $\NPEqC$ is nonempty, and $\PEq \cap \NPEqC \neq \emptyset$.
  Let $R$ be the \NPEq-complete equivalence relation which is also in \PEq.
  Then all problems in $\NPEq$ can be kernel reduced to $R$ in polynomial time, and $R$ can be decided in polynomial time.
  Therefore, $\PEq=\NPEq$.
  This is a contradiction with the hypothesis.
  Therefore $\PEq\cap\NPEqC=\emptyset$.
\end{proof}

A reminder: as stated in the introduction, polynomial-time kernel reductions compose, and $\NPEq$ is closed under polynomial-time kernel reductions.
We use these facts in the proof of the following theorem.

\intermediary
\begin{proof}
  By \autoref{prop:npeqc}, if $\PEq \neq \NPEq$, then $\PEq \cap \NPEqC = \emptyset$.
  Let $S$ be an \NPEq-complete problem.
  Choose $R_1=S$, $R_2=\emptyset$, $\mathcal{C}_1=\PEq$ and $\mathcal{C}_2=\NPEqC$.
  Note that since $\P$ and $\NPC$ are computably enumerable, so are $\PEq$ and $\NPEqC$.
  Then by \autoref{thm:diag}, there exists an equivalence relation $R$ which is in neither $\NPEqC$ nor $\PEq$, but which kernel reduces to $S\kj\emptyset$.
  Since $S\kj\emptyset$ trivially kernel reduces to $S$, and since polynomial-time kernel reductions compose, $R\kr S$.
  Since $\NPEq$ is closed under polynomial-time kernel reductions, $R\in\NPEq$.
\end{proof}

This also introduces a host of similar corollaries showing the existence of intermediary problems for other classes of equivalence relations which have $\kr$-complete problems; for more examples, see the original paper by Sch\"{o}ning \autocite{schoning}.
In particular, we can make the following stronger statement using \autoref{cor:hardproblems}.
(Note that $\PHEq$ is the class of equivalence relations in $\PH$, the polynomial hierarchy.)
\begin{corollary}\label{cor:pspace}
  If $\PSPACEEq\neq\PHEq$ then there exists an equivalence relation in $\PSPACEEq$ which is neither in $\PHEq$ nor $\PSPACEEq$-complete.
\end{corollary}
