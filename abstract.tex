%% abstract.tex - a brief summary of the work
%%
%% Copyright 2010, 2011, 2012, 2014 Jeffrey Finkelstein.
%%
%% This LaTeX markup document is made available under the terms of the Creative
%% Commons Attribution-ShareAlike 4.0 International License,
%% https://creativecommons.org/licenses/by-sa/4.0/.
\begin{abstract}
  In this paper, we examine a recently introduced type of effective reduction which applies solely to problems of equivalence or isomorphism: the ``kernel reduction''.
  Specifically, we examine reductions among languages in the complexity class consisting of all languages induced by equivalence relations for which membership can be decided by a non-deterministic polynomial time Turing machine.
  This class is called $\NPEq$; the definitions for $\PEq$ and $\coNPEq$ are analagous.

  We prove a general theorem which provides a problem which is hard under polynomial time kernel reductions for several classes of equivalence relations, including $\SKPEq$ and $\PSPACEEq$.
  In fact, such a problem is complete for $\PSPACEEq$ under polynomial time kernel reductions.
  We also show that if there is a complete problem under kernel reductions in \NPEq, then that problem is also complete under many-one reductions in \NP.
  Finally we use a proof of Ladner's theorem to show that if $\PEq\neq\NPEq$ and there are problems in $\NPEq$ which are complete under polynomial time kernel reductions then there are \NPEq-intermediary problems---problems which are in \NPEq, but not complete under kernel reductions and not in $\PEq$.
\end{abstract}
