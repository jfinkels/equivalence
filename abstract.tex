%%%%
%% abstract.tex
%%
%% Copyright 2011 Jeffrey Finkelstein
%%
%% Except where otherwise noted, this work is made available under the terms of
%% the Creative Commons Attribution-ShareAlike 3.0 license,
%% http://creativecommons.org/licenses/by-sa/3.0/.
%%
%% You are free:
%%    * to Share — to copy, distribute and transmit the work
%%    * to Remix — to adapt the work
%% Under the following conditions:
%%    * Attribution — You must attribute the work in the manner specified by
%%    the author or licensor (but not in any way that suggests that they
%%    endorse you or your use of the work).
%%    * Share Alike — If you alter, transform, or build upon this work, you may
%%    distribute the resulting work only under the same, similar or a 
%%    compatible license.
%%    * For any reuse or distribution, you must make clear to others the 
%%    license terms of this work. The best way to do this is with a link to the
%%    web page http://creativecommons.org/licenses/by-sa/3.0/.
%%    * Any of the above conditions can be waived if you get permission from
%%    the copyright holder.
%%    * Nothing in this license impairs or restricts the author's moral rights.
%%%%
\abstract{%
  In this paper, we examine a recently introduced type of reduction which applies solely to problems of equivalence or isomorphism: the ``kernel reduction''.
  Specifically, we examine reductions among languages in the complexity class consisting of all languages induced by equivalence relations for which membership can be decided by a non-deterministic polynomial time Turing machine.
  This class is called $\NPEq$; the definitions for $\PEq$ and $\coNPEq$ are analagous.

  We prove a general theorem which provides a complete problem under polynomial time kernel reductions for several classes of equivalence relations, including $\SKPEq$ and $\PSPACEEq$.
  We also show that if there is a complete problem under kernel reductions in \NPEq, then that problem is also complete under many-one reductions in \NP.
  Finally we use a proof of Ladner's theorem to show that if $\PEq\neq\NPEq$ then there are \NPEq-intermediary problems---problems which are in \NPEq, but not complete under kernel reductions and not in $\PEq$.}
