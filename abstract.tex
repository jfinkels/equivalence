%% abstract.tex - a brief summary of the work
%%
%% Copyright 2010, 2011, 2012, 2014 Jeffrey Finkelstein.
%%
%% This LaTeX markup document is made available under the terms of the Creative
%% Commons Attribution-ShareAlike 4.0 International License,
%% https://creativecommons.org/licenses/by-sa/4.0/.
\begin{abstract}
  % Foreword
  %
  % context (focus on anyone) why now? - current situation, and why the need is so important
  Today, the computational complexity of equivalence problems such as the graph isomorphism problem and the Boolean formula equivalence problem remain only partially understood.
  One of the most important tools for determining the (relative) difficulty of a computational problem is the many-one reduction, which provides a way to encode an instance of one problem into an instance of another.
  In equivalence problems, the goal is to determine if a pair of strings is related, so a many-one reduction with access to the entire pair may be too powerful.
  % need (focus on readers) why you? - why this is relevant to the reader, and why something needed to be done
  A recently introduced type of reduction, the \emph{kernel reduction}, defined only on equivalence problems, allows the transformation of each string in the pair independently.
  Understanding the limitations of the kernel reduction as compared with the many-one reduction improves our understanding of the limitations of computers in solving problems of equivalence.
  % task (focus on author) why me? - what was undertaken to address the need
  We investigated not only these limitations, but also whether classes of equivalence problems have complete problems under kernel reductions.
  % object (focus on document) why this document - what the document covers
  This paper provides a detailed collection of basic results about kernel reductions.

  % Summary
  %
  % findings (focus on author) what? - what the work revealed when performing the task
  After exploring possible definitions of complexity classes of equivalence relations, we prove that polynomial time kernel reductions are strictly less powerful than polynomial time many-one reductions.
  We also provide sufficient conditions for complete problems under kernel reductions, show that completeness under kernel reductions can sometimes imply completeness under many-one reductions, and finally prove that equivalence problems of intermediate difficulty can exist under the right conditions.
  % conclusion (focus on readers) so what? - what the findings mean for the audience
  Though kernel reductions share some basic properties with many-one reductions, ultimately the number and size of equivalence classes can prevent the existence of a kernel reduction, regardless of the complexity of the equivalence problem.
  % perspective (focus on anyone) what now? - what should be done next
  \todo{Need a final ``perspectives'' sentence here, focusing on anyone reading it: what should be done next?}
\end{abstract}
