\abstract{%
  In this paper, we examine a recently introduced type of reduction which applies solely to problems of equivalence or isomorphism: the ``kernel reduction''.
  Specifically, we examine reductions among languages in the complexity class consisting of all languages induced by equivalence relations for which membership can be decided by a non-deterministic polynomial time Turing machine.
  This class is called $\NPEq$; the definitions for $\PEq$ and $\coNPEq$ are analagous.

  We show that there is a complete problem under kernel reductions in \NPcoNPEq.
  We also show that if there is a complete problem under kernel reductions in \NPEq, then that problem is also complete under many-one reductions in \NP.
  Finally we use a proof of Ladner's theorem to show that if $\PEq\neq\NPEq$ then there are \NPEq-intermediary problems---problems which are in \NPEq, but not complete under kernel reductions and not in $\PEq$.}
