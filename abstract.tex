%% abstract.tex - a brief summary of the work
%%
%% Copyright 2010, 2011, 2012, 2014, 2015 Jeffrey Finkelstein.
%%
%% This LaTeX markup document is made available under the terms of the Creative
%% Commons Attribution-ShareAlike 4.0 International License,
%% https://creativecommons.org/licenses/by-sa/4.0/.
\begin{abstract}
  In the framework of computational complexity and in an effort to define a more natural reduction for problems of equivalence, we investigate the recently introduced \emph{kernel reduction}, a reduction that operates on each element of a pair independently.
  %% This reduction, defined only on equivalence problems, differs from the usual many-one reduction in that it transforms each string in a pair independently; for example, a kernel reduction from undirected to directed graph isomorphism is a function from undirected to directed graphs whereas a many-one reduction is a function from pairs of undirected graphs to pairs of directed graphs.
  This paper details the limitations and uses of kernel reductions.
  % Summary
  %
  % findings (focus on author) what? - what the work revealed when performing the task
  We show that kernel reductions are weaker than many-one reductions and provide conditions under which complete problems exist.
  % We prove that for polynomial time, kernel reductions are strictly weaker than many-one reductions.
  % We also provide sufficient conditions for completeness under kernel reductions, show the relationship between kernel and many-one completeness, and prove the existence of equivalence problems under intermediate difficulty under a standard assumption.
  % conclusion (focus on readers) so what? - what the findings mean for the audience
  Ultimately, the number and size of equivalence classes can dictate the existence of a kernel reduction. %%, regardless of the complexity of the equivalence problem.
  % perspective (focus on anyone) what now? - what should be done next
  We leave unsolved the unconditional existence of a complete problem under polynomial-time kernel reductions for the standard complexity classes.
  %% The most important open problem we leave unsolved is proving the unconditional existence of a complete problem under kernel reductions for some basic complexity classes that are well-known to have complete problems under many-one reductions.
\end{abstract}
