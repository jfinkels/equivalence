%%%%
%% npeqcompleteness.tex
%%
%% Copyright 2011 Jeffrey Finkelstein
%%
%% Except where otherwise noted, this work is made available under the terms of
%% the Creative Commons Attribution-ShareAlike 3.0 license,
%% http://creativecommons.org/licenses/by-sa/3.0/.
%%
%% You are free:
%%    * to Share — to copy, distribute and transmit the work
%%    * to Remix — to adapt the work
%% Under the following conditions:
%%    * Attribution — You must attribute the work in the manner specified by
%%    the author or licensor (but not in any way that suggests that they
%%    endorse you or your use of the work).
%%    * Share Alike — If you alter, transform, or build upon this work, you may
%%    distribute the resulting work only under the same, similar or a 
%%    compatible license.
%%    * For any reuse or distribution, you must make clear to others the 
%%    license terms of this work. The best way to do this is with a link to the
%%    web page http://creativecommons.org/licenses/by-sa/3.0/.
%%    * Any of the above conditions can be waived if you get permission from
%%    the copyright holder.
%%    * Nothing in this license impairs or restricts the author's moral rights.
%%%%
\section{Completeness in \texorpdfstring{\NPEq}{NPEq} and \texorpdfstring{\NP}{NP}}

In this section we examine the relationship between \NP-complete problems and \NPEq-complete problems.
Below, we will denote the set of \NP-complete languages by \NPC.

We will need some further definitions as well.
If $G$ is a graph, then $V(G)$ is the set of vertices in $G$ and $E(G)$ is the set of edges in $G$.
A \defn{clique} in $G$ is a subset of $V(G)$, $V'$, such that $\forall u,v\in V'$, $(u, v)\in E(G)$.
Recall that the language \lang{CLIQUE}, defined as $\{\pair{G}{k}|G \plain{has a clique of size} k\}$,  is \NP-complete.
If $G$ and $H$ are two graphs, then $G$ \defn{is isomorphic to} $H$ if $\exists\phi\colon V(G)\to V(H)$, a bijection, such that $\forall u,v\in V(G)$, $[(u,v)\in E(G)\iff (\phi(u), \phi(v))\in E(H)]$.
We denote this by $G\cong H$.
The much-studied graph isomorphism problem (\GI), the problem of deciding whether two given graphs are isomorphic, is one of few problems in $\NP$ not known to be either in $\P$ or \NP-complete.

We first show that there are problems in $\NPEq$ which are also \NP-complete (though they may not necessarily be \NPEq-complete).

\begin{theorem}\label{thm:npcnpeq}
  $\NPC\cap\NPEq\neq\emptyset$.
\end{theorem}
\begin{proof}
  Define $A$ by
  \begin{displaymath}
    A=\{\pair{\pair{G}{k}}{\pair{H}{k}}|G\cong H \plain{or} (G \plain{and} H \plain{both have a clique of size} k)\}.
  \end{displaymath}
  It is straightforward to verify that $A$ is an equivalence relation and in \NP, so it is in \NPEq.
  It remains to show that it is \NP-hard.

  The reduction is from the language \lang{CLIQUE} by the mapping $f$ defined by $f(\pair{G}{k})=\pair{\pair{G}{k}}{\pair{K_k}{k}}$ for all graphs $G$ and all $k\in\mathbb{N}$, where $K_k$ is the complete graph on $k$ vertices.
  Constructing the graph $K_k$ is obviously polynomial in $|\pair{G}{k}|$.
  Suppose $\pair{G}{k}\in \lang{CLIQUE}$, so $G$ has a clique of size $k$.
  Then $f(\pair{G}{k})=\pair{\pair{G}{k}}{\pair{K_k}{k}}\in A$ because both $G$ and $K_k$ have a clique of size $k$.
  Suppose $\pair{G}{k}\notin \lang{CLIQUE}$, so $G$ does not have a clique of size $k$.
  Thus $G$ cannot be isomorphic to $K_k$.
  Additionally, although $K_k$ certainly has a clique of size $k$, $G$ does not by hypothesis.
  Therefore $f(\pair{G}{k})\notin A$.

  We have $\pair{G}{k}\in \lang{CLIQUE} \iff f(\pair{G}{k})\in A$, so $\lang{CLIQUE}\mor A$ and hence $A$ is \NP-complete.
\end{proof}

Using this fact we can show that completeness in $\NPEq$ under $\kr$ reductions implies completeness in $\NP$ under $\mor$ reductions.

\begin{corollary}
  If an equivalence relation $R$ is \NPEq-complete then it is also \NP-complete.
\end{corollary}
\begin{proof}
  Let $A$ be an \NP-complete equivalence relation in \NPEq, which exists by the previous theorem.
  $R$ is in $\NP$ by hypothesis, and it is \NP-hard by a reduction from $A$.
  Since $R$ is \NPEq-complete, there exists a polynomial time kernel reduction, call it $f$, from $A$ to $R$.
  The polynomial time many-one reduction from $A$ to $R$ induced by $f$, namely $\pair{x}{y}\mapsto\pair{f(x)}{f(y)}$, proves that $R$ is \NP-complete.
\end{proof}

This corollary provides another proof of \cite[Proposition~8.1]{bcffm}.

\begin{proposition}[\cite{bcffm}, Proposition~8.1]
  If $\GI$ is \NPEq-complete then the polynomial time hierarchy collapses to the second level ($\PH=\STP$).
\end{proposition}
\begin{proof}
  If $\GI$ is \NPEq-complete then it is \NP-complete by the previous corollary.
  This implies the stated collapse (see \cite{schoning87}).
\end{proof}

We currently do not know whether $\NPEq$ has a complete problem under polynomial time kernel reductions.
In \cite{bcffm} the authors prove that there is a complete problem if $\NP=\coNP$.
Of course, under this hypothesis the polynomial hierarchy collapses, so the \NPcoNPEq-complete problem exhibited in \autoref{thm:completeproblem} is also an \NPEq-complete problem.
\begin{openproblem}
  Can we prove that $\NPEq$ has a complete problem unconditionally?
\end{openproblem}

We will for now consider the consequences of the assumption that there exists an \NPEq-complete problem.
In the following theorem, if an equivalence relation $B$ is ``complete under $\kri$ reductions in $\NPEq$'' we mean that every equivalence relation in $\NPEq$ reduces to $B$ by a polynomial time computable kernel reduction which is also injective (that is, ``one-to-one'').

\begin{theorem}
  There exists an \NP-complete language $B$ such that if $B$ is complete under $\kr$ reductions in $\NPEq$, then $B$ is not complete under $\kri$ reductions in $\NPEq$.
\end{theorem}
\begin{proof}
  The language $B$ we construct is a restriction of the language $A$ from the proof of \autoref{thm:npcnpeq}.
  Define $B$ by
  \begin{align*}
    B &= \{\pair{\pair{G}{k}}{\pair{H}{k}}|\pair{\pair{G}{k}}{\pair{H}{k}}\in A \plain{and} |V(G)|=|V(H)|\}
  \end{align*}
  As in the proof of \autoref{thm:npcnpeq}, it is again straightforward to verify that $B$ is an equivalence relation and in \NP, so it is in \NPEq.

  \begin{claim}
    $B$ is \NP-complete.
  \end{claim}
  To show that it is \NP-complete, we show that $A\mor B$.
  The transducer $f$ proceeds as follows on input $\pair{\pair{G}{k}}{\pair{H}{k}}$.

  Let $d$ be the absolute value of $|V(G)|-|V(H)|$.
  If $|V(G)|>|V(H)|$ then add $d$ disconnected vertices to $H$.
  Otherwise if $|V(H)|>|V(G)|$ then add $d$ disconnected vertices to $G$.
  Output the modified graphs in the appropriate encoding, $\pair{\pair{G}{k}}{\pair{H}{k}}$.

  This is obviously computable in polynomial time.
  To prove its correctness, first suppose $\pair{\pair{G}{k}}{\pair{H}{k}}\in A$.
  If $G\cong H$ then $f$ performs the identity transformation, and the output graphs are also isomorphic (since they are the same as the input graphs).
  If $G$ and $H$ both have a clique of size $k$ and $|V(G)|=|V(H)|$ then, again, $f$ performs the identity transformation, and the output graphs satisfy the conditions of $B$.
  If $G$ and $H$ both have a clique of size $k$ and $|V(G)|\neq|V(H)|$ then $f$ will add some disconnected vertices to either $G$ or $H$.
  But the modified graphs will still have the same cliques of size $k$, since $f$ adds only disconnected vertices which do not affect existing cliques.
  Therefore $f(\pair{\pair{G}{k}}{\pair{H}{k}})\in B$.

  Next suppose $f(\pair{\pair{G}{k}}{\pair{H}{k}})\in B$.
  Let the graphs modified by $f$ be called $\hat{G}$ and $\hat{H}$.
  If $f$ added vertices to one of the graphs, then $G$ must not be isomorphic to $H$, since they had a different number of vertices.
  In this case, it must be true that $\hat{G}$ and $\hat{H}$ both have a clique of size $k$ and $|V(\hat{G})|=|V(\hat{H})|$.
  Since the addition vertices are disconnected, they could not have contributed to a clique.
  Therefore the original graphs $G$ and $H$ both have a clique of size $k$.
  If $f$ did not add any vertices during its operation, then it performed the identity transformation on its input (by construction).
  In that case, $G$ and $H$ either are isomorphic or they both have a clique of size $k$ and both have the same number of vertices.
  This means that the output graphs satisfy the conditions of $A$.
  Therefore in all cases $\pair{\pair{G}{k}}{\pair{H}{k}}\in A$.

  This proves $A\mor B$ and hence $B$ is \NP-complete.

  Now, we will consider the equivalence classes in the equivalence relation $R=\{\pair{x}{y}|x$ and $y$ have the same number of $1$s$\}$.
  The proof that this is an equivalence relation is straightforward.
  There are an infinite number of equivalence classes in $R$; specifically, they are $[\lambda]$, $[1]$, $[11]$, $[111]$, etc.
  Each equivalence class is itself infinite as well: if $w\in\Sigma^*$ then $[w]$ contains $w$, $0w$, $00w$, $000w$, etc.

  Next, we will consider the equivalence classes in $B$.
  There are an infinite number of equivalence classes in $B$, as well (because there are an infinite number of graphs, and an infinite number of natural numbers).
  However, in $B$ each equivalence class contains only a finite number of elements.
  If $G$, then $[\pair{G}{k}]_B$ includes all pairs $\pair{H}{k}$, where $G$ is isomorphic to $H$, and all pairs $\pair{J}{k}$, where $G$ and $J$ have the same number of vertices and both have a clique of size $k$.
  Since there are only a finite number of graphs on $|V(G)|$ vertices, and any graph isomorphic to $G$ must have $|V(G)|$ vertices, the number of graphs isomorphic to $G$ is finite.
  Furthermore, the number of graphs with both $|V(G)|$ vertices and a clique of size $k$ is also finite.
  Therefore, the equivalence class $[\pair{G}{k}]_B$ is finite.

  \begin{claim}
    If $B$ is complete under $\kr$ reductions in $\NPEq$ then $B$ is not complete under $\kri$ in \NPEq.
  \end{claim}
  Since $R\in\PEq\subseteq\NPEq$ and $B$ is \NPEq-complete, $R\kr B$.
  By definition, there exists a polynomial time computable function $g$ such that $\pair{x}{y}\in\R\iff \pair{g(x)}{g(y)}\in B$.

  Let $w\in\sigmastar$.
  Then $g(w)=\pair{G}{k}$ for some graph $G$ and some $k\in\mathbb{N}$.
  By the above arguments, $[w]_R$ is infinite and $[g(w)]_B$ is finite.
  By \autoref{lem:image}, $g([w]_R)\subseteq [g(w)]_B$.
  Consider $g|_{[w]_R}$, that is, $g$ restricted to the domain $[w]_R$.
  Then $g|_{[w]_R}$ is a mapping from the infinite set $[w]_R$ to the finite set $[g(w)]_B$.
  By the pigeonhole principle, $g|_{[w]_R}$ is not injective.
  Hence the unrestricted reduction $g$ is not injective, and therefore $B$ is not $\kri$-complete in \NPEq.

  This concludes the proof of the theorem.
\end{proof}
