%% npeqcompleteness.tex - completeness in NP equivalence classes
%%
%% Copyright 2010, 2011, 2012, 2014, 2015 Jeffrey Finkelstein.
%%
%% This LaTeX markup document is made available under the terms of the Creative
%% Commons Attribution-ShareAlike 4.0 International License,
%% https://creativecommons.org/licenses/by-sa/4.0/.
\section
    [Relationship between completeness under kernel and many-one reductions]
    {Relationship between completeness \\ under kernel and many-one reductions}
\label{sec:npeqcompleteness}
% Foreword
%
% context (focus on anyone) why now? - current situation, and why the need is so important
A kernel reduction implies a many-one reduction, but does completeness under kernel reductions imply completeness under many-one reductions?
% need (focus on readers) why you? - why this is relevant to the reader, and why something needed to be done
Since polynomial-time kernel reductions are different from polynomial-time many-one reductions (\autoref{thm:different}), completeness in classes of equivalence problems may differ under these reductions as well.
% task (focus on author) why me? - what was undertaken to address the need
We determine the conditions under which completeness under kernel reductions implies completeness under many-one reductions.
% object (focus on document) why this document - what the document covers
%This section presents some information about the relationship between these two types of reductions.

% Summary
%
% findings (focus on author) what? - what the work revealed when performing the task
We find that completeness under many-one reductions follows as a straightforward consequence of completeness under kernel reductions as long as the relevant complexity class admits a complete problem under many-one reductions.
We also show that the kernel reduction is essentially too weak to allow for completeness under injective (that is, ``one-to-one'') reductions, for combinatorial reasons similar to those in \autoref{sec:limitations}.
Though we prove these results for $\NPEq$, they generalize in a natural way to any ``well-behaved'' complexity class (basically, any class containing a complete problem under many-one reductions).
% conclusion (focus on readers) so what? - what the findings mean for the audience
These results are more indication that when comparing the relative difficulty of equivalence problems, one should attempt to construct a kernel reduction instead of a many-one reduction.
% perspective (focus on anyone) what now? - what should be done next
The potential lack of a complete problem under injective kernel reductions suggests that a conjecture analagous to the Berman--Hartmanis conjecture, which states that all $\NP$-complete problems are isomorphic with respect to many-one reductions, may be false in $\NPEq$.

One can infer the existence of $\NP$-complete equivalence relations from the relation suggested in \autocite[Section~6.2]{fg11}, namely $\{\pair{0\phi}{1\phi} \, | \, \phi \in \textsc{Satisfiability}\}.$
(This relation is not itself an equivalence relation, but can be modified to guarantee the three necessary properties.)
Using this idea, we provide a strategy for constructing a more natural $\NP$-complete equivalence relation from an equivalence relation in $\NP$ and an arbitrary $\NP$-complete property.

Let \textsc{GI} denote the equivalence relation consisting of all pairs of isomorphic graphs.
A property, that is, a Boolean function, $\Pi$ is an \defn{$\NP$-complete property} if $L_\Pi$, the set of all strings for which $\Pi$ is true, is $\NP$-complete.
If, furthermore, the property satisfies $\pair{x}{y} \in R$ implies $\Pi(x) = \Pi(y)$ where $R$ is an equivalence relation, $\Pi$ is called a \emph{property on $R$}.
For example, Hamiltonicity, the property of having a cycle that includes each vertex, is an $\NP$-complete property on \textsc{GI}.

\begin{theorem}\label{thm:npceqrel}
  If $\Pi$ is an \NP-complete property on \textsc{GI}, then the equivalence relation $A$ defined by
  \begin{equation*}
    A = \left\{ \pair{G}{H} \, \middle| \, \pair{G}{H} \in \textsc{GI} \text{ or } (G \in L_\Pi \text{ and } H \in L_\Pi) \right\}
  \end{equation*}
  is an \NP-complete equivalence relation.
\end{theorem}
\begin{proof}
  It is straightforward to prove that $A$ is an equivalence relation, so it remains to show that it is $\NP$-complete.
  The language $A$ is in $\NP$ because both $R$ and $L_\Pi$ are in $\NP$ by hypothesis.
  Thus we need only show that $A$ is $\NP$-hard.

  Let $H$ be a graph satisfying $\Pi$; such a graph must exist because $\Pi$ is $\NP$-complete and therefore there must be at least one graph that satisfies $\Pi$ and at least one that does not (otherwise no many-one reduction to $L_\Pi$ could exist).
  The reduction proving that $A$ is $\NP$-complete is from $L_\Pi$, and the mapping is given by $G \mapsto \pair{G}{H}$.
  This function is computable in linear time; the size of $H$ is constant with respect to the size of $G$.

  Now we show that $G \in L_\Pi$ if and only if $\pair{G}{H} \in A$, for any graph $G$.
  If $G \in L_\Pi$, then $G \in L_\Pi$ and $H \in L_\Pi$, so $\pair{G}{H} \in A$.
  If $\pair{G}{H} \in A$, then either $G \in L_\Pi$ and $H \in L_\Pi$, in which case $G \in L_\Pi$, or $G$ is isomorphic to $H$, in which case $G$ is in $L_\Pi$ because $H$ is.
  In either case $G \in L_\Pi$.
  We conclude that $L_\Pi \mor A$, and so $A$ is an $\NP$-complete equivalence relation.
\end{proof}

\begin{example}\label{ex:npceqrel}
  The language
  \begin{equation*}
    \left\{ \pair{G}{H} \, \middle| \, \pair{G}{H} \in \textsc{GI} \text{ or } G \text{ and } H \text{ have a Hamiltonian cycle}  \right\}
  \end{equation*}
  is an \NP-complete equivalence relation.
\end{example}

There are other ways of constructing a natural $\NP$-complete equivalence relation.
For example, there is a finitely presented group whose word problem is $\NP$-complete \autocite[Corollary~1.1]{sbr02}, and the word problem is already an equivalence relation.
This may be considered ``more natural'' because it does not involve the disjunction of two distinct computational problems, though it lacks the simplicity of our approach.

\begin{example}
  As stated briefly above, \autoref{thm:npceqrel} can be generalized to isomorphism of structures other than graphs and/or larger complexity classes.
  For example, replacing \textsc{GI} with \textsc{FI}, the Boolean formula isomorphism problem, and an $\NP$-complete property on \textsc{GI} with a $\SigmaTwoP$-complete property on \textsc{FI} yields a $\SigmaTwoP$-complete equivalence relation.
\end{example}

\begin{corollary}
  If $R$ is $\NPEq$-complete, then $R$ is $\NP$-complete.
\end{corollary}
\begin{proof}
  This follows immediately from the existence of an $\NP$-complete equivalence relation, as in \autoref{ex:npceqrel}, and the fact that a kernel reduction implies a many-one reduction.
\end{proof}

This corollary provides a clearer proof of \autocite[Proposition~8.1]{bcffm}.

\begin{corollary}[{\autocite[Proposition~8.1]{bcffm}}]
  If \textsc{GI} is \NPEq-complete then the polynomial hierarchy collapses to the second level, that is, $\PH = \STP$.
\end{corollary}
\begin{proof}
  By the previous corollary, if \textsc{GI} is $\NPEq$-complete, then it is $\NP$-complete, which implies the stated collapse (see \autocite{schoning87}).
\end{proof}

\autoref{thm:npceqrel} also provides a simple method for proving the equivalence of $\P = \NP$ and $\PEq = \NPEq$.

\begin{theorem}\label{thm:pnppeqnpeq}
  $\P = \NP$ if and only if $\PEq = \NPEq$.
\end{theorem}
\begin{proof}
  If $\P = \NP$, then $\PEq = \NPEq$ by their definitions.
  Suppose now that $\PEq = \NPEq$.
  Let $A$ denote the $\NP$-complete equivalence relation defined in \autoref{ex:npceqrel}.
  Since $A \in \NPEq$ and $\PEq = \NPEq$ by hypothesis, $A \in \PEq$, and hence $A \in \P$.
  Since $\P$ is closed under $\mor$ reductions, any $\NP$-complete problem in $\P$ implies $\P = \NP$.
\end{proof}

As stated in \autoref{sec:generalcompleteness}, we do not know whether an $\NPEq$-complete problem exists.
In the following theorem we describe an equivalence relation that, if it were $\NPEq$-complete, would prove that injective kernel reductions are strictly weaker than general kernel reductions.
This is interesting because it again demonstrates that the number and size of equivalence classes is important when considering the (im)possibility of polynomial-time kernel reductions between equivalence relations.
In the following theorem, if an equivalence relation is ``complete under $\kri$ reductions in $\NPEq$'' we mean that every equivalence relation in $\NPEq$ reduces to it by a polynomial-time computable kernel reduction which is also injective (that is, ``one-to-one'').

\begin{theorem}\label{thm:inj}
  Let $\Pi$ be a property on \textsc{GI}.
  %% If for each natural number $n$ there is a graph $G$ of size $n$ such that $\Pi(G) = 1$, and 
  If the equivalence relation $A$ defined by
  \begin{equation*}
    A = \left\{ \pair{G}{H} \, \middle| \, \pair{G}{H} \in \textsc{GI} \text{ or } (G \in L_\Pi \text{ and } H \in L_\Pi \text{ and } |G| = |H|) \right\}
  \end{equation*}
  is complete for $\NPEq$ under $\kr$ reductions, then $A$ is not complete under $\kri$ reductions.
\end{theorem}

The only difference between the equivalence relation $A$ defined here and the one defined in \autoref{thm:npceqrel} is the requirement that $|G| = |H|$.
This means that although the number of equivalence classes in $A$ is infinite (at least one for each size), each of those equivalence classes is itself finite.
In contrast, consider the equivalence relation $S$ defined by
\begin{equation*}\label{eq:ones}
  S = \left\{ \pair{x}{y} \, \middle| \, x \text{ and } y \text{ have the same number of } 1 \text{s} \right\}.
\end{equation*}
The equivalence relation $S$ has an infinite number of equivalence classes: $[1]$, $[11]$, $[111]$, etc.
Each equivalence class is itself infinite as well: for each $w \in \Sigma^*$, the equivalence class $[w]$ contains $w$, $0w$, $00w$, etc.

\begin{proof}[Proof of \autoref{thm:inj}]
  Let $S$ be the equivalence relation defined in the preceding paragraph.
  The language $S$ is decidable in linear time by a deterministic Turing machine, hence it is in $\NP$.
  Since $A$ is $\NPEq$-complete by hypothesis, $S \kr A$.
  Thus there is a polynomial-time computable function $f$ such that $\pair{x}{y} \in S$ if and only if $\pair{f(x)}{f(y)} \in A$.

  By the discussion preceding this theorem, $[w]_S$ is infinite and $[f(w)]_A$ is finite.
  Since a kernel reduction must preserve equivalence classes, $f([w]_S) \subseteq [f(w)]_A$.
  Consider $f|_{[w]_S}$, that is, $f$ restricted to the domain $[w]_S$.
  Then $f|_{[w]_S}$ is a mapping from the infinite set $[w]_S$ to the finite set $[f(w)]_A$.
  By the pigeonhole principle, $f|_{[w]_S}$ is not injective.
  Hence the unrestricted reduction $f$ is not injective, and therefore $A$ is not $\kri$-complete in $\NPEq$.
\end{proof}
