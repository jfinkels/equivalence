% creates a slightly larger arrow head; see
% http://tex.stackexchange.com/questions/49009/how-can-i-increase-arrow-head-size-in-tkz-graph
\pgfarrowsdeclare{biggertip}{biggertip}{%
  \setlength{\arrowsize}{3pt}
  \addtolength{\arrowsize}{.5\pgflinewidth}
  \pgfarrowsrightextend{0}
  \pgfarrowsleftextend{-5\arrowsize}
}{%
  \setlength{\arrowsize}{3pt}
  \addtolength{\arrowsize}{.5\pgflinewidth}
  \pgfpathmoveto{\pgfpoint{-5\arrowsize}{3\arrowsize}}
  \pgfpathlineto{\pgfpointorigin}
  \pgfpathlineto{\pgfpoint{-5\arrowsize}{-3\arrowsize}}
  \pgfusepathqstroke
}

{\LARGE
\begin{figure}
  \caption{\textnormal{Polynomial time many-one reduction $f$ from $\textsf{GI}$ to $\textsf{DirGI}$.}}
  \begin{displaymath}
    \xymatrix{
      \begin{tikzpicture}[rotate=-45, scale=2]
        \GraphInit[vstyle=Normal]
        \SetGraphUnit{3}
        \SetVertexNoLabel
        \Vertices{square}{a,b,c,d}
        \Edges(c,a,b,c,d)
      \end{tikzpicture}
      & \cong & 
      \begin{tikzpicture}[rotate=-45, scale=2]
        \GraphInit[vstyle=Normal]
        \SetGraphUnit{3}
        \SetVertexNoLabel
        \Vertices{square}{a,b,c,d}
        \Edges(d,b,a,d,c)
      \end{tikzpicture}
      \\
      & \ar[dd]_f & \\
      & & \\
      & & \\
      \begin{tikzpicture}[rotate=-45, scale=2]
        \GraphInit[vstyle=Normal]
        \SetGraphUnit{3}
        \SetVertexNoLabel
        \Vertices{square}{a,b,c,d}
        \SetUpEdge[style={->,>=biggertip, bend left=10}]
        \Edges(a, b, a)
        \Edges(a, c, a)
        \Edges(b, c, b)
        \Edges(c, d, c)
      \end{tikzpicture}
      & \cong &
      \begin{tikzpicture}[rotate=-45, scale=2]
        \GraphInit[vstyle=Normal]
        \SetGraphUnit{3}
        \SetVertexNoLabel
        \Vertices{square}{a,b,c,d}
        \SetUpEdge[style={->,>=biggertip, bend left=10}]
        \Edges(a, b, a)
        \Edges(a, d, a)
        \Edges(b, d, b)
        \Edges(c, d, c)
      \end{tikzpicture}
  }
  \end{displaymath}
\end{figure}

\begin{figure}
  \caption{\textnormal{Polynomial time kernel reduction $g$ from $\textsf{GI}$ to $\textsf{DirGI}$.}}
  \begin{displaymath}
    \xymatrix{
      \begin{tikzpicture}[rotate=-45, scale=2]
        \GraphInit[vstyle=Normal]
        \SetGraphUnit{3}
        \SetVertexNoLabel
        \Vertices{square}{a,b,c,d}
        \Edges(c,a,b,c,d)
      \end{tikzpicture}
      \ar[ddd]_g & \cong & 
      \begin{tikzpicture}[rotate=-45, scale=2]
        \GraphInit[vstyle=Normal]
        \SetGraphUnit{3}
        \SetVertexNoLabel
        \Vertices{square}{a,b,c,d}
        \Edges(d,b,a,d,c)
      \end{tikzpicture}
      \ar[ddd]_g \\
      & & \\
      & & \\
      \begin{tikzpicture}[rotate=-45, scale=2]
        \GraphInit[vstyle=Normal]
        \SetGraphUnit{3}
        \SetVertexNoLabel
        \Vertices{square}{a,b,c,d}
        \SetUpEdge[style={->,>=biggertip, bend left=10}]
        \Edges(a, b, a)
        \Edges(a, c, a)
        \Edges(b, c, b)
        \Edges(c, d, c)
      \end{tikzpicture}
      & \cong &
      \begin{tikzpicture}[rotate=-45, scale=2]
        \GraphInit[vstyle=Normal]
        \SetGraphUnit{3}
        \SetVertexNoLabel
        \Vertices{square}{a,b,c,d}
        \SetUpEdge[style={->,>=biggertip, bend left=10}]
        \Edges(a, b, a)
        \Edges(a, d, a)
        \Edges(b, d, b)
        \Edges(c, d, c)
      \end{tikzpicture}
  }
  \end{displaymath}
\end{figure}
}
