%%%%
%% equivalence.tex
%%
%% Copyright 2010, 2011 Jeffrey Finkelstein
%%
%% Except where otherwise noted, this work is made available under the terms of
%% the Creative Commons Attribution-ShareAlike 3.0 license,
%% http://creativecommons.org/licenses/by-sa/3.0/.
%%
%% You are free:
%%    * to Share — to copy, distribute and transmit the work
%%    * to Remix — to adapt the work
%% Under the following conditions:
%%    * Attribution — You must attribute the work in the manner specified by
%%    the author or licensor (but not in any way that suggests that they
%%    endorse you or your use of the work).
%%    * Share Alike — If you alter, transform, or build upon this work, you may
%%    distribute the resulting work only under the same, similar or a 
%%    compatible license.
%%    * For any reuse or distribution, you must make clear to others the 
%%    license terms of this work. The best way to do this is with a link to the
%%    web page http://creativecommons.org/licenses/by-sa/3.0/.
%%    * Any of the above conditions can be waived if you get permission from
%%    the copyright holder.
%%    * Nothing in this license impairs or restricts the author's moral rights.
%%%%
\documentclass[draft]{article}

% package imports
\usepackage{aliascnt} % for correct autoref labeling of non-theorems
\usepackage[noend, noline, linesnumbered, boxed]{algorithm2e}
\usepackage{amsthm} % for theorems, definitions, lemmas, and styles
\usepackage{amsmath} % for \implies
\usepackage{amssymb} % for \nleq
\usepackage{complexity} % for typesetting complexity classes
%\usepackage{fullpage} % for making the text take up the full page
\usepackage[
  pdftitle={On the computational complexity of equivalence relations under kernel reductions},
  pdfauthor={Jeffrey Finkelstein}
]{hyperref} % for pdf links when rendering \ref and \cite commands
\usepackage[all]{xy} % for inclusion diagram

\dontprintsemicolon

% Define theorem, lemma, and definition environments and corresponding styles.
% Lemmata, corollaries, and definitions are numbered with the same counter as
% that for theorems. We have to do some black magic to get the \autoref labels
% to work correctly.
\newtheorem{theorem}{Theorem}[section]

\newaliascnt{lemma}{theorem}
\newtheorem{lemma}[lemma]{Lemma}
\aliascntresetthe{lemma}

\newaliascnt{proposition}{theorem}
\newtheorem{proposition}[proposition]{Proposition}
\aliascntresetthe{proposition}

\newaliascnt{corollary}{theorem}
\newtheorem{corollary}[corollary]{Corollary}
\aliascntresetthe{corollary}

\newaliascnt{openproblem}{theorem}
\theoremstyle{definition} \newtheorem{openproblem}[openproblem]{Open problem}
\aliascntresetthe{openproblem}

\newaliascnt{definition}{theorem}
\theoremstyle{definition} \newtheorem{definition}[definition]{Definition}
\aliascntresetthe{definition}

% for repeating theorems
\makeatletter
\newtheorem*{rep@theorem}{\rep@title}
\newcommand{\newreptheorem}[2]{%
\newenvironment{rep#1}[1]{%
 \def\rep@title{#2 \ref{##1}}%
 \begin{rep@theorem}}%
 {\end{rep@theorem}}}
\makeatother
\newreptheorem{theorem}{Theorem}

% define lemma, corollary, and definition context labels for \autoref command
% (theorem is already defined)
\newcommand{\lemmaname}{Lemma}
\newcommand{\corollaryname}{Corollary}
\newcommand{\definitionname}{Definition}
\newcommand{\propositionname}{Proposition}
\newcommand{\algocflinename}{Algorithm}

% create a proof sketch environment
\newenvironment{sketch}{\begin{proof}[Proof sketch]}{\end{proof}}

% custom shortcut commands
\newcommand{\plain}[1]{\,\text{#1}\,} % plain text inside math environments
\newcommand{\sigmastar}{\{0, 1\}^{*}} % the set of all binary strings
\newcommand{\kj}{\overset{ker}{\oplus}} % kernel join
\newcommand{\kr}{\leq^{P}_{ker}} % kernel-reduces
\newcommand{\nkr}{\nleq^{P}_{ker}} % does not kernel-reduce
\newcommand{\krnt}{\leq_{ker}} % kernel-reduces without time bound
\newcommand{\nkrnt}{\nleq_{ker}} % does not kernel-reduce without time bound
\newcommand{\kequiv}{\equiv^{P}_{ker}} % is equivalent under kernel reductions
\newcommand{\kequivnt}{\equiv_{ker}} % is equivalent without time bound
\newcommand{\kri}{\leq^{P}_{ker,1\text{--}1}} % 1-1 kernel-reduces
\newcommand{\mor}{\leq^{P}_{m}} % many-one reduces
\newcommand{\moequiv}{\equiv^{P}_m} % is equivalent under many-one reductions
\newcommand{\symdiff}{\bigtriangleup} % set symmetric difference
%\newcommand{\tequiv}{\equiv^{p}_T} % is equivalent under Turing reductions
\newcommand{\defn}[1]{\emph{#1}} % emphasize words which are being defined
\newcommand{\pair}[2]{\langle#1,#2\rangle} % pairing function
\newcommand{\triple}[3]{\langle#1,#2,#3\rangle} % generalization of pairing fn.

\newcommand{\printintermediarytheorem}{If $\PEq\neq\NPEq$ and $\NPEqC$ is non-empty, then there exists an equivalence relation in $\NPEq$ which is neither in $\PEq$ nor $\NPEqC$.}

% create the creative commons license
\newcommand{\license}{Copyright 2010, 2011 Jef{}frey Finkelstein.
  This work is licensed under the Creative Commons Attribution-ShareAlike License.
  To view a copy of this license, visit \mbox{\url{http://creativecommons.org/licenses/by-sa/3.0/}}}

% redefine footnote so it has no reference and no number
\long\def\symbolfootnote#1{\begingroup%
\def\thefootnote{\fnsymbol{footnote}}\footnotetext{#1}\endgroup} 

% define the author, title, and date
\author{Jef{}frey Finkelstein} 
\title{On the computational complexity of equivalence relations under kernel
  reductions}
\date{\today} 

\begin{document}

\maketitle
\symbolfootnote{\license}

\section{Introduction}

In this paper we examine the power of ``kernel reductions'' on languages induced by equivalence relations, specifically for which membership can be decided by either a deterministic or a non-deterministic Turing machine running in polynomial time.
Given two equivalence relations $R$ and $S$, each of which can be expressed as a set of pairs of strings, a kernel reduction from $R$ to $S$ is a function $f$ for which $\pair{x}{y}\in R\iff\pair{f(x)}{f(y)}\in S$.
This function maps each \emph{member} of the pair in $R$ to an element of a pair in the relation $S$, instead of mapping the \emph{entire} pair to another pair.
The full definition of ``kernel reductions'' is given by Fortnow and Grochow in \cite{fg11} (and is repeated below), though the idea has existed before then.
In fact, all polynomial time many-one reductions to and from the graph isomorphism problem are, to the best of our knowledge, in fact kernel reductions, though they have not before been called by this name.
The same seems to go for polynomial time many-one reductions to and from other equivalence problems.
This kind of reduction is more natural than the usual many-one reduction for problems of equivalence.
It is therefore important to study the power of these reductions and how they can help further classify currently known and newly discovered complexity classes.

\section{Preliminaries}

The complexity classes $\P$, $\NP$, and $\FP$ (polynomial time computable functions) have the usual definitions.

If $f\colon S\to T$ is a well-defined function and $S'\subseteq S$, then \defn{$f$ restricted to the domain $S'$} is the function $f'\colon S'\to T$ defined by $f'(x)=f(x)$ for all $x\in S'$.
We denote this restricted function on a smaller domain by $f|_{S'}$.

If $\Sigma$ is an alphabet then $\Sigma^*$ is the set of all strings over the alphabet $\Sigma$.
If $x$ and $y$ are elements of $\Sigma^*$, then we denote by $\pair{x}{y}$ the \defn{pairwise encoding} of $x$ and $y$, which is itself an element of $\Sigma^*$.
As usual, a language over an alphabet $\Sigma$ is a subset of $\Sigma^*$.

Given a universe $U$, $R\subseteq U\times U$ is an \defn{equivalence relation on $U$} if $R$ is
\begin{enumerate}
\item reflexive: for all $x\in U$, $(x,x)\in R$
\item symmetric: for all $x,y\in U$, $(x,y)\in R$ implies $(y,x)\in R$
\item transitive: for all $x,y,z\in U$, $(x,y)\in R$ and $(y,z)\in R$ implies $(x,z)\in R$
\end{enumerate}
An equivalence relation $R$ can be encoded as a language by taking the pairwise encoding of each pair in $R$.
In this way we can study the computational complexity of classes of languages which represent equivalence relations.
In this paper we will abuse notation and write $\pair{x}{y}\in R$, but what we really mean is $(x,y)\in R$ and $\pair{x}{y}\in L_R$, the language induced by $R$.

The \defn{equivalence class} of $x$ with respect to an equivalence relation $R$ on $U$ is $\{y\in U|(x,y)\in R\}$. It is denoted $[x]_R$, or if the context is clear, simply $[x]$.
Each element $x\in U$ is in exactly one equivalence class, so the equivalence classes of an equivalence relation on $U$ provide a partition of $U$.

A \defn{complete invariant} for an equivalence relation $R$ on $U$ is a function $f\colon U\to T$ such that for all $x,y\in U$, $(x,y)\in R$ if and only if $f(x)=f(y)$.
In \autoref{sec:definitions} below we will define generalizations of the complete invariant which accept as input an additional witness to the equivalence of $x$ and $y$.

\defn{$\PEq$} is the class of equivalence relations for which membership can be decided by a Turing maching running in deterministic polynomial time.
\defn{$\NPEq$} is the class of equivalence relations for which membership can be decided by a Turing maching running in non-deterministic polynomial time.
In other words, $\PEq$ is the set of (languages induced by) equivalence relations which are in \P, and $\NPEq$ is the set of (languages induced by) equivalence relations which are in \NP.
As usual, $\PEq\subseteq\NPEq$.
%\defn{$\Ker$} is the class of equivalence relations which have a polynomial time computable complete invariant.

We now require a natural notion of reduction among equivalence relations.
If $R$ and $S$ are equivalence relations on $\Sigma^*$, we say $R$ \defn{kernel reduces to} $S$ if there exists a computable $f\colon\Sigma^*\to\Sigma^*$ such that $\forall x,y\in\Sigma^*$, $\pair{x}{y}\in R\iff \pair{f(x)}{f(y)}\in S$.
We denote this by $R\krnt S$.
If $f$ is computable in polynomial time, then we say $R$ \defn{polynomial time kernel reduces to} $S$ and use the notation $R\kr S$.

Notice the difference between a kernel reduction and a regular old many-one reduction: a kernel reduction maps $\pair{x}{y}\in R$ to $\pair{f(x)}{f(y)}\in S$, whereas a many-one reduction maps $\pair{x}{y}\in R$ to $f(\pair{x}{y})\in S$, for some polynomial time computable function $f$.
Informally, a function which computes a many-one reduction has access to both $x$ and $y$ but a function which computes a kernel reduction has access to only one of $x$ and $y$ at a time.
Note that since it is more restrictive, a kernel reduction induces a many-one reduction (namely the function $\pair{x}{y}\mapsto\pair{f(x)}{f(y)}$).

As an analog to polynomial time many-one completeness in \NP, we define a similar notion of polynomial time completeness under kernel reductions in \NPEq.
An equivalence relation $S$ is \defn{\NPEq-complete} if for all $R\in\NPEq$, $R\kr S$.
Similarly, an equivalence relation $S$ is \defn{\PEq-complete} if for all $R\in\PEq$, $R\kr S$.

\section{Definitions of \texorpdfstring{\NPEq}{NPEq}}\label{sec:definitions}

In this section we examine possible alternate definitions of \NPEq.
The main property of languages in $\NP$ is that membership in each language is verifiable in polynomial time, given a witness to the membership.
We propose here several possible definitions of $\NPEq$ in order to determine which make sense, which are too restrictive, and which are equivalent.

For the sake of brevity, in all definitions below, when we write $\exists w$, we mean $\exists w$ with length polynomially bounded with respect to the length of $x$, $y$, or the pair $\pair{x}{y}$ (depending on the requirements of the particular definition).

The first two definitions are analogs of the two fundamental definitions of \NP.
\begin{definition}\label{def:npeq1}
  An equivalence relation $R$ is in $\NPEqOne$ if there exists a non-deterministic Turing machine, call it $N$, which halts in time polynomial in the length of the input, such that
  \begin{displaymath}
    \pair{x}{y}\in R\iff N(\pair{x}{y})\plain{accepts}
  \end{displaymath}
\end{definition}
\begin{definition}\label{def:npeq2}
  An equivalence relation $R$ is in $\NPEqTwo$ if there exists a language $L\in\P$ such that
  \begin{displaymath}
    \pair{x}{y}\in R\iff \exists w\colon \pair{\pair{x}{y}}{w}\in L
  \end{displaymath}
\end{definition}

The next two definitions attempt to require that the witness language is itself an equivalence relation, instead of an arbitrary language in $\P$, as in \autoref{def:npeq2}.
\begin{definition}\label{def:npeq3}
  An equivalence relation $R$ is in $\NPEqThree$ if there exists an equivalence relation $R'\in \PEq$ such that
  \begin{displaymath}
    \pair{x}{y}\in R\iff \exists w_x,w_y\colon \pair{\pair{x}{w_x}}{\pair{y}{w_y}}\in R'
  \end{displaymath}
\end{definition}
\begin{definition}\label{def:npeq4}
  An equivalence relation $R$ is in $\NPEqFour$ if there exists an equivalence relation $R'\in \PEq$ such that
  \begin{displaymath}
    \pair{x}{y}\in R\iff \exists w\colon \pair{\pair{x}{w}}{\pair{y}{w}}\in R'
  \end{displaymath}
\end{definition}

The next two definitions attempt to allow the possibility of not just a simple string which witnesses the equivalence of $x$ and $y$, but a ``witness function'' which may map $x$ and $y$, along with witness strings, to an equivalence relation in \PEq.
\begin{definition}\label{def:npeq5}
  An equivalence relation $R$ is in $\NPEqFive$ if there exists an equivalence relation $R'\in \PEq$ and a function $f\in\FP$ such that
  \begin{displaymath}
    \pair{x}{y}\in R\iff \exists w_x,w_y\colon \pair{f(x, w_x)}{f(y, w_y)}\in R'
  \end{displaymath}
\end{definition}
\begin{definition}\label{def:npeq6}
  An equivalence relation $R$ is in $\NPEqSix$ if there exists an equivalence relation $R'\in \PEq$ and a function $f\in\FP$ such that
  \begin{displaymath}
    \pair{x}{y}\in R\iff \exists w\colon \pair{f(x, w)}{f(y, w)}\in R'
  \end{displaymath}
\end{definition}

The final two definitions attempt to describe equivalence relations for which there is a ``witnessed complete invariant'', which maps equivalent strings to equal strings when given access to some witness of their equivalence.
We say that an equivalence relation $R$ on a universe $U$ has a \defn{one-witness complete invariant} if there exists a function $f\colon U\times S\to T$ such that $(x,y)\in R$ if and only if $\exists w\in S\colon f(x, w)=f(y, w)$, and we say that it has a \defn{two-witness complete invariant} if $(x, y)\in R$ if and only if $\exists w_x, w_y\in S\colon f(x, w_x)=f(y, w_y)$.
\begin{definition}\label{def:npeq7}
  An equivalence relation $R$ is in $\NPEqSeven$ if it has a polynomial time computable two-witness complete invariant, that is, a function $f\in\FP$ such that
  \begin{displaymath}
    \pair{x}{y}\in R\iff \exists w_x, w_y\colon f(x, w_x) = f(y, w_y)
  \end{displaymath}
\end{definition}
\begin{definition}\label{def:npeq8}
  An equivalence relation $R$ is in $\NPEqEight$ if it has a polynomial time computable one-witness complete invariant, that is, a function $f\in\FP$ such that
  \begin{displaymath}
    \pair{x}{y}\in R\iff \exists w\colon f(x, w) = f(y, w)
  \end{displaymath}
\end{definition}

\autoref{fig:inclusions} shows the inclusions among each of the classes of equivalence relations defined above.
\begin{figure}
  \caption{\label{fig:inclusions}Inclusions among possible definitions of equivalence relations verifiable in deterministic polynomial time.}
  \begin{displaymath}
    \xymatrix{%
      \NPEqEight \ar[r] & \NPEqSix \ar[d] \ar[r] & \NPEqFour \ar[d] \\
      \NPEqSeven \ar[r] & \NPEqFive \ar[r] & \NPEqThree \ar[r] & \NPEqTwo \ar@{<->}[r] & \NPEqOne }
  \end{displaymath}
\end{figure}
The main ideas of these inclusions are presented in the following theorem (the complete proofs are tedious and so are omitted here).
\begin{theorem}\mbox{}
  \begin{enumerate}
  \item $\NPEqOne=\NPEqTwo$
  \item $\NPEqEight\subseteq\NPEqSix$ and $\NPEqSeven\subseteq\NPEqFive$
  \item $\NPEqSix\subseteq\NPEqFour$ and $\NPEqFive\subseteq\NPEqThree$
  \item $\NPEqSix\subseteq\NPEqFive$ and $\NPEqFour\subseteq\NPEqThree$
  \item $\NPEqThree\subseteq\NPEqTwo$
  \end{enumerate}
\end{theorem}
\begin{sketch}\mbox{}
  \begin{enumerate}
  \item Follows immediately from the standard definitions of \NP.
  \item Choose the relation $R'$ in the definitions of $\NPEqSix$ and $\NPEqFive$ to be the equality relation.
  \item Hardcode the function $f$ from the definitions of $\NPEqSix$ and $\NPEqFive$ into the relation $R'$ in the definitions of $\NPEqFour$ and $\NPEqThree$.
  \item Choose $w_x$ and $w_y$ in $\NPEqFive$ and $\NPEqThree$ to be equal to the $w$ from $\NPEqSix$ and $\NPEqFour$.
  \item Define $L$ to be the language $L=\{\pair{\pair{x}{y}}{w}|\pair{\pair{x}{w}}{\pair{y}{w}}\in R'\}$.\qedhere
  \end{enumerate}
\end{sketch}

We would like to be able to show that $\NPEq_2$ (or $\NPEq_1$, though it seems more difficult) is contained in any of the other classes which have an equivalence relation as the witness language in \P, but this would require simulating an arbitrary language, which is not necessarily an equivalence relation, by some constructed equivalence relation.
We are not guaranteed anything about the structure of the arbitrary language, and it is therefore difficult to construct an equivalence relation which represents that language.

\begin{openproblem}
  Does one of the complexity classes defined here have a complete problem under $\kr$ reductions?
\end{openproblem}

\section{Basic facts about kernel reductions}

In this section we provide some basic facts about kernel reductions.

\begin{proposition}\label{prop:compose}
  If $R\kr S$ and $S\kr T$ then $R\kr T$. In other words, polynomial time kernel reductions compose.
\end{proposition}
\begin{proof}
  Let $f$ and $g$ be the polynomial time kernel reductions from $R$ to $S$ and from $S$ to $T$, respectively.
  Then $g\circ f$ computes a polynomial time kernel reduction from $R$ to $T$.
  It is polynomial time computable because polynomial time computable functions compose, and $\pair{x}{y}\in R$ if and only if $\pair{f(x)}{f(y)}\in S$ if and only if $\pair{g(f(x))}{g(f(y))}\in T$.
  Therefore $R\kr T$.
\end{proof}

\begin{proposition}\label{prop:numbers}
  Let $R$ and $S$ be equivalence relations on $\sigmastar$.
  Suppose $R$ has $n$ equivalence classes and $S$ has $m$ equivalence classes.
  If $n>m$ then $R\nkrnt S$ (that is, $R$ does not kernel reduce to $S$, regardless of any time bound on the function computing the reduction).
\end{proposition}
\begin{proof}
  Assume with the intention of producing a contradiction that $R\krnt S$.
  Then there exists a computable function $f$ such that $\forall x,y$, $\pair{x}{y}\in R\iff \pair{f(x)}{f(y)}\in S$.

  Since $R$ has $n$ non-empty equivalence classes which form a partition of $\sigmastar$, then $\exists r_1,\ldots,r_n\in\sigmastar$ such that $R=[r_1]_R\cup\cdots\cup[r_n]_R$.
  Since each element of $R$ is in exactly one equivalence class, $\forall i,j\leq n$, $i=j\iff\pair{r_i}{r_j}\in R\iff\pair{f(r_i)}{f(r_j)}\in S$.
  Therefore the image of each $r_i$ is in some equivalence class in $S$.
  Also, $\forall i,j\leq n$, $i\neq j\iff \pair{r_i}{r_j}\notin R\iff \pair{f(r_i)}{f(r_j)}\notin S$.
  Therefore, the image of each $r_i$ does not relate to the image of any other $r_j$, for $i\neq j$, and $i,j\leq n$.
  Therefore each of the equivalence classes $[f(r_1)]_S,\ldots,[f(r_n)]_S$ is disjoint, so $S$ has at least $n$ equivalence classes.
  But $n>m$.
  This is a contradiction with the hypothesis that $S$ has $m$ equivalence classes.

  Therefore $R\nkrnt S$.
\end{proof}

\begin{corollary}\label{cor:finite}
  \mbox{}
  \begin{enumerate}
    \renewcommand{\labelenumi}{\roman{enumi}.}
  \item If an equivalence relation has a finite number of equivalence classes, then it is not \PEq-complete.
  \item If an equivalence relation has a finite number of equivalence classes, then it is not \NPEq-complete.
  \end{enumerate}
\end{corollary}
\begin{proof}
  Suppose $R$ is an equivalence relation.
  Assume with the intention of producing a contradiction that $R$ is \PEq-complete.
  $R_{eq}=\{\pair{x}{y}|x=y\}$ has an infinite number of equivalence classes (specifically, one for each binary string), and $R$ has a finite number of equivalence classes, by assumption.
  By \autoref{prop:numbers}, $R\nkr R_{eq}$.
  But since $R_{eq}\in\PEq$, then $R_{eq}\kr R$.
  This is a contradiction.
  Therefore $R$ is not \PEq-complete.

  The proof is the same in \NPEq, because $R_{eq}\in\PEq\subseteq\NPEq$.
\end{proof}

The following proposition is presented without proof, because its proof is nearly the same as the proof of the corresponding statement for polynomial time many-one reductions in \NP.

\begin{proposition}\label{prop:closed_under_kr}
  $\NPEq$ is closed under polynomial time kernel reductions, that is if $S\in\NPEq$ and $R\kr S$ then $R\in\NPEq$.
\end{proposition}

The next lemma states that kernel reductions must preserve ``related-ness'' of pairs of elements by mapping equivalence classes in the domain to equivalence classes in the codomain.

\begin{lemma}\label{lem:image}
  Let $R$ and $S$ be equivalence relations on $U$, and let $w\in U$.
  Suppose $R\krnt S$, by some computable function $f$.
  Then $f([w]_R)\subseteq [f(w)]_S$.
  In other words, the image of an equivalence class of $R$ is a subset of an equivalence class of $S$.
\end{lemma}
\begin{proof}
  Since $w\in [w]_R$, $f(w)\in f([w]_R)$.
  Let $x\in f([w]_R)$.
  Then $(x, f(w))\in S$, so $x\in [f(w)]_S$.
  Therefore $f([w]_R)\subseteq [f(w)]_S$.
\end{proof}

\section{Polynomial time kernel reductions in \texorpdfstring{\NPEq}{NPEq}}

In this section we examine the power of polynomial time kernel reductions on languages in \NPEq.

In the following theorem, by ``$\kri$-complete'' we mean ``complete under injective polynomial time kernel reductions''.

\begin{theorem}
  There exists an \NP-complete language $A$ such that if $A$ is $\kr$-complete
  in $\NPEq$, then $A$ is not $\kri$-complete in $\NPEq$.
\end{theorem}
\begin{proof}
  Suppose $G_1=(V_1, E_1)$ and $G_2=(V_2, E_2)$ are two undirected graphs.
  Let $A=\{\pair{\pair{G_1}{k_1}}{\pair{G_2}{k_2}}| k_1=k_2$ and $(G_1\cong G_2$ or $(G_1$ has a clique of size $k_1$ and $G_2$ has a clique of size $k_2$ and $|V_1|=|V_2|))\}$.
  Then $A$ is $\mor$-complete in $\NP$ by a reduction from \lang{CLIQUE}.
  $A$ is obviously in $\NP$---the witness is the either the isomorphism or the cliques.
  
  To describe the reduction, we first need to define an auxiliary function, $C(G, k)$.
  Let $C$ be defined as follows on all undirected graphs $G=(V, E)$ and integers $k$.
  If $|V| < k$, then $C(G, k)$ outputs the complete graph on $k$ vertices, $K_k$.
  If $|V|\geq k$, then $C(G, k)$ outputs the complete graph on $k$ vertices with $|V|-k$ additional disconnected vertices (so that the total number of vertices in the output graph is $|V|$).

  Now we can define the reduction $f$ from \lang{CLIQUE} to $A$ by $f(\pair{G}{k})=\pair{\pair{G}{k}}{\pair{C(G, k)}{k}}$.
  To prove the correctness of $f$, first suppose $\pair{G}{k}\in$ \lang{CLIQUE}, so $G$ has a clique of size $k$, which implies $|V|\geq k$.
  Hence, $G$ has a clique of size $k$, $C(G, k)$ has a clique of size $k$, and $|V|$ equals the number of vertices in $C(G, k)$ (by construction).
  Therefore $\pair{\pair{G}{k}}{\pair{C(G, k)}{k}}\in A$.
  Suppose $\pair{G}{k}\notin$ \lang{CLIQUE}, so $G$ does not have a clique of size $k$.
  $C(G, k)$ has a clique of size $k$ by definition, so the only case left to check is whether $G$ is isomorphic to $C(G, k)$.
  If they were isomorphic, then $G$ would have a clique of size $k$, but this is a contradiction with the hypothesis, so $\pair{\pair{G}{k}}{\pair{C(G, k)}{k}}\notin A$.
  Therefore \lang{CLIQUE}$\mor A$, and hence $A$ is $\mor$-complete in \NP.

  Now, we will consider the equivalence classes in the equivalence relation $R=\{\pair{x}{y}|x$ and $y$ have the same number of $1$s$\}$.
  The proof that this is an equivalence relation is straightforward.
  There are an infinite number of equivalence classes in $R$; specifically, they are $[\lambda]$, $[1]$, $[11]$, $[111]$, etc.
  Each equivalence class is itself infinite as well: if $w\in\sigmastar$ then $[w]$ contains $w$, $0w$, $00w$, $000w$, etc.

  Next, we will consider the equivalence classes in $A$.
  There are an infinite number of equivalence classes in $A$, as well (because there are an infinite number of graphs, and an infinite number of natural numbers).
  However, in $A$ each equivalence class contains only a finite number of elements.
  If $G=(V,E)$, then $[\pair{G}{k}]$ includes all pairs $\pair{H}{k}$, where $G$ is isomorphic to $H$, and all pairs $\pair{J}{k}$, where $G$ and $J$ have the same number of vertices and both have a clique of size $k$.
  Since there are only a finite number of graphs on $|V|$ vertices, and any graph isomorphic to $G$ must have $|V|$ vertices, the number of graphs isomorphic to $G$ is finite.
  Furthermore, the number of graphs with both $|V|$ vertices and a clique of size $k$ is also finite.
  Therefore, the equivalence class $[\pair{G}{k}]$ is finite.

  Now suppose $A$ is $\kr$-complete in $\NPEq$, as specified in the hypothesis of this theorem.
  Since $R\in\PEq\subseteq\NPEq$, then $R\kr A$, so $\exists g\in\FP$ such that $w\in\R\iff g(w)\in A$.

  Let $w\in\sigmastar$.
  Then $g(w)=\pair{G}{k}$ for some graph $G$ and some $k\in\mathbb{N}$.
  By the above arguments, $[w]_R$ is infinite and $[g(w)]_A=[\pair{G}{k}]_A$ is finite.
  By \autoref{lem:image}, $g([w]_R)\subseteq [g(w)]_A$.
  Consider $g|_{[w]_R}$, that is, $g$ restricted to the domain $[w]_R$.
  Then $g|_{[w]_R}$ is a mapping from an infinite set (specifically $[w]_R$) to a finite set (specifically $[g(w)]_A=[\pair{G}{k}]_A$).
  By the pigeonhole principle, $g|_{[w]_R}$ is not injective.
  Hence the unrestricted reduction $g$ is not injective, and therefore $A$ is not $\kri$-complete in \NPEq.
\end{proof}

\section{Existence of intermediary problems}

We will denote by $\NPEqC$ the set of equivalence relations which are $\kr$-complete for \NPEq.

The main result of this section is as follows.
\begin{reptheorem}{thm:intermediary}
  \printintermediarytheorem
\end{reptheorem}
To show this, we will adapt a proof of Ladner's original theorem\cite{ladner} showing that if $\P\neq\NP$ then there exist problems in $\NP$ which are neither in $\P$ nor \NP-complete.
The proof we follow can be found in \cite{bdg95}, which is an adaptation of Sch\"{o}ning's proof of the ``uniform diagonalization theorem''\cite{schoning}, which is itself a generalization of Ladner's original method.

First we need to provide some technical definitions and machinery.

\begin{definition}
  A class of languages $\mathcal{C}$ is \defn{closed under finite variations} if and only if $A\in \mathcal{C}$ and $A\symdiff B$ (the symmetric difference of $A$ and $B$) is finite implies $B\in \mathcal{C}$ for all $B$.
\end{definition}

\begin{definition}
  Given an alphabet $\Sigma$ with at least two different symbols, say $0$ and $1$, the \defn{kernel join} of two equivalence relations $R$ and $S$ over $\Sigma^*$ is $R\kj S=\{\pair{x0}{y0}|\pair{x}{y}\in R\}\cup\{\pair{x1}{y1}|\pair{x}{y}\in S\}$.
\end{definition}

\begin{lemma}\label{lem:join}
  If $R$ and $S$ are equivalence relations on $\Sigma^*$, then $R\kj S$ is an equivalence relation.
\end{lemma}
\begin{proof}
  Let $x$, $y$ and $z$ be non-empty strings in $\Sigma^*$.
  
  Since $x$ is non-empty, $x=x'0$ or $x=x'1$.
  Since $R$ is reflexive, $\pair{x'}{x'}\in R$.
  Hence $\pair{x}{x}\in R\kj S$.
  Therefore $R\kj S$ is reflexive.
  
  Suppose $\pair{x}{y}\in R\kj S$.
  Then either $x=x'0$, $y=y'0$ and $\pair{x'}{y'}\in R$ or $x=x'1$, $y=y'1$ and $\pair{x'}{y'}\in S$.
  In either case, symmetry follows from the symmetry of $R$ or $S$.

  Suppose $\pair{x}{y}$ and $\pair{y}{z}$ are both in $R\kj S$.
  In the case that $x=x'0$, $y=y'0$ and $\pair{x'}{y'}\in R$, and that $z=z'0$ and $\pair{y'}{z'}\in R$, then by the transitivity of $R$, $\pair{x'}{z'}\in R$, so $\pair{x}{z}\in R\kj S$.
  The argument is similar in the case that $x=x'1$, $y=y'1$ and $z=z'1$.
  It is a contradiction for the other two cases to exist, since $y$ cannot be equal to both $y'0$ and $y'1$.

  Since $R\kj S$ is reflexive, symmetric and transitive, $R\kj S$ is an equivalence relation.
\end{proof}

\begin{proposition}\label{prop:symdiff}
  Symmetric difference of equivalence relations preserves symmetry.
\end{proposition}
\begin{proof}
  Let $R$ and $S$ be equivalence relations.
  Let $(x,y)\in(R\symdiff S)$.
  In the case that $(x,y)\in R$ and $(x,y)\notin S$, then $(y,x)\in R$.
  If $(y,x)$ were in $S$, then $(x,y)$ would also be in $S$, by symmetry, but this is a contradiction.
  Hence $(y,x)\in R$ and $(y,x)\notin S$.
  The argument for the other case is symmetric.
  Therefore $(x,y)\in(R\symdiff S)\implies (y,x)\in(R\symdiff S)$.
\end{proof}

\begin{definition}
  Let $r\colon\mathbb{N}\to\mathbb{N}$ be a computable function such that $r(m)>m$ for all $m$.
  Define the set $G[r]$ as
  \begin{displaymath}
    G[r]=\{x\in\Sigma^*|r^n(0)\leq|x|<r^{n+1}(0) \plain{for some even} n\}
  \end{displaymath}
  where $r^n(m)$ denotes the $n$-fold application of $r$ to $m$:
  \begin{displaymath}
    \overbrace{r\circ r\circ r\circ\cdots\circ r}^{n \plain{times}}(m)
  \end{displaymath}
  $G[r]$ is called the \defn{gap language} generated by $r$.
\end{definition}

\begin{lemma}\label{lem:gap_p}
  If $r$ is time constructible, then $G[r]\in\P$.
\end{lemma}
\begin{proof}
  Proof omitted.
\end{proof}

We will denote the Cartesian product $G[r]\times G[r]$ by the slightly more succinct ${G[r]}^2$, and $\overline{G[r]}\times\overline{G[r]}$ by $\overline{G[r]}^2$.
Elements of ${G[r]}^2$ are pairs of strings whose lengths are in the ``even gaps'' of $r$, while elements of $\overline{G[r]}^2$ are pairs of strings whose lengths are in the ``odd gaps'' of $r$.

\begin{lemma}
  ${G[r]}^2$ and $\overline{G[r]}^2$ are \defn{partial equivalence relations} (that is, they are symmetric and transitive).
\end{lemma}
\begin{proof}
  We will prove the theorem for ${G[r]}^2$; a symmetric argument proves the theorem for $\overline{G[r]}^2$.

  Let $x,y\in\Sigma^*$.
  Suppose $\pair{x}{y}\in {G[r]}^2$, so the lengths of $x$ and $y$ are both in an even gap of $r$.
  Then $\pair{y}{x}\in{G[r]}^2$, so ${G[r]}^2$ is symmetric.
  Now let $z\in\Sigma^*$ and suppose also that $\pair{y}{z}\in {G[r]}^2$.
  Then $y$ and $z$ are both in an even gap of $r$, so $x$, $y$ and $z$ are all in some even gap of $r$, and hence $\pair{x}{z}\in {G[r]}^2$.
  Therefore ${G[r]}^2$ is transitive.
\end{proof}

We are now prepared to prove the main technical theorem which will allow us to construct an equivalence relation which is ``between'' two complexity classes.

\begin{theorem}\label{thm:diag}
  Let $R_1$ and $R_2$ be decidable equivalence relations, and let $\mathcal{C}_1$ and $\mathcal{C}_2$ be classes of decidable equivalence relations such that:
  \begin{enumerate}
  \item $R_1\notin\mathcal{C}_1$
  \item $R_2\notin\mathcal{C}_2$
  \item $\mathcal{C}_1$ and $\mathcal{C}_2$ are computably enumerable
  \item $\mathcal{C}_1$ and $\mathcal{C}_2$ are closed under finite variations
  \end{enumerate}
  Then there exists a decidable equivalence relation $R$ such that:
  \begin{enumerate}
  \item $R\notin \mathcal{C}_1$
  \item $R\notin \mathcal{C}_2$
  \item $R\kr R_1\kj R_2$
  \end{enumerate}
\end{theorem}
\begin{proof} %% TODO use \pair command for pairs
  Let $P_1, P_2, \ldots$ and $Q_1, Q_2, \ldots$ be enumerations of Turing machines deciding the languages in $\mathcal{C}_1$ and $\mathcal{C}_2$ respectively.
  Define the functions
  \begin{eqnarray*}
    r_1(n)=\underset{i\leq n}{max}\{|z_{i,n}|\}+1 & \text{and} &
    r_2(n)=\underset{i\leq n}{max}\{|z'_{i,n}|\}+1
  \end{eqnarray*}
  where $z_{i,n}$ is the smallest word in $\Sigma^*$ such that there exists an $x\in\Sigma^*$, with $n<|x|\leq|z_{i,n}|$, such that $(z_{i,n}, x)\in(L(P_i)\symdiff R_1)$, and $z'_{i,n}$ is the smallest word in $\Sigma^*$ such that there exists an $x'\in\Sigma^*$, with $n<|x'|\leq|z'_{i,n}|$, such that $(z'_{i,n}, x')\in(L(Q_i)\symdiff R_2)$.
  Note that it also suffices to find an $x$ and $x'$ such that $(x, z_{i,n})\in(L(P_i)\symdiff R_1)$ and $(x', z'_{i,n})\in(L(Q_i)\symdiff R_2)$, since symmetric difference on equivalence relations preserves symmetry by \autoref{prop:symdiff}.
  The more important requirement is that $|x|\leq|z_{i,n}|$, since we will require below that both $x$ and $z_{i,n}$ are in the same gap of a specific function.

  We claim that $z_{i,n}$ and $z'_{i,n}$ always exist.
  Assume that no such $z_{i,n}$ exists, so there are no words such that there exists an $x\in\Sigma^*$, with $n<|x|\leq|z_{i,n}|$, such that $(z_{i,n}, x)\in(L(P_i)\symdiff R_1)$.
  Therefore, there are no pairs in $(L(P_i)\symdiff R_1)$ with both elements of length greater than $n$.
  Then there are a finite number of pairs in $L(P_i)\symdiff R_1$, so $R_1$ is a finite variation of $L(P_i)$.
  Since $\mathcal{C}_1$ is closed under finite variations, $R_1\in\mathcal{C}_1$.
  This is a contradiction with the hypothesis that $R_1\notin\mathcal{C}_1$.
  Therefore such a $z_{i,n}$ always exists.
  The argument that $z'_{i,n}$ always exists is similar.

  Since $L(P_i)$ and $L(Q_i)$ are decidable for all $i$, and since $R_1$ and $R_2$ are decidable, so are $L(P_i)\symdiff R_1$ and $L(Q_i)\symdiff R_2$.
  For each $n$, there is a procedure which always halts and which computes $z_{i,n}$.
  A similar procedure computes $z'_{i,n}$.
  The procedure which computes the maximum of a finite set of numbers and which adds one to that value always halts as well, so $r_1$ and $r_2$ are total computable functions.

  Let $r\ge max(r_1,r_2)$ be a non-decreasing time constructible function, which exists by \autoref{nondec}.
  Now for all $n$ and all $i\leq n$, each element of the pairs $(z_{i,n},x)$ have length between $n$ and $r_1(n)$, by construction.
  The same is true for $(z'_{i,n}, x')$ between $n$ and $r_2(n)$.
  Notice that $(z_{i,n}, x)$ and $(z'_{i,n}, x')$ are ``witnesses'' that $R_1\neq L(P_i)$ and $R_2\neq L(Q_i)$ respectively.
  Hence for all $n$, there are some witnesses between $n$ and $r(n)$ that for all $i\leq n$, $R_1\neq L(P_i)$ and $R_2\neq L(Q_i)$.

  Define $R=({G[r]}^2\cap R_1)\cup(\overline{G[r]}^2\cap R_2)$, so $R$ is equal to pairs of $R_1$ in the ``even gaps'' of r and $R$ is equal to pairs of $R_2$ in the ``odd gaps'' of $r$.
  It remains to show that $R$ is an equivalence relation which satisfies the properties stated in the theorem.

  First we show that $R$ is indeed an equivalence relation.
  Symmetry and transitivity follow from the symmetry and transitivity of $R_1$, $R_2$, ${G[r]}^2$ and ${G[r]}^2$.
  To show reflexivity, suppose $x\in G[r]$.
  Hence $\pair{x}{x}\in {G[r]}^2$.
  Since $R_1$ is an equivalence relation, $\pair{x}{x}\in R_1$.
  Therefore $\pair{x}{x}\in({G[r]}^2\cap R_1)$, so $\pair{x}{x}\in R$.
  The argument for the case that $x\in\overline{G[r]}$ is similar.
  Therefore $R$ is reflexive, symmetric and transitive.

  Next we show that $R\notin\mathcal{C}_1$.
  The argument which proves $R\notin\mathcal{C}_2$ is symmetric.
  Assume $R\in\mathcal{C}_1$ in order to produce a contradiction.
  Then there exists an $i$ such that $R=L(P_i)$.
  Let $m$ be an even integer such that $r^m(0)\geq i$.
  By construction, there exists a pair $\pair{x}{z}$ such that $r^m(0)\leq|x|\leq|z|<r^{m+1}(0)$ and $\pair{x}{z}\in(L(P_i)\symdiff R_1)$.
  Since $m$ is even, $\pair{x}{z}\in {G[r]}^2$.
  Since $R$ is equal to $R_1$ where it coincides with ${G[r]}^2$, then $\pair{x}{z}\in(L(P_i)\symdiff R)$.
  This is a contradiction with the hypothesis that $R=L(P_i)$.
  Therefore $R\notin\mathcal{C}_1$.

  Finally, we show that $R\kr R_1\kj R_2$.
  First, we note that $R_1\kj R_2$ is an equivalence relation by \autoref{lem:join}, so a kernel reduction here is syntactically possible.
  Since $G[r]\in\P$ by \autoref{lem:gap_p}, the function defined by
  \begin{displaymath}
    f(x)=
    \begin{cases}
      x0 & \text{if}\, x\in G[r]\\
      x1 & \text{if}\, x\notin G[r]\\
    \end{cases}
  \end{displaymath}
  is a polynomial time computable function.
  
  To show that $f$ computes the reduction from $R$ to $R_1\kj R_2$ correctly, suppose first that $\pair{x}{y}\in R$, so $\pair{x}{y}$ is in either $({G[r]}^2\cap R_1)$ or $\overline{G[r]}^2\cap R_2)$.
  In the former case, both $x$ and $y$ are in $G[r]$, so $f(x)=x0$ and $f(y)=y0$, and both $x$ and $y$ are in $R_1$, so $\pair{x0}{y0}=\pair{f(x)}{f(y)}\in R_1$.
  The argument for the latter case is symmetric.

  For the converse, suppose $\pair{f(x)}{f(y)}\in R_1\kj R_2$.
  Then $f(x)$ and $f(y)$ either both end with $0$ or both end with $1$.
  In the case that both end with $0$, then there exist some strings $w_x$ and $w_y$ such that $f(x)=w_x0$, $f(y)=w_y0$ and $\pair{w_x}{w_y}\in R_1$.
  By construction of $f$, $w_x$ must equal the input $x$ and $w_y$ must equal the input $y$, so $\pair{x}{y}\in R_1$.
  Also by construction, $f(x)=x0$ if and only if $x\in G[x]$ and $f(y)=y0$ if and only if $y\in G[x]$, so $\pair{x}{y}\in{G[r]}^2$.
  Hence $\pair{x}{y}\in({G[r]}^2\cap R_1\subseteq R$.
  The argument for the case that both $f(x)$ and $f(y)$ end with $1$ is symmetric, and shows that $\pair{x}{y}\in(\overline{G[r]}^2\cap R_2)\subseteq R$.
  Therefore $\pair{x}{y}\in R$ if and only if $\pair{f(x)}{f(y)}\in R_1\kj R_2$, so $f$ correctly computes the reduction from $R$ to $R_1\kj R_2$.

  Since we have shown that the equivalence relation $R$ satisfies the properties in the statement of the theorem, this concludes the proof.
\end{proof}

We would now like to show that the result of this theorem holds when $\mathcal{C}_1=\PEq$ and $\mathcal{C}_2=\NPEqC$.

\begin{proposition}
  \mbox{} % to put the next item on a new line
  \begin{enumerate}
  \item $\PEq$ is computably enumerable.
  \item $\NPEqC$ is computably enumerable.
  \end{enumerate}
\end{proposition}
\begin{proof}
  These statements are true since any subset of a computably enumerable set is computably enumerable, and these are both subsets of $\NP$.
  Note that $\NPEqC$ may be empty.
\end{proof}

\begin{proposition}\label{prop:npeqc}
  If $\PEq\neq\NPEq$ and $\NPEqC$ is non-empty, then $\PEq\cap\NPEqC=\emptyset$.
\end{proposition}
\begin{proof}
  Assume $\PEq\cap\NPEqC\neq\emptyset$.
  Let $R$ be the \NPEq-complete equivalence relation which is also in \PEq.
  Then all problems can be kernel reduced to $R$ in polynomial time, and $R$ can be decided in polynomial time.
  Therefore, $\PEq=\NPEq$.
  This is a contradiction with the hypothesis.
  Therefore $\PEq\cap\NPEqC=\emptyset$.
\end{proof}

\begin{theorem}\label{thm:intermediary}
  \printintermediarytheorem
\end{theorem}
\begin{proof}
  The hypothesis of this theorem is the same as in \autoref{prop:npeqc}, so $\PEq\cap\NPEqC=\emptyset$.
  Let $S$ be an \NPEq-complete problem.
  Choose $R_1=S$, $R_2=\emptyset$, $\mathcal{C}_1=\PEq$ and $\mathcal{C}_2=\NPEqC$.
  Then by \autoref{thm:diag}, there exists an equivalence relation $R$ which is in neither $\NPEqC$ nor $\PEq$, but which kernel reduces to $S\kj\emptyset$.
  Since $S\kj\emptyset$ trivially kernel reduces to $S$, and since polynomial time kernel reductions compose by \autoref{prop:compose}, $R\kr S$.
  Since $\NPEq$ is closed under polynomial time kernel reductions by \autoref{prop:closed_under_kr}, $R\in\NPEq$.
\end{proof}

\section{Open problems}
The existence of a complete problem for $\NPEq$ under $\kr$ reductions continues to elude us.

\bibliographystyle{amsalpha} \bibliography{references}

\end{document}
