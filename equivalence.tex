%%%%
%% equivalence.tex
%%
%% Copyright 2010 Jeffrey Finkelstein
%%
%% Except where otherwise noted, this work is made available under the terms of
%% the Creative Commons Attribution-ShareAlike 3.0 license,
%% http://creativecommons.org/licenses/by-sa/3.0/.
%%
%% You are free:
%%    * to Share — to copy, distribute and transmit the work
%%    * to Remix — to adapt the work
%% Under the following conditions:
%%    * Attribution — You must attribute the work in the manner specified by
%%    the author or licensor (but not in any way that suggests that they
%%    endorse you or your use of the work).
%%    * Share Alike — If you alter, transform, or build upon this work, you may
%%    distribute the resulting work only under the same, similar or a 
%%    compatible license.
%%    * For any reuse or distribution, you must make clear to others the 
%%    license terms of this work. The best way to do this is with a link to the
%%    web page http://creativecommons.org/licenses/by-sa/3.0/.
%%    * Any of the above conditions can be waived if you get permission from
%%    the copyright holder.
%%    * Nothing in this license impairs or restricts the author's moral rights.
%%%%
\documentclass{article}

% package imports
\usepackage{aliascnt} % for correct autoref labeling of non-theorems
\usepackage[noend, noline, linesnumbered, boxed]{algorithm2e}
\usepackage{amsthm} % for theorems, definitions, lemmas, and styles
\usepackage{amsmath} % for \implies
\usepackage{amssymb} % for \nleq
\usepackage{complexity} % for typesetting complexity classes
\usepackage{cclicenses} % for adding creative commons license symbols
%\usepackage{fullpage} % for making the text take up the full page
\usepackage[
  backref=page, % backlinks in references
  pdftitle={On the computational complexity of equivalence relations under
    kernel reductions},
  pdfauthor={Jeffrey Finkelstein}
]{hyperref} % for pdf links when rendering \ref and \cite commands

% Define theorem, lemma, and definition environments and corresponding styles.
% Lemmata, corollaries, and definitions are numbered with the same counter as
% that for theorems. We have to do some black magic to get the \autoref labels
% to work correctly.
\newtheorem{theorem}{Theorem}[section]

\newaliascnt{lemma}{theorem}
\newtheorem{lemma}[lemma]{Lemma}
\aliascntresetthe{lemma}

\newaliascnt{corollary}{theorem}
\newtheorem{corollary}[corollary]{Corollary}
\aliascntresetthe{corollary}

\newaliascnt{definition}{theorem}
\theoremstyle{definition} \newtheorem{definition}[definition]{Definition}
\aliascntresetthe{definition}

% define lemma, corollary, and definition context labels for \autoref command
% (theorem is already defined)
\newcommand{\lemmaname}{Lemma}
\newcommand{\corollaryname}{Corollary}
\newcommand{\definitionname}{Definition}

% custom shortcut commands
%\newcommand{\plain}[1]{\,\text{#1}\,} % plain text inside math environments
\newcommand{\sigmastar}{\{0, 1\}^{*}} % the set of all binary strings
\newcommand{\kr}{\leq^{p}_{ker}} % kernel-reduces
\newcommand{\nkr}{\nleq^{p}_{ker}} % does not kernel-reduce
\newcommand{\krnt}{\leq_{ker}} % kernel-reduces without time bound
\newcommand{\nkrnt}{\nleq_{ker}} % does not kernel-reduce without time bound
\newcommand{\kequiv}{\equiv^{p}_{ker}} % is equivalent under kernel reductions
\newcommand{\kequivnt}{\equiv_{ker}} % is equivalent without time bound
\newcommand{\kri}{\leq^{p}_{ker,1\text{--}1}} % 1-1 kernel-reduces
\newcommand{\mor}{\leq^{p}_{m}} % many-one reduces
\newcommand{\moequiv}{\equiv^{p}_m} % is equivalent under many-one reductions
%\newcommand{\tequiv}{\equiv^{p}_T} % is equivalent under Turing reductions
\newcommand{\defn}[1]{\emph{#1}} % emphasize words which are being defined
\newcommand{\pair}[2]{\langle#1,#2\rangle} % pairing function
\newcommand{\triple}[3]{\langle#1,#2,#3\rangle} % generalization of pairing fn.

% create the creative commons license
\newcommand{\license}{\begin{tabular}[h]{r l}
    \bysa & \parbox{275pt}{Copyright 2010 Jeffrey Finkelstein. 
      Except where otherwise noted, this work is licensed
      under http://creativecommons.org/licenses/by-sa/3.0/}
\end{tabular}}

% redefine footnote so it has no reference and no number
\long\def\symbolfootnote#1{\begingroup%
\def\thefootnote{\fnsymbol{footnote}}\footnotetext{#1}\endgroup} 

% define the author, title, and date
\author{Jeffrey Finkelstein} 
\title{On the computational complexity of equivalence relations under kernel
  reductions}
\date{\today} 

\begin{document}

\maketitle
\symbolfootnote{\license}

\begin{theorem}\label{thm:numbers}
  Let $R$ and $S$ be equivalence relations on $\sigmastar$. Suppose $R$ has $n$
  equivalence classes and $S$ has $m$ equivalence classes. If $n>m$ then
  $R\nkrnt S$ (that is, $R$ does not kernel reduce to $S$, regardless of any
  time bound on the function computing the reduction).
\end{theorem}
\begin{proof}
  Assume with the intention of producing a contradiction that $R\krnt S$. Then
  there exists a computable function $f$ such that $\forall x,y$,
  $\pair{x}{y}\in R\iff \pair{f(x)}{f(y)}\in S$.

  Since $R$ has $n$ non-empty equivalence classes which form a partition of
  $\sigmastar$, then $\exists r_1,\ldots,r_n\in\sigmastar$ such that
  $R=[r_1]_R\cup\cdots\cup[r_n]_R$.  Since each element of $R$ is in exactly
  one equivalence class, $\forall i,j\leq n$, $i=j\iff\pair{r_i}{r_j}\in
  R\iff\pair{f(r_i)}{f(r_j)}\in S$. Therefore the image of each $r_i$ is in
  some equivalence class in $S$.  Also, $\forall i,j\leq n$, $i\neq j\iff
  \pair{r_i}{r_j}\notin R\iff \pair{f(r_i)}{f(r_j)}\notin S$. Therefore, the
  image of each $r_i$ does not relate to the image of any other $r_j$, for
  $i\neq j$, and $i,j\leq n$. Therefore each of the equivalence classes
  $[f(r_1)]_S,\ldots,[f(r_n)]_S$ is disjoint, so $S$ has at least $n$
  equivalence classes. But $n>m$. This is a contradiction with the hypothesis
  that $S$ has $m$ equivalence classes.

  Therefore $R\nkrnt S$.
\end{proof}

\begin{corollary}\label{cor:finite}
  \mbox{}
  \begin{enumerate}
    \renewcommand{\labelenumi}{\roman{enumi}.}
  \item If an equivalence relation has a finite number of equivalence classes,
    then it is not \PEq-complete.
  \item If an equivalence relation has a finite number of equivalence classes,
    then it is not \NPEq-complete.
  \end{enumerate}
\end{corollary}
\begin{proof}
  Suppose $R$ is an equivalence relation. Assume with the intention of
  producing a contradiction that $R$ is
  \PEq-complete. $R_{eq}=\{\pair{x}{y}|x=y\}$ has an infinite number of
  equivalence classes (specifically, one for each binary string), and $R$ has a
  finite number of equivalence classes, by assumption. By
  \autoref{thm:numbers}, $R\nkr R_{eq}$. But since $R_{eq}\in\PEq$, then
  $R_{eq}\kr R$. This is a contradiction. Therefore $R$ is not \PEq-complete.

  The proof is the same in \NPEq, because $R_{eq}\in\PEq\subseteq\NPEq$.
\end{proof}

\begin{theorem}\label{thm:repr_kr}
  Let $R,S$ be equivalence relations on $\sigmastar$. Suppose
  $R\in\PEq$. Suppose also that $R$ has $n$ equivalence classes and the number
  of equivalence classes of $S$ is greater than or equal to $n$ (possibly
  countably infinite). Let $REP(R)$ be a set of representatives of equivalence
  classes in $R$, and $REP(S)$ be a set of representatives of equivalence
  classes in $S$. If there exists a function $f\colon REP(R)\to REP(S)$ such
  that $f\in\FP$ and $f$ is injective, then $R\kr S$.
\end{theorem}
\begin{proof}
  Since $R\in\PEq$, $\exists M_R$, a deterministic Turing machine running in
  polynomial time, such that $\pair{x}{y}\in R\iff M_R(\pair{x}{y})$
  accepts. Construct $g\in\FP$ on input $w\in\sigmastar$:\\
  \begin{algorithm}[H]
    \For{$r_i\in REP(R)$}{
      \If{$M_R(\pair{w}{r_i})$ accepts}{
        \Return{$f(r_i)$}
      }
    }
  \end{algorithm}
  Notice that each $w\in\sigmastar$ is in exactly one equivalence class in $R$,
  because the equivalence classes partition $\sigmastar$, so
  $M_R(\pair{w}{r_i})$ accepts exactly once during the loop, when
  $\pair{w}{r_i}\in R$.

  Suppose $\pair{x}{y}\in R$, so $\pair{x}{r_i}\in R$ and $\pair{y}{r_i}\in R$
  for some $r_i\in REP(R)$. Then $g(x)$ outputs $f(r_i)$ and $g(y)$ outputs
  $f(r_i)$. Since $\pair{f(r_i)}{f(r_i)}\in S$, then $\pair{g(x)}{g(y)}\in S$.

  Suppose $\pair{x}{y}\notin R$. So $\exists r_i, r_j\in REP(R)$, with $r_i\neq
  r_j$, such that $\pair{x}{r_i}\in R$ and $\pair{y}{r_j}\in R$. Then $g(x)$
  outputs $f(r_i)$ and $g(y)$ outputs $f(r_j)$. Since $f$ is injective,
  $r_i\neq r_j\implies f(r_i)\neq f(r_j)$. Since $f(r_i)$ and $f(r_j)$ are
  distinct elements in $REP(S)$, they are representatives of two distinct
  equivalence classes. Since every element of $S$ is in exactly one equivalence
  class, and since $[f(r_i)]\neq[f(r_j)]$, it follows that
  $\pair{f(r_i)}{f(r_j)}=\pair{g(x)}{g(y)}\notin S$.

  Therefore $\pair{x}{y}\in R\iff \pair{g(x)}{g(y)}\in S$, so $R\kr S$.
\end{proof}

%% TODO define computably enumerable
\begin{theorem}\label{thm:reps}
  Let $R,S$ be equivalence relations on $\sigmastar$. Suppose
  $R\in\PEq$. Suppose also that $R$ has $n$ equivalence classes and the number
  of equivalence classes of $S$ is greater than or equal to $n$ (possibly
  countably infinite). Let $REP(R)$ be a set of representatives of equivalence
  classes in $R$, and $REP(S)$ be a set of representatives of equivalence
  classes in $S$. If $REP(R)$ and $REP(S)$ are computably enumerable then $R\kr
  S$.
\end{theorem}
\begin{proof}
  We compute the kernel reduction using the following procedure.

  Enumerate $REP(R)$ to produce $\{r_1, r_2, \ldots, r_n\}$, where each $r_i$
  is a representative of a distinct equivalence class of $R$. Enumerate
  $REP(S)$ to produce $\{s_1, s_2, \ldots, s_n\}$, where each $s_i$ is a
  representative of a distinct equivalence class of $S$. Note that we only need
  to output representatives of $n$ equivalence classes of $S$ though it may
  have more.

  Define $f\in\FP$ by $f(r_i)=s_i$, $\forall i\leq n$. We now wish to show that
  $f$ is injective. Suppose $r_i,r_j\in R$ and $r_i\neq r_j$. Then $f(r_i)=s_i$
  and $f(r_j)=s_j$. $r_i\neq r_j\implies i\neq j\implies s_i\neq s_j\implies
  f(s_i)\neq f(s_j)$. Therefore $f$ is injective.

  The function $f$ now satisfies the conditions in the hypothesis of
  \autoref{thm:repr_kr}, so the result follows.
\end{proof}

Note that computing the representatives of $R$ and $S$ takes some amount of
time \emph{independent} of the size of the input to the transducer $f$.

\begin{lemma}\label{lem:image}
  Let $R,S$ be equivalence relations on $\sigmastar$, and let
  $w\in\sigmastar$. Suppose $R\kr S$, by some $f\in\FP$. Then
  $f([w]_R)\subseteq [f(w)]_S$. In other words, the image of an equivalence
  class of $R$ is a subset of an equivalence class of $S$.
\end{lemma}
\begin{proof}
  Since $w\in [w]_R$, $f(w)\in f([w]_R)$. Let $x\in f([w]_R)$. Then $(x,
  f(w))\in S$, so $x\in [f(w)]_S$. Therefore $f([w]_R)\subseteq [f(w)]_S$.
\end{proof}

\begin{theorem}
  There exists an \NP-complete language $A$ such that if $A$ is $\kr$-complete
  in $\NPEq$, then $A$ is not $\kri$-complete in $\NPEq$.
\end{theorem}
\begin{proof}
  Let $A=\{\pair{\pair{G_1}{k_1}}{\pair{G_2}{k_2}}| k_1=k_2$ and $(G_1\cong
  G_2$ or $(G_1$ has a clique of size $k_1$ and $G_2$ has a clique of size
  $k_2$ and $|V_1|=|V_2|))\}$, where $G_1=(V_1, E_1)$ and $G_2=(V_2, E_2)$. The
  only difference between this equivalence relation and $R_{KC}$ is the
  condition that $G_1$ and $G_2$ must have the same number of vertices.

  $A$ is $\mor$-complete in $\NP$ by a reduction from $CLIQUE$ similar to the
  one in \autoref{thm:rkc_npc}.  If $G=(V,E)$, we can compute in polynomial
  time the reduction $f$ defined by
  $\pair{G}{k}\mapsto\pair{\pair{G}{k}}{\pair{K^+_k}{k}}$, where $K^+_k$ is the
  complete graph of $k$ vertices with $|V|-k$ vertices added (without any
  connecting edges), if $|V|\geq k$. If $|V|<k$, then $K^+_k=K_k$. To show $f$
  is a many-one reduction from $CLIQUE$, first suppose $\pair{G}{k}\in CLIQUE$,
  so $G$ has a clique of size $k$, which implies $|V|\geq k$. Hence, $G$ has a
  clique of size $k$, $K^+_k$ has a clique of size $k$, and $|V|$ equals the
  number of vertices in $K^+_k$ (by construction). Therefore
  $\pair{\pair{G}{k}}{\pair{K^+_k}{k}}\in A$. Suppose $\pair{G}{k}\notin
  CLIQUE$, so $G$ does not have a clique of size $k$. $K^+_k$ has a clique of
  size $k$ by definition, so the only case left to check is whether $G$ is
  isomorphic to $K^+_k$. If they were isomorphic, then $G$ would have a clique
  of size $k$, but this is a contradiction with the hypothesis, so
  $\pair{\pair{G}{k}}{\pair{K^+_k}{k}}\notin A$. Therefore $CLIQUE\mor A$, and
  hence $A$ is $\mor$-complete in $\NP$.

  Now, we will consider the equivalence classes in the equivalence relation
  $R_{bc}$, as defined in \autoref{def:rbc}. There are an infinite number of
  equivalence classes in $R_{bc}$; specifically, they are $[\lambda], [1],
  [11], [111],$ etc. Each equivalence class is itself infinite as well: if
  $w\in\sigmastar$ then $[w]$ contains $w, 0w, 00w, 000w,$ etc.

  % TODO is the number of n vertex graphs correct?

  Next, we will consider the equivalence classes in $A$. There are an infinite
  number of equivalence classes in $A$, as well (because there are an infinite
  number of graphs, and an infinite number of natural numbers). However, in $A$
  each equivalence class contains only a finite number of elements. If
  $G=(V,E)$, then $[\pair{G}{k}]$ includes all pairs $\pair{H}{k}$, where $G$
  is isomorphic to $H$, and all pairs $\pair{J}{k}$, where $G$ and $J$ have the
  same number of vertices and both have a clique of size $k$. Since there are
  only a finite number of graphs on $|V|$ vertices, and any graph isomorphic to
  $G$ must have $|V|$ vertices, the number of graphs isomorphic to $G$ is
  finite. Furthermore, the number of graphs with both $|V|$ vertices and a
  clique of size $k$ is also finite. Therefore, the equivalence class
  $[\pair{G}{k}]$ is finite.

  Now suppose $A$ is $\kr$-complete in $\NPEq$, as specified in the hypothesis
  of this theorem. Since $R_{bc}\in\PEq\subseteq\NPEq$, then $R_{bc}\kr A$, so
  $\exists f\in\FP$ such that $w\in\R_{bc}\iff f(w)\in A$.

  Let $w\in\sigmastar$. Then $f(w)=\pair{G}{k}$ for some graph $G$ and some
  $k\in\mathbb{N}$. By the above arguments, $[w]_{R_{bc}}$ is infinite and
  $[f(w)]_A=[\pair{G}{k}]_A$ is finite. By \autoref{lem:image},
  $f([w]_{R_{bc}})\subseteq [f(w)]_A$. Consider $f|_{[w]_{R_{bc}}}$, that is,
  $f$ restricted to the domain $[w]_{R_{bc}}$. Then $f|_{[w]_{R_{bc}}}$ is a
  mapping from an infinite set (specifically $[w]_{R_{bc}}$) to a finite set
  (specifically $[f(w)]_A=[\pair{G}{k}]_A$). By the pigeonhole principle,
  $f|_{[w]_{R_{bc}}}$ is not injective. Hence the unrestricted reduction $f$ is
  not injective, and therefore $A$ is not $\kri$-complete in \NPEq.
\end{proof}

\bibliographystyle{amsalpha} \bibliography{references}

\end{document}
