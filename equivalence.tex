%%%%
%% equivalence.tex
%%
%% Copyright 2010, 2011 Jeffrey Finkelstein
%%
%% Except where otherwise noted, this work is made available under the terms of
%% the Creative Commons Attribution-ShareAlike 3.0 license,
%% http://creativecommons.org/licenses/by-sa/3.0/.
%%
%% You are free:
%%    * to Share — to copy, distribute and transmit the work
%%    * to Remix — to adapt the work
%% Under the following conditions:
%%    * Attribution — You must attribute the work in the manner specified by
%%    the author or licensor (but not in any way that suggests that they
%%    endorse you or your use of the work).
%%    * Share Alike — If you alter, transform, or build upon this work, you may
%%    distribute the resulting work only under the same, similar or a 
%%    compatible license.
%%    * For any reuse or distribution, you must make clear to others the 
%%    license terms of this work. The best way to do this is with a link to the
%%    web page http://creativecommons.org/licenses/by-sa/3.0/.
%%    * Any of the above conditions can be waived if you get permission from
%%    the copyright holder.
%%    * Nothing in this license impairs or restricts the author's moral rights.
%%%%
\documentclass{article}

% package imports
\usepackage{aliascnt} % for correct autoref labeling of non-theorems
\usepackage[noend, noline, linesnumbered, boxed]{algorithm2e}
\usepackage{amsthm} % for theorems, definitions, lemmas, and styles
\usepackage{amsmath} % for \implies
\usepackage{amssymb} % for \nleq
\usepackage{complexity} % for typesetting complexity classes
%\usepackage{fullpage} % for making the text take up the full page
\usepackage[
  pdftitle={On the computational complexity of equivalence relations under kernel reductions},
  pdfauthor={Jeffrey Finkelstein}
]{hyperref} % for pdf links when rendering \ref and \cite commands

% Define theorem, lemma, and definition environments and corresponding styles.
% Lemmata, corollaries, and definitions are numbered with the same counter as
% that for theorems. We have to do some black magic to get the \autoref labels
% to work correctly.
\newtheorem{theorem}{Theorem}[section]

\newaliascnt{lemma}{theorem}
\newtheorem{lemma}[lemma]{Lemma}
\aliascntresetthe{lemma}

\newaliascnt{proposition}{theorem}
\newtheorem{proposition}[proposition]{Proposition}
\aliascntresetthe{proposition}

\newaliascnt{corollary}{theorem}
\newtheorem{corollary}[corollary]{Corollary}
\aliascntresetthe{corollary}

\newaliascnt{definition}{theorem}
\theoremstyle{definition} \newtheorem{definition}[definition]{Definition}
\aliascntresetthe{definition}

% for repeating theorems
\makeatletter
\newtheorem*{rep@theorem}{\rep@title}
\newcommand{\newreptheorem}[2]{%
\newenvironment{rep#1}[1]{%
 \def\rep@title{#2 \ref{##1}}%
 \begin{rep@theorem}}%
 {\end{rep@theorem}}}
\makeatother
\newreptheorem{theorem}{Theorem}

% define lemma, corollary, and definition context labels for \autoref command
% (theorem is already defined)
\newcommand{\lemmaname}{Lemma}
\newcommand{\corollaryname}{Corollary}
\newcommand{\definitionname}{Definition}
\newcommand{\propositionname}{Proposition}

% custom shortcut commands
\newcommand{\plain}[1]{\,\text{#1}\,} % plain text inside math environments
\newcommand{\sigmastar}{\{0, 1\}^{*}} % the set of all binary strings
\newcommand{\kj}{\overset{ker}{\oplus}} % kernel join
\newcommand{\kr}{\leq^{P}_{ker}} % kernel-reduces
\newcommand{\nkr}{\nleq^{P}_{ker}} % does not kernel-reduce
\newcommand{\krnt}{\leq_{ker}} % kernel-reduces without time bound
\newcommand{\nkrnt}{\nleq_{ker}} % does not kernel-reduce without time bound
\newcommand{\kequiv}{\equiv^{P}_{ker}} % is equivalent under kernel reductions
\newcommand{\kequivnt}{\equiv_{ker}} % is equivalent without time bound
\newcommand{\kri}{\leq^{P}_{ker,1\text{--}1}} % 1-1 kernel-reduces
\newcommand{\mor}{\leq^{P}_{m}} % many-one reduces
\newcommand{\moequiv}{\equiv^{P}_m} % is equivalent under many-one reductions
\newcommand{\symdiff}{\bigtriangleup} % set symmetric difference
%\newcommand{\tequiv}{\equiv^{p}_T} % is equivalent under Turing reductions
\newcommand{\defn}[1]{\emph{#1}} % emphasize words which are being defined
\newcommand{\pair}[2]{\langle#1,#2\rangle} % pairing function
\newcommand{\triple}[3]{\langle#1,#2,#3\rangle} % generalization of pairing fn.

\newcommand{\printintermediarytheorem}{If $\PEq\neq\NPEq$ and $\NPEqC$ is non-empty, then there exists an equivalence relation in $\NPEq$ which is neither in $\PEq$ nor $\NPEqC$.}

% create the creative commons license
\newcommand{\license}{Copyright 2010, 2011 Jef{}frey Finkelstein.
  This work is licensed under the Creative Commons Attribution-ShareAlike License.
  To view a copy of this license, visit \mbox{\url{http://creativecommons.org/licenses/by-sa/3.0/}}}

% redefine footnote so it has no reference and no number
\long\def\symbolfootnote#1{\begingroup%
\def\thefootnote{\fnsymbol{footnote}}\footnotetext{#1}\endgroup} 

% define the author, title, and date
\author{Jef{}frey Finkelstein} 
\title{On the computational complexity of equivalence relations under kernel
  reductions}
\date{\today} 

\begin{document}

\maketitle
\symbolfootnote{\license}

\section{Basic Results}
\begin{proposition}\label{prop:compose}
  If $R\kr S$ and $S\kr T$ then $R\kr T$. In other words, polynomial time kernel reductions compose.
\end{proposition}
\begin{proof}
  Let $f$ and $g$ be the polynomial time kernel reductions from $R$ to $S$ and from $S$ to $T$, respectively.
  Then $g\circ f$ computes a polynomial time kernel reduction from $R$ to $T$.
  It is polynomial time computable because polynomial time computable functions compose, and $\pair{x}{y}\in R$ if and only if $\pair{f(x)}{f(y)}\in S$ if and only if $\pair{g(f(x))}{g(f(y))}\in T$.
  Therefore $R\kr T$.
\end{proof}

\begin{theorem}\label{thm:numbers}
  Let $R$ and $S$ be equivalence relations on $\sigmastar$.
  Suppose $R$ has $n$ equivalence classes and $S$ has $m$ equivalence classes.
  If $n>m$ then $R\nkrnt S$ (that is, $R$ does not kernel reduce to $S$, regardless of any time bound on the function computing the reduction).
\end{theorem}
\begin{proof}
  Assume with the intention of producing a contradiction that $R\krnt S$.
  Then there exists a computable function $f$ such that $\forall x,y$, $\pair{x}{y}\in R\iff \pair{f(x)}{f(y)}\in S$.

  Since $R$ has $n$ non-empty equivalence classes which form a partition of $\sigmastar$, then $\exists r_1,\ldots,r_n\in\sigmastar$ such that $R=[r_1]_R\cup\cdots\cup[r_n]_R$.
  Since each element of $R$ is in exactly one equivalence class, $\forall i,j\leq n$, $i=j\iff\pair{r_i}{r_j}\in R\iff\pair{f(r_i)}{f(r_j)}\in S$.
  Therefore the image of each $r_i$ is in some equivalence class in $S$.
  Also, $\forall i,j\leq n$, $i\neq j\iff \pair{r_i}{r_j}\notin R\iff \pair{f(r_i)}{f(r_j)}\notin S$.
  Therefore, the image of each $r_i$ does not relate to the image of any other $r_j$, for $i\neq j$, and $i,j\leq n$.
  Therefore each of the equivalence classes $[f(r_1)]_S,\ldots,[f(r_n)]_S$ is disjoint, so $S$ has at least $n$ equivalence classes.
  But $n>m$.
  This is a contradiction with the hypothesis that $S$ has $m$ equivalence classes.

  Therefore $R\nkrnt S$.
\end{proof}

\begin{corollary}\label{cor:finite}
  \mbox{}
  \begin{enumerate}
    \renewcommand{\labelenumi}{\roman{enumi}.}
  \item If an equivalence relation has a finite number of equivalence classes, then it is not \PEq-complete.
  \item If an equivalence relation has a finite number of equivalence classes, then it is not \NPEq-complete.
  \end{enumerate}
\end{corollary}
\begin{proof}
  Suppose $R$ is an equivalence relation.
  Assume with the intention of producing a contradiction that $R$ is \PEq-complete.
  $R_{eq}=\{\pair{x}{y}|x=y\}$ has an infinite number of equivalence classes (specifically, one for each binary string), and $R$ has a finite number of equivalence classes, by assumption.
  By \autoref{thm:numbers}, $R\nkr R_{eq}$.
  But since $R_{eq}\in\PEq$, then $R_{eq}\kr R$.
  This is a contradiction.
  Therefore $R$ is not \PEq-complete.

  The proof is the same in \NPEq, because $R_{eq}\in\PEq\subseteq\NPEq$.
\end{proof}

%% TODO extend this theorem to countably infinite sets
\begin{theorem}\label{thm:repr_kr}
  Let $R,S$ be equivalence relations on $\sigmastar$.
  Suppose $R\in\PEq$.
  Suppose also that $R$ has $n$ equivalence classes and the number of equivalence classes of $S$ is greater than or equal to $n$ (possibly countably infinite).
  Let $REP(R)$ be a set of representatives of equivalence classes in $R$, and $REP(S)$ be a set of representatives of equivalence classes in $S$.
  If there exists a function $f\colon REP(R)\to REP(S)$ such that $f\in\FP$ and $f$ is injective, then $R\kr S$.
\end{theorem}
\begin{proof}
  Since $R\in\PEq$, $\exists M_R$, a deterministic Turing machine running in polynomial time, such that $\pair{x}{y}\in R\iff M_R(\pair{x}{y})$ accepts.
  Construct $g\in\FP$ on input $w\in\sigmastar$:\\
  \begin{algorithm}[H]
    \For{$r_i\in REP(R)$}{
      \If{$M_R(\pair{w}{r_i})$ accepts}{
        \Return{$f(r_i)$}
      }
    }
  \end{algorithm}
  Notice that each $w\in\sigmastar$ is in exactly one equivalence class in $R$, because the equivalence classes partition $\sigmastar$, so $M_R(\pair{w}{r_i})$ accepts exactly once during the loop, when $\pair{w}{r_i}\in R$.

  Suppose $\pair{x}{y}\in R$, so $\pair{x}{r_i}\in R$ and $\pair{y}{r_i}\in R$ for some $r_i\in REP(R)$.
  Then $g(x)$ outputs $f(r_i)$ and $g(y)$ outputs $f(r_i)$.
  Since $\pair{f(r_i)}{f(r_i)}\in S$, then $\pair{g(x)}{g(y)}\in S$.

  Suppose $\pair{x}{y}\notin R$.
  So $\exists r_i, r_j\in REP(R)$, with $r_i\neq r_j$, such that $\pair{x}{r_i}\in R$ and $\pair{y}{r_j}\in R$.
  Then $g(x)$ outputs $f(r_i)$ and $g(y)$ outputs $f(r_j)$.
  Since $f$ is injective, $r_i\neq r_j\implies f(r_i)\neq f(r_j)$.
  Since $f(r_i)$ and $f(r_j)$ are distinct elements in $REP(S)$, they are representatives of two distinct equivalence classes.
  Since every element of $S$ is in exactly one equivalence class, and since $[f(r_i)]\neq[f(r_j)]$, it follows that $\pair{f(r_i)}{f(r_j)}=\pair{g(x)}{g(y)}\notin S$.

  Therefore $\pair{x}{y}\in R\iff \pair{g(x)}{g(y)}\in S$, so $R\kr S$.
\end{proof}

%% TODO define computably enumerable
\begin{theorem}\label{thm:reps}
  Let $R,S$ be equivalence relations on $\sigmastar$.
  Suppose $R\in\PEq$.
  Suppose also that $R$ has $n$ equivalence classes and the number of equivalence classes of $S$ is greater than or equal to $n$ (possibly countably infinite).
  Let $REP(R)$ be a set of representatives of equivalence classes in $R$, and $REP(S)$ be a set of representatives of equivalence classes in $S$.
  If $REP(R)$ and $REP(S)$ are computably enumerable then $R\kr S$.
\end{theorem}
\begin{proof}
  We compute the kernel reduction using the following procedure.

  Enumerate $REP(R)$ to produce $\{r_1, r_2, \ldots, r_n\}$, where each $r_i$ is a representative of a distinct equivalence class of $R$.
  Enumerate $REP(S)$ to produce $\{s_1, s_2, \ldots, s_n\}$, where each $s_i$ is a representative of a distinct equivalence class of $S$.
  Note that we only need to output representatives of $n$ equivalence classes of $S$ though it may have more.

  Define $f\in\FP$ by $f(r_i)=s_i$, $\forall i\leq n$.
  We now wish to show that $f$ is injective.
  Suppose $r_i,r_j\in R$ and $r_i\neq r_j$.
  Then $f(r_i)=s_i$ and $f(r_j)=s_j$. $r_i\neq r_j\implies i\neq j\implies s_i\neq s_j\implies f(s_i)\neq f(s_j)$.
  Therefore $f$ is injective.

  The function $f$ now satisfies the conditions in the hypothesis of \autoref{thm:repr_kr}, so the result follows.
\end{proof}

Note that computing the representatives of $R$ and $S$ takes some amount of time \emph{independent} of the size of the input to the transducer $f$.

\begin{lemma}\label{lem:image}
  Let $R,S$ be equivalence relations on $\sigmastar$, and let $w\in\sigmastar$.
  Suppose $R\kr S$, by some $f\in\FP$.
  Then $f([w]_R)\subseteq [f(w)]_S$.
  In other words, the image of an equivalence class of $R$ is a subset of an equivalence class of $S$.
\end{lemma}
\begin{proof}
  Since $w\in [w]_R$, $f(w)\in f([w]_R)$.
  Let $x\in f([w]_R)$.
  Then $(x, f(w))\in S$, so $x\in [f(w)]_S$.
  Therefore $f([w]_R)\subseteq [f(w)]_S$.
\end{proof}

\begin{theorem}
  There exists an \NP-complete language $A$ such that if $A$ is $\kr$-complete
  in $\NPEq$, then $A$ is not $\kri$-complete in $\NPEq$.
\end{theorem}
\begin{proof}
  Suppose $G_1=(V_1, E_1)$ and $G_2=(V_2, E_2)$ are two undirected graphs.
  Let $A=\{\pair{\pair{G_1}{k_1}}{\pair{G_2}{k_2}}| k_1=k_2$ and $(G_1\cong G_2$ or $(G_1$ has a clique of size $k_1$ and $G_2$ has a clique of size $k_2$ and $|V_1|=|V_2|))\}$.
  Then $A$ is $\mor$-complete in $\NP$ by a reduction from \lang{CLIQUE}.
  
  To describe the reduction, we first need to define a function, $C(G, k)$.
  Let $C$ be defined as follows on all undirected graphs $G=(V, E)$ and integers $k$.
  If $|V| < k$, then $C(G, k)$ outputs the complete graph on $k$ vertices, $K_k$.
  If $|V|\geq k$, then $C(G, k)$ outputs the complete graph on $k$ vertices with $|V|-k$ additional disconnected vertices (so that the total number of vertices in the output graph is $|V|$).

  Now the reduction from \lang{CLIQUE} to $A$, call it $f$, is defined by $f(\pair{G}{k})=\pair{\pair{G}{k}}{\pair{C(G, k)}{k}}$.
  To show $f$ is a many-one reduction from \lang{CLIQUE}, first suppose $\pair{G}{k}\in$ \lang{CLIQUE}, so $G$ has a clique of size $k$, which implies $|V|\geq k$.
  Hence, $G$ has a clique of size $k$, $C(G, k)$ has a clique of size $k$, and $|V|$ equals the number of vertices in $C(G, k)$ (by construction).
  Therefore $\pair{\pair{G}{k}}{\pair{C(G, k)}{k}}\in A$.
  Suppose $\pair{G}{k}\notin$ \lang{CLIQUE}, so $G$ does not have a clique of size $k$.
  $C(G, k)$ has a clique of size $k$ by definition, so the only case left to check is whether $G$ is isomorphic to $C(G, k)$.
  If they were isomorphic, then $G$ would have a clique of size $k$, but this is a contradiction with the hypothesis, so $\pair{\pair{G}{k}}{\pair{C(G, k)}{k}}\notin A$.
  Therefore \lang{CLIQUE}$\mor A$, and hence $A$ is $\mor$-complete in \NP.

  Now, we will consider the equivalence classes in the equivalence relation $R=\{\pair{x}{y}|x$ and $y$ have the same number of $1$s$\}$.
  There are an infinite number of equivalence classes in $R$; specifically, they are $[\lambda]$, $[1]$, $[11]$, $[111]$, etc.
  Each equivalence class is itself infinite as well: if $w\in\sigmastar$ then $[w]$ contains $w$, $0w$, $00w$, $000w$, etc.

  Next, we will consider the equivalence classes in $A$.
  There are an infinite number of equivalence classes in $A$, as well (because there are an infinite number of graphs, and an infinite number of natural numbers).
  However, in $A$ each equivalence class contains only a finite number of elements.
  If $G=(V,E)$, then $[\pair{G}{k}]$ includes all pairs $\pair{H}{k}$, where $G$ is isomorphic to $H$, and all pairs $\pair{J}{k}$, where $G$ and $J$ have the same number of vertices and both have a clique of size $k$.
  Since there are only a finite number of graphs on $|V|$ vertices, and any graph isomorphic to $G$ must have $|V|$ vertices, the number of graphs isomorphic to $G$ is finite.
  Furthermore, the number of graphs with both $|V|$ vertices and a clique of size $k$ is also finite.
  Therefore, the equivalence class $[\pair{G}{k}]$ is finite.

  Now suppose $A$ is $\kr$-complete in $\NPEq$, as specified in the hypothesis of this theorem.
  Since $R\in\PEq\subseteq\NPEq$, then $R\kr A$, so $\exists g\in\FP$ such that $w\in\R\iff g(w)\in A$.

  Let $w\in\sigmastar$.
  Then $g(w)=\pair{G}{k}$ for some graph $G$ and some $k\in\mathbb{N}$.
  By the above arguments, $[w]_R$ is infinite and $[g(w)]_A=[\pair{G}{k}]_A$ is finite.
  By \autoref{lem:image}, $g([w]_R)\subseteq [g(w)]_A$.
  Consider $g|_{[w]_R}$, that is, $g$ restricted to the domain $[w]_R$.
  Then $g|_{[w]_R}$ is a mapping from an infinite set (specifically $[w]_R$) to a finite set (specifically $[g(w)]_A=[\pair{G}{k}]_A$).
  By the pigeonhole principle, $g|_{[w]_R}$ is not injective.
  Hence the unrestricted reduction $g$ is not injective, and therefore $A$ is not $\kri$-complete in \NPEq.
\end{proof}

\section{Existence of intermediary problems}

We will denote by $\NPEqC$ the set of equivalence relations which are $\kr$-complete for \NPEq.

The main result of this section is as follows.
\begin{reptheorem}{thm:intermediary}
  \printintermediarytheorem
\end{reptheorem}
To show this, we will follow a proof of Ladner's original theorem showing that that if $\P\neq\NP$ then there exist problems which are not in $\P$ and not \NP-complete.
The proof we follow can be found in \cite{bdg95}.

\begin{definition}
  A class of languages $\mathcal{C}$ is \defn{closed under finite variations} if and only if $A\in \mathcal{C}$ and $A\symdiff B$ is finite implies $B\in \mathcal{C}$ for all $B$.
\end{definition}

\begin{definition}
  Given an alphabet $\Sigma$ with at least two different symbols, say $0$ and $1$, the \defn{kernel join} of two equivalence relations $R$ and $S$ over $\Sigma^*$ is $R\kj S=\{\pair{x0}{y0}|\pair{x}{y}\in R\}\cup\{\pair{x1}{y1}|\pair{x}{y}\in S\}$.
\end{definition}

\begin{theorem}
  If $R$ and $S$ are equivalence relations on $\Sigma^*$, then $R\kj S$ is an equivalence relation.
\end{theorem}
\begin{proof}
  %% TODO Are non-empty strings okay?

  Let $x$, $y$ and $z$ be non-empty strings in $\Sigma^*$.
  
  Since $x$ is non-empty, $x=x'0$ or $x=x'1$.
  Since $R$ is reflexive, $\pair{x'}{x'}\in R$.
  Hence $\pair{x}{x}\in R\kj S$.
  Therefore $R\kj S$ is reflexive.
  
  Suppose $\pair{x}{y}\in R\kj S$.
  Then either $x=x'0$, $y=y'0$ and $\pair{x'}{y'}\in R$ or $x=x'1$, $y=y'1$ and $\pair{x'}{y'}\in S$.
  In either case, symmetry follows from the symmetry of $R$ or $S$.

  Suppose $\pair{x}{y}$ and $\pair{y}{z}$ are both in $R\kj S$.
  In the case that $x=x'0$, $y=y'0$ and $\pair{x'}{y'}\in R$, and that $z=z'0$ and $\pair{y'}{z'}\in R$, then by the transitivity of $R$, $\pair{x'}{z'}\in R$, so $\pair{x}{z}\in R\kj S$.
  The argument is similar in the case that $x=x'1$, $y=y'1$ and $z=z'1$.
  It is a contradiction for the other two cases to exist, since $y$ cannot be equal to both $y'0$ and $y'1$.

  Since $R\kj S$ is reflexive, symmetric and transitive, $R\kj S$ is an equivalence relation.
\end{proof}

\begin{proposition}\label{prop:symdiff}
  Symmetric difference of equivalence relations preserves symmetry.
\end{proposition}
\begin{proof}
  Let $R$ and $S$ be equivalence relations.
  Let $(x,y)\in(R\symdiff S)$.
  In the case that $(x,y)\in R$ and $(x,y)\notin S$, then $(y,x)\in R$.
  If $(y,x)$ were in $S$, then $(x,y)$ would also be in $S$, by symmetry, but this is a contradiction.
  Hence $(y,x)\in R$ and $(y,x)\notin S$.
  The argument for the other case is symmetric.
  Therefore $(x,y)\in(R\symdiff S)\implies (y,x)\in(R\symdiff S)$.
\end{proof}

\begin{definition}
  Let $r\colon\mathbb{N}\to\mathbb{N}$ be a computable function such that $r(m)>m$ for all $m$.
  Define the set $G[r]$ as
  \begin{displaymath}
    G[r]=\{x\in\Sigma^*|r^n(0)\leq|x|<r^{n+1}(0) \plain{for some even} n\}
  \end{displaymath}
  where $r^n(m)$ denotes the $n$-fold application of $r$ to $m$:
  \begin{displaymath}
    \overbrace{r\circ r\circ r\circ\cdots\circ r}^{n \plain{times}}(m)
  \end{displaymath}
  $G[r]$ is called the \defn{gap language} generated by $r$.
\end{definition}

\begin{lemma}\label{lem:gap_p}
  If $r$ is time constructible, then $G[r]\in\P$.
\end{lemma}
\begin{proof}
\end{proof}

We will denote the Cartesian product $G[r]\times G[r]$ by the slightly more succinct ${G[r]}^2$, and $\overline{G[r]}\times\overline{G[r]}$ by $\overline{G[r]}^2$.
Elements of ${G[r]}^2$ are pairs of strings whose lengths are in the ``even gaps'' of $r$, while elements of $\overline{G[r]}^2$ are pairs of strings whose lengths are in the ``odd gaps'' of $r$.

\begin{lemma}
  ${G[r]}^2$ and $\overline{G[r]}^2$ are \defn{partial equivalence relations} (that is, they are symmetric and transitive).
\end{lemma}
\begin{proof}
  We will prove the theorem for ${G[r]}^2$; a symmetric argument proves the theorem for $\overline{G[r]}^2$.

  Let $x,y\in\Sigma^*$.
  Suppose $\pair{x}{y}\in {G[r]}^2$, so the lengths of $x$ and $y$ are both in an even gap of $r$.
  Then $\pair{y}{x}\in{G[r]}^2$, so ${G[r]}^2$ is symmetric.
  Now let $z\in\Sigma^*$ and suppose also that $\pair{y}{z}\in {G[r]}^2$.
  Then $y$ and $z$ are both in an even gap of $r$, so $x$, $y$ and $z$ are all in some even gap of $r$, and hence $\pair{x}{z}\in {G[r]}^2$.
  Therefore ${G[r]}^2$ is transitive.
\end{proof}

\begin{theorem}\label{thm:diag}
  Let $R_1$ and $R_2$ be decidable equivalence relations, and let $\mathcal{C}_1$ and $\mathcal{C}_2$ be classes of decidable equivalence relations such that:
  \begin{enumerate}
  \item $R_1\notin\mathcal{C}_1$
  \item $R_2\notin\mathcal{C}_2$
  \item $\mathcal{C}_1$ and $\mathcal{C}_2$ are computably enumerable
  \item $\mathcal{C}_1$ and $\mathcal{C}_2$ are closed under finite variations
  \end{enumerate}
  Then there exists a decidable equivalence relation $R$ such that:
  \begin{enumerate}
  \item $R\notin \mathcal{C}_1$
  \item $R\notin \mathcal{C}_2$
  \item $R\kr R_1\kj R_2$
  \end{enumerate}
\end{theorem}
\begin{proof} %% TODO use \pair command for pairs
  Let $P_1, P_2, \ldots$ and $Q_1, Q_2, \ldots$ be enumerations of Turing machines deciding the languages in $\mathcal{C}_1$ and $\mathcal{C}_2$ respectively.
  Define the functions
  \begin{eqnarray*}
    r_1(n)=\underset{i\leq n}{max}\{|z_{i,n}|\}+1 & \text{and} &
    r_2(n)=\underset{i\leq n}{max}\{|z'_{i,n}|\}+1
  \end{eqnarray*}
  where $z_{i,n}$ is the smallest word in $\Sigma^*$ such that there exists an $x\in\Sigma^*$, with $n<|x|\leq|z_{i,n}|$, such that $(z_{i,n}, x)\in(L(P_i)\symdiff R_1)$, and $z'_{i,n}$ is the smallest word in $\Sigma^*$ such that there exists an $x'\in\Sigma^*$, with $n<|x'|\leq|z'_{i,n}|$, such that $(z'_{i,n}, x')\in(L(Q_i)\symdiff R_2)$.
  Note that it also suffices to find an $x$ and $x'$ such that $(x, z_{i,n})\in(L(P_i)\symdiff R_1)$ and $(x', z'_{i,n})\in(L(Q_i)\symdiff R_2)$, since symmetric difference on equivalence relations preserves symmetry by \autoref{prop:symdiff}.

  We claim that $z_{i,n}$ and $z'_{i,n}$ always exist.
  Assume that no such $z_{i,n}$ exists, so there are no words such that there exists an $x\in\Sigma^*$, with $n<|x|\leq|z_{i,n}|$, such that $(z_{i,n}, x)\in(L(P_i)\symdiff R_1)$.
  Therefore, there are no pairs in $(L(P_i)\symdiff R_1)$ with both elements of length greater than $n$.
  Then there are a finite number of pairs in $L(P_i)\symdiff R_1$, so $R_1$ is a finite variation of $L(P_i)$.
  Since $\mathcal{C}_1$ is closed under finite variations, $R_1\in\mathcal{C}_1$.
  This is a contradiction with the hypothesis that $R_1\notin\mathcal{C}_1$.
  Therefore such a $z_{i,n}$ always exists.
  The argument that $z'_{i,n}$ always exists is similar.

  Since $L(P_i)$ and $L(Q_i)$ are decidable for all $i$, and since $R_1$ and $R_2$ are decidable, so are $L(P_i)\symdiff R_1$ and $L(Q_i)\symdiff R_2$.
  For each $n$, there is a procedure which always halts and which computes $z_{i,n}$ (using a sort of linear search).
  A similar procedure computes $z'_{i,n}$.
  The procedure which computes the maximum of a finite set of numbers and which adds one to that value always halts as well, so $r_1$ and $r_2$ are total computable functions.

  Let $r\ge max(r_1,r_2)$ be a non-decreasing time constructible function, which exists by \autoref{nondec}.
  Now for all $n$ and all $i\leq n$, each element of the pairs $(z_{i,n},x)$ have length between $n$ and $r_1(n)$, by construction.
  The same is true for $(z'_{i,n}, x')$ between $n$ and $r_2(n)$.
  Notice that $(z_{i,n}, x)$ and $(z'_{i,n}, x')$ are ``witnesses'' that $R_1\neq L(P_i)$ and $R_2\neq L(Q_i)$ respectively.
  Hence for all $n$, there are some witnesses between $n$ and $r(n)$ that for all $i\leq n$, $R_1\neq L(P_i)$ and $R_2\neq L(Q_i)$.

  Define $R=({G[r]}^2\cap R_1)\cup(\overline{G[r]}^2\cap R_2)$, so $R$ is equal to pairs of $R_1$ in the ``even gaps'' of r and $R$ is equal to pairs of $R_2$ in the ``odd gaps'' of $r$.
  It remains to show that $R$ is an equivalence relation which satisfies the properties stated in the theorem.

  First we show that $R$ is indeed an equivalence relation.
  Symmetry and transitivity follow from the symmetry and transitivity of $R_1$, $R_2$, ${G[r]}^2$ and ${G[r]}^2$.
  To show reflexivity, suppose $x\in G[r]$.
  Hence $\pair{x}{x}\in {G[r]}^2$.
  Since $R_1$ is an equivalence relation, $\pair{x}{x}\in R_1$.
  Therefore $\pair{x}{x}\in({G[r]}^2\cap R_1)$, so $\pair{x}{x}\in R$.
  The argument for the case that $x\in\overline{G[r]}$ is similar.
  Therefore $R$ is reflexive, symmetric and transitive.

  Next we show that $R\notin\mathcal{C}_1$.
  The argument which proves $R\notin\mathcal{C}_2$ is symmetric.
  Assume $R\in\mathcal{C}_1$ in order to produce a contradiction.
  Then there exists an $i$ such that $R=L(P_i)$.
  Let $m$ be an even integer such that $r^m(0)\geq i$.
  By construction, there exists a pair $\pair{x}{z}$ such that $r^m(0)\leq|x|\leq|z|<r^{m+1}(0)$ and $\pair{x}{z}\in(L(P_i)\symdiff R_1)$.
  Since $m$ is even, $\pair{x}{z}\in {G[r]}^2$.
  Since $R$ is equal to $R_1$ where it coincides with ${G[r]}^2$, then $\pair{x}{z}\in(L(P_i)\symdiff R)$.
  This is a contradiction with the hypothesis that $R=L(P_i)$.
  Therefore $R\notin\mathcal{C}_1$.

  Finally, we show that $R\kr R_1\kj R_2$.
  Since $G[r]\in\P$ by \autoref{lem:gap_p}, the function defined by
  \begin{displaymath}
    f(x)=
    \begin{cases}
      x0 & \text{if}\, x\in G[r]\\
      x1 & \text{if}\, x\notin G[r]\\
    \end{cases}
  \end{displaymath}
  is a polynomial time computable function.
  
  To show that $f$ computes the reduction from $R$ to $R_1\kj R_2$ correctly, suppose first that $\pair{x}{y}\in R$, so $\pair{x}{y}$ is in either $({G[r]}^2\cap R_1)$ or $\overline{G[r]}^2\cap R_2)$.
  In the former case, both $x$ and $y$ are in $G[r]$, so $f(x)=x0$ and $f(y)=y0$, and both $x$ and $y$ are in $R_1$, so $\pair{x0}{y0}=\pair{f(x)}{f(y)}\in R_1$.
  The argument for the latter case is symmetric.

  For the converse, suppose $\pair{f(x)}{f(y)}\in R_1\kj R_2$.
  Then $f(x)$ and $f(y)$ either both end with $0$ or both end with $1$.
  In the case that both end with $0$, then there exist some strings $w_x$ and $w_y$ such that $f(x)=w_x0$, $f(y)=w_y0$ and $\pair{w_x}{w_y}\in R_1$.
  By construction of $f$, $w_x$ must equal the input $x$ and $w_y$ must equal the input $y$, so $\pair{x}{y}\in R_1$.
  Also by construction, $f(x)=x0$ if and only if $x\in G[x]$ and $f(y)=y0$ if and only if $y\in G[x]$, so $\pair{x}{y}\in{G[r]}^2$.
  Hence $\pair{x}{y}\in({G[r]}^2\cap R_1\subseteq R$.
  The argument for the case that both $f(x)$ and $f(y)$ end with $1$ is symmetric, and shows that $\pair{x}{y}\in(\overline{G[r]}^2\cap R_2)\subseteq R$.
  Therefore $\pair{x}{y}\in R$ if and only if $\pair{f(x)}{f(y)}\in R_1\kj R_2$, so $f$ correctly computes the reduction from $R$ to $R_1\kj R_2$.

  Since we have shown that the equivalence relation $R$ satisfies the properties in the statement of the theorem, this concludes the proof.
\end{proof}

\begin{proposition}
  \mbox{} % to put the next item on a new line
  \begin{enumerate}
  \item $\PEq$ is computably enumerable.
  \item $\NPEqC$ is computably enumerable.
  \end{enumerate}
\end{proposition}
\begin{proof}
  These statements are true since any subset of a computably enumerable set is computably enumerable, and these are both subsets of $\NP$.
  Note that $\NPEqC$ may be empty.
\end{proof}

\begin{theorem}\label{thm:npeqc}
  If $\PEq\neq\NPEq$ and $\NPEqC$ is non-empty, then $\PEq\cap\NPEqC=\emptyset$.
\end{theorem}
\begin{proof}
  Assume $\PEq\cap\NPEqC\neq\emptyset$.
  Let $R$ be the \NPEq-complete equivalence relation which is also in \PEq.
  Then all problems can be kernel reduced to $R$ in polynomial time, and $R$ can be decided in polynomial time.
  Therefore, $\PEq=\NPEq$.
  This is a contradiction with the hypothesis.
  Therefore $\PEq\cap\NPEqC=\emptyset$.
\end{proof}

\begin{theorem}\label{thm:intermediary}
  \printintermediarytheorem
\end{theorem}
\begin{proof}
  The hypothesis of this theorem is the same as in \autoref{thm:npeqc}, so $\PEq\cap\NPEqC=\emptyset$.
  Let $S$ be an \NPEq-complete problem.
  Choose $R_1=S$, $R_2=\emptyset$, $\mathcal{C}_1=\PEq$ and $\mathcal{C}_2=\NPEqC$.
  Then by \autoref{thm:diag}, there exists an equivalence relation $R$ which is in neither $\NPEqC$ nor $\PEq$, but which kernel reduces to $S\kj\emptyset$.
  Since $S\kj\emptyset$ trivially kernel reduces to $S$, and since polynomial time kernel reductions compose by \autoref{prop:compose}, $R\kr S$.
  Since $\NPEq$ is closed under polynomial time kernel reductions, $R\in\NPEq$.
\end{proof}

\bibliographystyle{amsalpha} \bibliography{references}

\end{document}
