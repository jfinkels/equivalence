%% definitions.tex - definitions of kernel reductions and equivalence classes
%%
%% Copyright 2010, 2011, 2012, 2014, 2015 Jeffrey Finkelstein.
%%
%% This LaTeX markup document is made available under the terms of the Creative
%% Commons Attribution-ShareAlike 4.0 International License,
%% https://creativecommons.org/licenses/by-sa/4.0/.
\section{Definitions of \texorpdfstring{\NPEq}{NPEq}}
\label{sec:definitions}
% Foreword
%
% context (focus on anyone) why now? - current situation, and why the need is so important
The main property of languages in $\NP$ is that membership in each language is verifiable in polynomial time, given a witness to the membership.
% need (focus on readers) why you? - why this is relevant to the reader, and why something needed to be done
Many important equivalence problems are in $\NP$, and some are even $\NP$-complete, but these are complete under traditional many-one reductions, not kernel reductions.
%%% relevant existing work, given as part of the need
% task (focus on author) why me? - what was undertaken to address the need
We wish to define $\NPEq$ as the class of equivalence problems that are efficiently verifiable, just as we define $\NP$ as the class of all computational problems.
One way to define $\NPEq$ is simply as the subclass of $\NP$ that includes only equivalence problems.
% object (focus on document) why this document - what the document covers
This section provides some other possible definitions based on our intuition about ``efficiently verifiable'' equivalence problems and compares those definitions.

% Summary
%
% findings (focus on author) what? - what the work revealed when performing the task
%%% We are unable to show that any of the definitions below are equivalent to the definition of $\NPEq$ as the subclass of $\NP$ containing only equivalence problems (except the corresponding definition based on an $\NP$-verifier).
We show that the alternative definitions of $\NPEq$ form a hierarchy below $\NPEq$ as defined above.
In other words, $\NPEq$ is the most general class of efficiently verifiable equivalence problems.
% conclusion (focus on readers) so what? - what the findings mean for the audience
When attempting to prove that there are complete problems in $\NPEq$ under kernel reductions, we must therefore use this most general definition.
%%For now, we must use that definition when discussing completeness in $\NPEq$ under kernel reductions.
% perspective (focus on anyone) what now? - what should be done next
It remains to show whether any of the (non-equal) alternative definitions are distinct, and whether any of them has a complete problem under kernel reductions.

For the sake of brevity, in all definitions below, when we write $\exists w$, we mean $\exists w$ with length polynomially bounded with respect to the length of $x$, $y$, or the pair $\pair{x}{y}$ (depending on the requirements of the particular definition).

The first definition is the analog of the fundamental definition of $\NP$; it is the formal definition of the class $\NPEq$ introduced in the previous section.

\begin{definition}\label{def:npeq1}
  An equivalence relation $R$ is in $\NPEq$ if there is a polynomial $p$ and a nondeterministic Turing machine $N$ such that for each $x$ and $y$, the machine $N$ halts in time $p(\left|\pair{x}{y}\right|)$ and
  \begin{displaymath}
    \pair{x}{y} \in R \iff N(\pair{x}{y}) \text{ accepts}.
  \end{displaymath}
\end{definition}
%% \begin{definition}\label{def:npeq2}
%%   An equivalence relation $R$ is in $\NPEqTwo$ if there exists a language $L\in\P$ such that
%%   \begin{displaymath}
%%     \pair{x}{y}\in R\iff \exists w\colon \pair{\pair{x}{y}}{w}\in L
%%   \end{displaymath}
%% \end{definition}

Just as there is a definition of $\NP$ using polynomial-time verifiers, there is an equivalent definition for $\NPEq$ using polynomial-time verifiers.
However, this definition feels a bit unnatural when dealing with equivalence relations, since the witness language would be a relation (of the form ``$(x, y)$ relates to $w$''), but not an \emph{equivalence} relation.
The next two definitions attempt to require that the witness language is itself an equivalence relation, instead of an arbitrary language in $\P$.
Each of these ``witness equivalence relations'' is a set of pairs of pairs, in which each inner pair includes a witness string.

\begin{definition}\label{def:npeq3}
  Suppose $R'$ is an equivalence relation in $\PEq$.
  An equivalence relation $R$ is a \emph{two-witness projection} of $R'$ if
  \begin{displaymath}
    \pair{x}{y} \in R \iff \exists w_x, w_y \colon \pair{\pair{x}{w_x}}{\pair{y}{w_y}} \in R'\kern-0.3em.
  \end{displaymath}
  The class $\Proj_2$ is the collection of all two-witness projections of equivalence relations in $\PEq$.
\end{definition}

\begin{definition}\label{def:npeq4}
  Suppose $R'$ is an equivalence relation in $\PEq$.
  An equivalence relation $R$ is a \emph{one-witness projection} of $R'$ if
  \begin{displaymath}
    \pair{x}{y} \in R \iff \exists w \colon \pair{\pair{x}{w}}{\pair{y}{w}} \in R'\kern-0.3em.
  \end{displaymath}
  The class $\Proj_1$ is the collection of all one-witness projections of equivalence relations in $\PEq$.
\end{definition}

The next two definitions attempt to allow the possibility of not just a simple string which witnesses the equivalence of $x$ and $y$, but a ``witness function'' which may map $x$ and $y$, along with witness strings, to an equivalence relation in \PEq.

\begin{definition}\label{def:npeq5}
  Suppose $R$ and $R'$ are equivalence relations.
  A function $f$ is a \emph{nondeterministic polynomial-time two-witness kernel reduction} from $R$ to $R'$ if $f$ is in $\FP$ and
  \begin{displaymath}
    \pair{x}{y} \in R \iff \exists w_x, w_y \colon \pair{f(x, w_x)}{f(y, w_y)} \in R'\kern-0.3em.
  \end{displaymath}
  The class $\Cl_2$ is the closure of $\PEq$ under these reductions.
\end{definition}

\begin{definition}\label{def:npeq6}
  Suppose $R$ and $R'$ are equivalence relations.
  A function $f$ is a \emph{nondeterministic polynomial-time one-witness kernel reduction} from $R$ to $R'$ if $f$ is in $\FP$ and
  \begin{displaymath}
    \pair{x}{y} \in R \iff \exists w \colon \pair{f(x, w)}{f(y, w)} \in R'\kern-0.3em.
  \end{displaymath}
  The class $\Cl_1$ is the closure of $\PEq$ under these reductions.
\end{definition}

The final two definitions attempt to describe equivalence relations for which there is a ``witnessed complete invariant''\kern-0.5em,\kern0.5em which maps equivalent strings to \emph{equal} strings when given access to some witness of their equivalence.
%% We say that an equivalence relation $R$ on a universe $U$ has a \defn{one-witness complete invariant} if there is a function $f \colon U \times S \to T$ such that $(x, y) \in R$ if and only if $\exists w \in S \colon f(x, w) = f(y, w)$, and we say that it has a \defn{two-witness complete invariant} if $(x, y) \in R$ if and only if $\exists w_x, w_y\in S\colon f(x, w_x)=f(y, w_y)$.

\begin{definition}\label{def:npeq7}
  Suppose $R$ is an equivalence relation.
  A function $f$ is a \emph{nondeterministic polynomial-time two-witness complete invariant} for $R$ if $f$ is in $\FP$ and
  \begin{displaymath}
    \pair{x}{y} \in R \iff \exists w_x, w_y \colon f(x, w_x) = f(y, w_y).
  \end{displaymath}
  The class $\NKer_2$ is the collection of all equivalence relations that admit such a function.
\end{definition}

\begin{definition}\label{def:npeq8}
  Suppose $R$ is an equivalence relation.
  A function $f$ is a \emph{nondeterministic polynomial-time one-witness complete invariant} for $R$ if $f$ is in $\FP$ and
  \begin{displaymath}
    \pair{x}{y} \in R \iff \exists w \colon f(x, w) = f(y, w).
  \end{displaymath}
  The class $\NKer_1$ is the collection of all equivalence relations that admit such a function.
\end{definition}

The definitions of these complexity classes yield a chain of inclusions beginning with $\NKer_1$ and terminating with $\NPEq$.
%% \begin{figure}
%%   \caption{\label{fig:inclusions}Inclusions among possible definitions of equivalence relations verifiable in deterministic polynomial time.}
%%   \begin{center}
%%     \begin{tikzpicture}[->]
%%       \node at (0, 0) (8) {$\NPEq_8$};
%%       \node at (-1, 1) (7) {$\NPEq_7$};
%%       \node at (1, 1) (6) {$\NPEq_6$};
%%       \node at (3, 1) (4) {$\NPEq_4$};
%%       \node at (0, 2) (5) {$\NPEq_5$};
%%       \node at (2, 2) (3) {$\NPEq_3$};
%%       \node at (0, 3) (2) {$\NPEq_1$};
%%       %% \node at (2, 3) (1) {$\NPEq_1$};
%%       \draw (8) to (7);
%%       \draw (8) to (6);
%%       \draw[<->] (6) to (4);
%%       \draw (7) to (5);
%%       \draw (6) to (5);
%%       \draw[<->] (5) to (3);
%%       \draw (5) to (2);
%%       %% \draw[<->] (2) to (1);
%%     \end{tikzpicture}
%%   \end{center}
%% \end{figure}
\begin{theorem}\label{thm:definitions}
  $\Proj_1 = \Cl_1 = \NKer_1 \subseteq \NKer_2 \subseteq \Proj_2 = \Cl_2 \subseteq \NPEq$.
\end{theorem}
\begin{sketch}
  $\Proj_1 \subseteq \Cl_1$ by choosing the kernel reduction $f$ to be the identity function.
  $\Cl_1 \subseteq \NKer_1$ by choosing the complete invariant $f'$ to be
  \begin{equation*}
    f'(x, w') =
    \begin{cases}
      w'0 & \text{if } \pair{f(x, v)}{f(y, v)} \in R', \text{ where } w' = (y, v) \\
      x1 & \text{otherwise},
    \end{cases}
  \end{equation*}
  where $f$ is the kernel reduction.
  $\NKer_1 \subseteq \Proj_1$ by choosing $R'$ to be the equality relation after an application of the complete invariant $f$ to both the left pair and the right pair in the relation.

  $\NKer_1 \subseteq \NKer_2$ by choosing both $w_x$ and $w_y$ to be the witness $w$.
  $\NKer_2 \subseteq \Cl_2$ by choosing $R'$ to be the equality relation.

  $\Proj_2 \subseteq \Cl_2$ by choosing $f$ to be the identity function.
  $\Cl_2 \subseteq \Proj_2$ by hardcoding the function $f$ into the relation $R'$.

  $\Proj_2 \subseteq \NPEq$ by defining $N$ to nondeterministically choose $w_x$ and $w_y$ then verify that $\pair{x}{w_x}$ and $\pair{y}{w_y}$ are related under $R'$.
\end{sketch}

We are unable to show $\Cl_2 \subseteq \NKer_2$ using the technique that shows $\Cl_1 \subseteq \NKer_1$ because the complete invariant $f'$ cannot access both of the necessary witnesses for the kernel reduction $f$ in a symmetric way.
The best we can do is show this inclusion under an assumption.

Our one-witness and two-witness complete invariants are generalizations of the deterministic complete invariant, as defined in \autoref{sec:preliminaries}.
In \autocite{fg11}, the authors define the class $\Ker$ as the set of all equivalence relations $R$ that have a polynomial-time computable complete invariant.
They provide evidence that $\Ker$ and $\PEq$ are different by showing that equality of the two classes implies some unlikely collapses in ``higher'' complexity classes.
Unfortunately, we are only able to show that $\Cl_2 \subseteq \NKer_2$ under the assumption that $\Ker = \PEq$.

\begin{corollary}
  If $\Ker = \PEq$, then $\Proj_2 = \Cl_2 = \NKer_2$.
\end{corollary}
\begin{proof}
  By the previous theorem, it suffices to show $\Proj_2 \subseteq \NKer_2$.
  Suppose $R \in \Proj_2$, so there is an $R' \in \PEq$ such that $\pair{x}{y} \in R$ if and only if there are $w_x$ and $w_y$ such that $\pair{\pair{x}{w_x}}{\pair{y}{w_y}} \in R'$.
  Since $\Ker = \PEq$, there is a function $f \in \FP$ such that $\pair{\pair{x}{w_x}}{\pair{y}{w_y}} \in R'$ if and only if $f(x, w_x) = f(y, w_y)$.
  Thus there is a function $f$ such that $\pair{x}{y} \in R$ if and only if there are $w_x$ and $w_y$ such that $f(x, w_x) = f(y, w_y)$.
  Therefore $R \in \NKer_2$.
\end{proof}

%% %% These are stated in the introduction, so don't need to be restated here.
%% \begin{openproblem}
%%   Does one of the complexity classes defined here have a complete problem under $\kr$ reductions?
%% \end{openproblem}
%% \begin{openproblem}
%%   Can we prove equality for the remaining classes, or are some of these classes provably distinct?
%% \end{openproblem}
