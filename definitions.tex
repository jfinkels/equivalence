%% definitions.tex - definitions of kernel reductions and equivalence classes
%%
%% Copyright 2010, 2011, 2012, 2014 Jeffrey Finkelstein.
%%
%% This LaTeX markup document is made available under the terms of the Creative
%% Commons Attribution-ShareAlike 4.0 International License,
%% https://creativecommons.org/licenses/by-sa/4.0/.
\section{Definitions of \texorpdfstring{\NPEq}{NPEq}}
\label{sec:definitions}
% Foreword
%
% context (focus on anyone) why now? - current situation, and why the need is so important
The main property of languages in $\NP$ is that membership in each language is verifiable in polynomial time, given a witness to the membership.
% need (focus on readers) why you? - why this is relevant to the reader, and why something needed to be done
Many important equivalence problems are in $\NP$, and some are even $\NP$-complete, but these are complete under traditional many-one reductions, not kernel reductions.
%%% relevant existing work, given as part of the need
% task (focus on author) why me? - what was undertaken to address the need
We wish to define $\NPEq$ as the class of equivalence problems that are efficiently verifiable, just as we define $\NP$ as the class of all computational problems.
One way to define $\NPEq$ is simply as the subclass of $\NP$ that includes only equivalence problems.
% object (focus on document) why this document - what the document covers
This section provides some other possible definitions based on our intuition about ``efficiently verifiable'' equivalence problems and compares those definitions.

% Summary
%
% findings (focus on author) what? - what the work revealed when performing the task
%%% We are unable to show that any of the definitions below are equivalent to the definition of $\NPEq$ as the subclass of $\NP$ containing only equivalence problems (except the corresponding definition based on an $\NP$-verifier).
We show that the alternative definitions of $\NPEq$ form a hierarchy below $\NPEq$ as defined above.
In other words, $\NPEq$ is the most general class of efficiently verifiable equivalence problems.
% conclusion (focus on readers) so what? - what the findings mean for the audience
When attempting to prove that there are complete problems in $\NPEq$ under kernel reductions, we must therefore use this most general definition.
%%For now, we must use that definition when discussing completeness in $\NPEq$ under kernel reductions.
% perspective (focus on anyone) what now? - what should be done next
It remains to show whether any of the (non-equal) alternative definitions are distinct, and whether any of them has a complete problem under kernel reductions.

For the sake of brevity, in all definitions below, when we write $\exists w$, we mean $\exists w$ with length polynomially bounded with respect to the length of $x$, $y$, or the pair $\pair{x}{y}$ (depending on the requirements of the particular definition).

The first two definitions are analogs of the two fundamental definitions of \NP.
$\NPEq_1$ coincides with the definition of $\NPEq$ we gave in the previous section.
\begin{definition}\label{def:npeq1}
  An equivalence relation $R$ is in $\NPEqOne$ if there exists a non-deterministic Turing machine, call it $N$, which halts in time polynomial in the length of the input, such that
  \begin{displaymath}
    \pair{x}{y}\in R\iff N(\pair{x}{y})\plain{accepts}
  \end{displaymath}
\end{definition}
\begin{definition}\label{def:npeq2}
  An equivalence relation $R$ is in $\NPEqTwo$ if there exists a language $L\in\P$ such that
  \begin{displaymath}
    \pair{x}{y}\in R\iff \exists w\colon \pair{\pair{x}{y}}{w}\in L
  \end{displaymath}
\end{definition}

The next two definitions attempt to require that the witness language is itself an equivalence relation, instead of an arbitrary language in $\P$, as in \autoref{def:npeq2}.
Each of these ``witness equivalence relations'' is a set of pairs of pairs, in which each inner pair includes a witness string.
\begin{definition}\label{def:npeq3}
  An equivalence relation $R$ is in $\NPEqThree$ if there exists an equivalence relation $R'\in \PEq$ such that
  \begin{displaymath}
    \pair{x}{y}\in R\iff \exists w_x,w_y\colon \pair{\pair{x}{w_x}}{\pair{y}{w_y}}\in R'
  \end{displaymath}
\end{definition}
\begin{definition}\label{def:npeq4}
  An equivalence relation $R$ is in $\NPEqFour$ if there exists an equivalence relation $R'\in \PEq$ such that
  \begin{displaymath}
    \pair{x}{y}\in R\iff \exists w\colon \pair{\pair{x}{w}}{\pair{y}{w}}\in R'
  \end{displaymath}
\end{definition}

The next two definitions attempt to allow the possibility of not just a simple string which witnesses the equivalence of $x$ and $y$, but a ``witness function'' which may map $x$ and $y$, along with witness strings, to an equivalence relation in \PEq.
\begin{definition}\label{def:npeq5}
  An equivalence relation $R$ is in $\NPEqFive$ if there exists an equivalence relation $R'\in \PEq$ and a function $f\in\FP$ such that
  \begin{displaymath}
    \pair{x}{y}\in R\iff \exists w_x,w_y\colon \pair{f(x, w_x)}{f(y, w_y)}\in R'
  \end{displaymath}
\end{definition}
\begin{definition}\label{def:npeq6}
  An equivalence relation $R$ is in $\NPEqSix$ if there exists an equivalence relation $R'\in \PEq$ and a function $f\in\FP$ such that
  \begin{displaymath}
    \pair{x}{y}\in R\iff \exists w\colon \pair{f(x, w)}{f(y, w)}\in R'
  \end{displaymath}
\end{definition}

The final two definitions attempt to describe equivalence relations for which there is a ``witnessed complete invariant''\kern-0.5em, which maps equivalent strings to equal strings when given access to some witness of their equivalence.
We say that an equivalence relation $R$ on a universe $U$ has a \defn{one-witness complete invariant} if there exists a function $f\colon U\times S\to T$ such that $(x,y)\in R$ if and only if $\exists w\in S\colon f(x, w)=f(y, w)$, and we say that it has a \defn{two-witness complete invariant} if $(x, y)\in R$ if and only if $\exists w_x, w_y\in S\colon f(x, w_x)=f(y, w_y)$.
\begin{definition}\label{def:npeq7}
  An equivalence relation $R$ is in $\NPEqSeven$ if it has a polynomial time computable two-witness complete invariant, that is, a function $f\in\FP$ such that
  \begin{displaymath}
    \pair{x}{y}\in R\iff \exists w_x, w_y\colon f(x, w_x) = f(y, w_y)
  \end{displaymath}
\end{definition}
\begin{definition}\label{def:npeq8}
  An equivalence relation $R$ is in $\NPEqEight$ if it has a polynomial time computable one-witness complete invariant, that is, a function $f\in\FP$ such that
  \begin{displaymath}
    \pair{x}{y}\in R\iff \exists w\colon f(x, w) = f(y, w)
  \end{displaymath}
\end{definition}

The definitions of these complexity classes produce a hierarchy of inclusions beginning with $\NPEqEight$ at the bottom and terminating with $\NPEqOne$ at the top (\autoref{fig:inclusions}).
\begin{figure}
  \caption{\label{fig:inclusions}Inclusions among possible definitions of equivalence relations verifiable in deterministic polynomial time.}
  \begin{center}
    \begin{tikzpicture}[->]
      \node at (0, 0) (8) {$\NPEq_8$};
      \node at (-1, 1) (7) {$\NPEq_7$};
      \node at (1, 1) (6) {$\NPEq_6$};
      \node at (3, 1) (4) {$\NPEq_4$};
      \node at (0, 2) (5) {$\NPEq_5$};
      \node at (2, 2) (3) {$\NPEq_3$};
      \node at (0, 3) (2) {$\NPEq_2$};
      \node at (2, 3) (1) {$\NPEq_1$};
      \draw (8) to (7);
      \draw (8) to (6);
      \draw[<->] (6) to (4);
      \draw (7) to (5);
      \draw (6) to (5);
      \draw[<->] (5) to (3);
      \draw (5) to (2);
      \draw[<->] (2) to (1);
    \end{tikzpicture}
  \end{center}
\end{figure}
\begin{theorem}\label{thm:definitions}\mbox{}
  \begin{enumerate}
  \item $\NPEqOne = \NPEqTwo$,
  \item $\NPEqEight \subseteq \NPEqSix$ and $\NPEqSeven \subseteq \NPEqFive$,
  \item $\NPEqEight \subseteq \NPEqSeven$ and $\NPEqSix \subseteq \NPEqFive$,
  \item $\NPEqSix = \NPEqFour$ and $\NPEqFive = \NPEqThree$.
  \end{enumerate}
\end{theorem}
\begin{sketch}\mbox{}
  \begin{enumerate}
  \item Follows immediately from the standard definitions of \NP.
  \item Choose the relation $R'$ in the definitions of $\NPEqSix$ and $\NPEqFive$ to be the equality relation.
  \item Choose $w_x$ and $w_y$ in $\NPEqSeven$ and $\NPEqFive$ to be equal to the $w$ from $\NPEqEight$ and $\NPEqSix$.
  \item
    For the forward direction, hard-code the function $f$ from the definitions of $\NPEqSix$ and $\NPEqFive$ into the relation $R'$ in the definitions of $\NPEqFour$ and $\NPEqThree$.
    For the reverse direction, keep the relation $R'$ and choose the identity function to be $f$ in the definitions of $\NPEqSix$ and $\NPEqFive$.
    \qedhere
  \end{enumerate}
\end{sketch}

In \autocite{fg11}, the authors define the class $\Ker$ as the set of all equivalence relations $R$ that have a polynomial time computable \emph{complete invariant} (that is, a function $f \in \FP$ such that $(x, y) \in R$ if and only if $f(x) = f(y)$).
Our one-witness and two-witness complete invariants are special cases of the general complete invariant.
They provide evidence that $\Ker$ and $\PEq$ are different by showing that equality of the two classes implies some unlikely collapses in ``higher'' complexity classes.
Our intuition is that $\NPEq_8$ is different from $\NPEq_6$ (and similarly that $\NPEq_7$ is different from $\NPEq_5$), since the former requires equality while the latter requires only equivalence, in a very broad sense.
If this intuition is correct, then we can provide further evidence that $\Ker$ and $\PEq$ are different.

\begin{theorem}
  If $\Ker = \PEq$, then $\NPEq_8 = \NPEq_6 = \NPEq_4$ and $\NPEq_7 = \NPEq_5 = \NPEq_3$.
\end{theorem}
\begin{proof}
  Since $\NPEq_8 \subseteq \NPEq_6 = \NPEq_4$ and $\NPEq_7 \subseteq \NPEq_5 = \NPEq_3$ unconditionally, it suffices to show $\NPEq_4 \subseteq \NPEq_8$ and $\NPEq_3 \subseteq \NPEq_7$.

  First, suppose $R \in \NPEq_3$, so there is an $R' \in \PEq$ such that $\pair{x}{y} \in R$ if and only if there are $w_x$ and $w_y$ such that $\pair{\pair{x}{w_x}}{\pair{y}{w_y}} \in R'$.
  Since $\Ker = \PEq$, there is a function $f \in \FP$ such that $\pair{\pair{x}{w_x}}{\pair{y}{w_y}} \in R'$ if and only if $f(x, w_x) = f(y, w_y)$.
  Thus there is a function $f$ such that $\pair{x}{y} \in R$ if and only if there are $w_x$ and $w_y$ such that $f(x, w_x) = f(y, w_y)$.
  Therefore $R \in \NPEq_7$.

  The proof that $\NPEq_4 \subseteq \NPEq_8$ is similar; the only exception is that there is a single witness $w$ instead of two witnesses $w_x$ and $w_y$.
\end{proof}

%% %% These are stated in the introduction, so don't need to be restated here.
%% \begin{openproblem}
%%   Does one of the complexity classes defined here have a complete problem under $\kr$ reductions?
%% \end{openproblem}
%% \begin{openproblem}
%%   Can we prove equality for the remaining classes, or are some of these classes provably distinct?
%% \end{openproblem}
