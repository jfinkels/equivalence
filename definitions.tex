%%%%
%% definitions.tex
%%
%% Copyright 2011, 2012 Jeffrey Finkelstein
%%
%% Except where otherwise noted, this work is made available under the terms of
%% the Creative Commons Attribution-ShareAlike 3.0 license,
%% http://creativecommons.org/licenses/by-sa/3.0/.
%%
%% You are free:
%%    * to Share — to copy, distribute and transmit the work
%%    * to Remix — to adapt the work
%% Under the following conditions:
%%    * Attribution — You must attribute the work in the manner specified by
%%    the author or licensor (but not in any way that suggests that they
%%    endorse you or your use of the work).
%%    * Share Alike — If you alter, transform, or build upon this work, you may
%%    distribute the resulting work only under the same, similar or a 
%%    compatible license.
%%    * For any reuse or distribution, you must make clear to others the 
%%    license terms of this work. The best way to do this is with a link to the
%%    web page http://creativecommons.org/licenses/by-sa/3.0/.
%%    * Any of the above conditions can be waived if you get permission from
%%    the copyright holder.
%%    * Nothing in this license impairs or restricts the author's moral rights.
%%%%
\section{Definitions of \texorpdfstring{\NPEq}{NPEq}}
\label{sec:definitions}

In this section we examine possible alternate definitions of \NPEq.
The main property of languages in $\NP$ is that membership in each language is verifiable in polynomial time, given a witness to the membership.
We propose here several possible definitions of $\NPEq$ in order to determine which make sense, which are too restrictive, and which are equivalent.

For the sake of brevity, in all definitions below, when we write $\exists w$, we mean $\exists w$ with length polynomially bounded with respect to the length of $x$, $y$, or the pair $\pair{x}{y}$ (depending on the requirements of the particular definition).

The first two definitions are analogs of the two fundamental definitions of \NP.
\begin{definition}\label{def:npeq1}
  An equivalence relation $R$ is in $\NPEqOne$ if there exists a non-deterministic Turing machine, call it $N$, which halts in time polynomial in the length of the input, such that
  \begin{displaymath}
    \pair{x}{y}\in R\iff N(\pair{x}{y})\plain{accepts}
  \end{displaymath}
\end{definition}
\begin{definition}\label{def:npeq2}
  An equivalence relation $R$ is in $\NPEqTwo$ if there exists a language $L\in\P$ such that
  \begin{displaymath}
    \pair{x}{y}\in R\iff \exists w\colon \pair{\pair{x}{y}}{w}\in L
  \end{displaymath}
\end{definition}

The next two definitions attempt to require that the witness language is itself an equivalence relation, instead of an arbitrary language in $\P$, as in \autoref{def:npeq2}.
\begin{definition}\label{def:npeq3}
  An equivalence relation $R$ is in $\NPEqThree$ if there exists an equivalence relation $R'\in \PEq$ such that
  \begin{displaymath}
    \pair{x}{y}\in R\iff \exists w_x,w_y\colon \pair{\pair{x}{w_x}}{\pair{y}{w_y}}\in R'
  \end{displaymath}
\end{definition}
\begin{definition}\label{def:npeq4}
  An equivalence relation $R$ is in $\NPEqFour$ if there exists an equivalence relation $R'\in \PEq$ such that
  \begin{displaymath}
    \pair{x}{y}\in R\iff \exists w\colon \pair{\pair{x}{w}}{\pair{y}{w}}\in R'
  \end{displaymath}
\end{definition}

The next two definitions attempt to allow the possibility of not just a simple string which witnesses the equivalence of $x$ and $y$, but a ``witness function'' which may map $x$ and $y$, along with witness strings, to an equivalence relation in \PEq.
\begin{definition}\label{def:npeq5}
  An equivalence relation $R$ is in $\NPEqFive$ if there exists an equivalence relation $R'\in \PEq$ and a function $f\in\FP$ such that
  \begin{displaymath}
    \pair{x}{y}\in R\iff \exists w_x,w_y\colon \pair{f(x, w_x)}{f(y, w_y)}\in R'
  \end{displaymath}
\end{definition}
\begin{definition}\label{def:npeq6}
  An equivalence relation $R$ is in $\NPEqSix$ if there exists an equivalence relation $R'\in \PEq$ and a function $f\in\FP$ such that
  \begin{displaymath}
    \pair{x}{y}\in R\iff \exists w\colon \pair{f(x, w)}{f(y, w)}\in R'
  \end{displaymath}
\end{definition}

The final two definitions attempt to describe equivalence relations for which there is a ``witnessed complete invariant'', which maps equivalent strings to equal strings when given access to some witness of their equivalence.
We say that an equivalence relation $R$ on a universe $U$ has a \defn{one-witness complete invariant} if there exists a function $f\colon U\times S\to T$ such that $(x,y)\in R$ if and only if $\exists w\in S\colon f(x, w)=f(y, w)$, and we say that it has a \defn{two-witness complete invariant} if $(x, y)\in R$ if and only if $\exists w_x, w_y\in S\colon f(x, w_x)=f(y, w_y)$.
\begin{definition}\label{def:npeq7}
  An equivalence relation $R$ is in $\NPEqSeven$ if it has a polynomial time computable two-witness complete invariant, that is, a function $f\in\FP$ such that
  \begin{displaymath}
    \pair{x}{y}\in R\iff \exists w_x, w_y\colon f(x, w_x) = f(y, w_y)
  \end{displaymath}
\end{definition}
\begin{definition}\label{def:npeq8}
  An equivalence relation $R$ is in $\NPEqEight$ if it has a polynomial time computable one-witness complete invariant, that is, a function $f\in\FP$ such that
  \begin{displaymath}
    \pair{x}{y}\in R\iff \exists w\colon f(x, w) = f(y, w)
  \end{displaymath}
\end{definition}

\autoref{fig:inclusions} shows the inclusions among each of the classes of equivalence relations defined above.
\begin{figure}
  \caption{\label{fig:inclusions}Inclusions among possible definitions of equivalence relations verifiable in deterministic polynomial time.}
  \begin{displaymath}
    \xymatrix{%
      \NPEqSeven \ar[r] & \NPEqFive \ar[r] & \NPEqThree \ar[r] & \NPEqTwo \ar@{<->}[r] & \NPEqOne \\
      \NPEqEight \ar[u] \ar[r] & \NPEqSix \ar[u] \ar[r] & \NPEqFour \ar[u]}
  \end{displaymath}
\end{figure}
The main ideas of these inclusions are presented in the following theorem (the complete proofs are tedious and so are omitted here).
\begin{theorem}\mbox{}
  \begin{enumerate}
  \item $\NPEqOne=\NPEqTwo$
  \item $\NPEqEight\subseteq\NPEqSix$ and $\NPEqSeven\subseteq\NPEqFive$
  \item $\NPEqSix\subseteq\NPEqFour$ and $\NPEqFive\subseteq\NPEqThree$
  \item $\NPEqEight\subseteq\NPEqSeven$, $\NPEqSix\subseteq\NPEqFive$, and $\NPEqFour\subseteq\NPEqThree$
  \item $\NPEqThree\subseteq\NPEqTwo$
  \end{enumerate}
\end{theorem}
\begin{sketch}\mbox{}
  \begin{enumerate}
  \item Follows immediately from the standard definitions of \NP.
  \item Choose the relation $R'$ in the definitions of $\NPEqSix$ and $\NPEqFive$ to be the equality relation.
  \item Hard-code the function $f$ from the definitions of $\NPEqSix$ and $\NPEqFive$ into the relation $R'$ in the definitions of $\NPEqFour$ and $\NPEqThree$.
  \item Choose $w_x$ and $w_y$ in $\NPEqSeven$, $\NPEqFive$, and $\NPEqThree$ to be equal to the $w$ from $\NPEqEight$, $\NPEqSix$, and $\NPEqFour$.
  \item Define $L=\lb\pair{\pair{x}{y}}{\pair{w_x}{w_y}}\st\pair{\pair{x}{w_x}}{\pair{y}{w_y}}\in R'\rb$, so the witness that $\pair{x}{y}\in L$ is the pair $\pair{w_x}{w_y}$.\qedhere
  \end{enumerate}
\end{sketch}

We would like to be able to show that $\NPEq_2$ (or $\NPEq_1$, though it seems more difficult) is contained in any of the other classes which have an equivalence relation as the witness language in \P, but this would require simulating an arbitrary language, which is not necessarily an equivalence relation, by some constructed equivalence relation.
We are not guaranteed anything about the structure of the arbitrary language, and it is therefore difficult to construct an equivalence relation which represents that language.

\begin{openproblem}
  Does one of the complexity classes defined here have a complete problem under $\kr$ reductions?
\end{openproblem}
\begin{openproblem}
  Are any of these possible definitions of polynomially verifiable equivalence relations equivalent? Are any of them provably distinct?
\end{openproblem}
