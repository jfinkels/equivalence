\begin{lemma}\label{lem:gi_kr_dirgi}$GI\kr DirGI$\end{lemma}
\begin{proof}
  Construct machine $M\in\FP$ on input $G=(V,E)$:\\
  \begin{algorithm}[H]
    $V'\gets V$\;
    $E'\gets\{\}$\;
    \ForEach{$(u, v)\in E$}{
      add directed edges $(u, v)$ and $(v, u)$ to $E'$
    }
    \Return{$G'=(V', E')$}
  \end{algorithm}
  Notice that this machine constructs the graph with all undirected edges of
  the input graph replaced with two complementary directed edges incident to
  both vertices.

  %% TODO complete this proof
\end{proof}

\begin{lemma}\label{lem:dirgi_kr_gi}$DirGI\kr GI$\end{lemma}
\begin{proof}
  %% TODO complete this proof
\end{proof}

\begin{corollary}\label{cor:gi_ke_dirgi}$GI\kequiv DirGI$\end{corollary}
\begin{proof}This follows directly from Lemmas \ref{lem:gi_kr_dirgi} and
  \ref{lem:dirgi_kr_gi} and Definition \ref{def:kr}.\end{proof}

From Corollary \ref{cor:gi_ke_dirgi}, we can use a kernel reduction to directed
graph isomorphism to show that a language kernel reduces to (undirected) graph
isomorphism. 

%% TODO are these kernel reductions? they must be...

%% There are other languages that are polynomial time kernel equivalent to
%% graph isomorphism, including but not limited to finite automata equivalence,
%% context-free grammar equivalence, multigraph isomorphism, hypergraph
%% isomorphism, etc.

\begin{theorem}\label{thm:rpar_gi}$R_{par}\kr GI$\end{theorem}
\begin{proof}
  Construct $M\in \FP$ on input $w\in\sigmastar$:\\
  \begin{algorithm}[H]
    $V_w\gets\{v_p, v_o\}$\;
    $E_w\gets\{\}$ \tcp{undirected edges}\;
    \For{$i=1$ \KwTo $|w|$}{
      \If{$w_i=1$}{
        \eIf{$(v_o, v_p)\in E_w$}{add undirected edge $(v_o, v_p)$ to $E_w$\;}
            {remove $(v_o, v_p)$ from $E_w$\;}
      }
    }
    \Return $G_w=(V_w, E_w)$
  \end{algorithm}

  Suppose $(x, y)\in R_{par}$, so either $x$ and $y$ both have even parity or
  $x$ and $y$ both have odd parity. 

  If $x$ and $y$ both have even parity, $x$ contains $2k$ ones and $y$ contains
  $2l$ ones, for some $k,l\in\mathbb{N}$. Since $2k$ is even, machine $M$ on
  input $x$ adds then removes the edge $(v_o, v_p)$ to and from $E_x$ an equal
  number of times. Similarly for $M$ on input $y$. Therefore $M(x)$ outputs
  $G_x=(V_x, E_x)$, where $V_x=\{v_o, v_p\}$ and $E_x=\{\}$, and $M(y)$ outputs
  $G_y=(V_y, E_y)$, where $V_y=\{v_o, v_p\}$ and $E_y=\{\}$. Then $G_x$ is
  isomorphic to $G_y$ by the identity function, $I:V_x\to V_y$, defined by
  $I(v)=v, \forall v\in V_x$.

  If $x$ and $y$ both have odd parity, $x$ contains $2k+1$ ones and $y$
  contains $2l+1$ ones, for some $k,l\in\mathbb{N}$. Since $2k+1$ is odd,
  machine $M$ on input $x$ adds edge $(v_o, v_p)$ to $E_x$ one more time than
  it removes the edge. Similarly for $M$ on input $y$. Therefore $M(x)$ outputs
  $G_x=(V_x, E_x)$, where $V_x=\{v_o, v_p\}$ and $E_x=\{(v_o, v_p)\}$, and
  $M(y)$ outputs $G_y=(V_y, E_y)$, where $V_y=\{v_o, v_p\}$ and $E_y=\{(v_o,
  v_p)\}$. Then $G_x$ is isomorphic to $G_y$ by the identity function,
  $I:V_x\to V_y$, defined by $I(v)=v, \forall v\in V_x$.

  Suppose $(x, y)\notin R_{par}$, so without loss of generality, $x$ has even
  parity and $y$ has odd parity. Then $x$ contains $2k$ ones and $y$ contains
  $2l+1$ ones, for some $k,l\in\mathbb{N}$. Since $2k$ is even, machine $M$ on
  input $x$ adds then removes the edge $(v_o, v_p)$ to and from $E_x$ an equal
  number of times. Since $2l+1$ is odd, machine $M$ on input $y$ adds edge
  $(v_o, v_p)$ to $E_y$ one more time than it removes the edge. Therefore
  $M(x)$ outputs $G_x=(V_x, E_x)$, where $V_x=\{v_o, v_p\}$ and $E_x=\{\}$, and
  $M(y)$ outputs $G_y=(V_y, E_y)$, where $V_y=\{v_o, v_p\}$ and $E_y=\{(v_o,
  v_p)\}$. Since $(v_o, v_p)\in E_y$ but $(v_o, v_p)\notin E_x$, so no
  bijection exists between $V_x$ and $V_y$ which preserves edges. Therefore,
  $G_x$ is not isomorphic to $G_y$.

  Therefore $(x, y)\in R_{par} \iff (M(x), M(y)) \in GI$, so $R_{par} \kr GI$.
\end{proof}

\begin{theorem}\label{thm:rbc_gi}$R_{bc}\kr GI$\end{theorem}
\begin{proof}
  Construct $M\in \FP$ on input $w \in \sigmastar$:\\
  \begin{algorithm}[H]
    $V_w\gets\{v_1, v_2, \ldots, v_{|w|}, v_{zero}, v_{one,0}, v_{one,1},
    v_{one,2}\}$\;
    $E_w\gets\{(v_{one,0}, v_{one,1}), (v_{one,1}, v_{one,2}), (v_{one,2},
    v_{one,0})\}$\tcp{undirected edges}\; 
    \For{$i=1$ \KwTo $|w|$}{
      \eIf{$w_i=1$}{
        add undirected edge $(v_i, v_{one,0})$ to $E_w$\;
      }{
        add undirected edge $(v_i, v_{zero})$ to $E_w$\;
      }
    }
    \Return{$G_w=(V_w, E_w)$}
  \end{algorithm}
  
  Suppose $(x, y)\in R_{bc}$, so $x$ and $y$ have the same number of ones, say
  $k\in\mathbb{N}$. Assume $|x|=|y|=n$, so both $x$ and $y$ have $n-k$
  zeros. Define $E_{w,1}=\{(v_i, v_{one,0})|i\in\{1,2,\ldots,n\}, w_i = 1\}$
  and $E_{w,0}=\{(v_i, v_{zero})|i\in\{1,2,\ldots,n\}, w_i = 0\}$, so $E_x =
  E_{x,1}\cup E_{x,0}$ and $E_y = E_{y,1} \cup E_{y,0}$ by construction. Define
  $V_{w,b}=\{v_i|w_i=b\}$, so $V_x=V_{x,1} \cup V_{x,0}$ and $V_y=V_{y,1} \cup
  V_{y,0}$. Note that $|V_{x,1}|=|V_{y,1}|=k$ and
  $|V_{x,0}|=|V_{y,0}|=n-k$. Since $|V_{x,1}|=|V_{y,1}|=k$, there exists a
  bijection between them, call it $\phi_1:V_{x,1}\to V_{y,1}$. Similarly, since
  $|V_{x,0}|=|V_{y,0}|=k$, there exists a bijection between them, call it
  $\phi_0:V_{x,0}\to V_{y,0}$. Define $\phi:V_x\to V_y$ by
  \begin{displaymath}
    \phi(v) = 
    \begin{cases}
      \phi_0(v) & \plain{if} v = v_i \plain{and} x_i = 1, \plain{for some}
      i\in\{1,\ldots,n\}\\ 
      \phi_1(v) & \plain{if} v = v_i \plain{and} x_i = 0, \plain{for some}
      i\in\{1,\ldots,n\}\\ 
      v & \plain{if} v \in \{v_{zero}, v_{one,0}, v_{one,1}, v_{one,2}\}
    \end{cases}
  \end{displaymath}
  for all $v\in V_x$. Notice that each $v_i$ ``corresponds'' to a
  single $x_i$, because each $x_i$ can be either a one or a zero,
  exclusively.

  Since the only edges in $E_x$ are the edges $(v_i, v_{one,0})$ when $x_i=1$
  and $(v_i, v_{zero})$ when $x_i=0$, then $(v_i, v_{one,0})\in E_x \iff
  (\phi(v_i), \phi(v_{one,0}))=(\phi_1(v_i), v_{one,0})\in E_y$, and $(v_i,
  v_{zero})\in E_x \iff (\phi(v_i), \phi(v_{zero})) = (\phi_0(v_i),
  v_{zero})\in E_y$. Therefore $\phi$ describes a graph isomorphism, so $G_1$
  is isomorphic to $G_2$.
  
  Suppose $(x, y)\notin R_{bc}$, so $x$ and $y$ have a different number of
  ones. Let $k$ be the number of ones in $x$, and $l$ be the number of ones in
  $y$, with $k\neq l$. Suppose without loss of generality that $k>l$. Define
  $E_{w,0}$ and $E_{w,1}$ as above. Now $|E_{x,1}|=k$ and $|E_{y,1}|=l$. Since
  $k>l$, $E_{x,1}$ has at least one more edge adjacent to the triangle created
  by the vertices $\{v_{one,0},v_{one,1},v_{one,2}\}$ than does $E_{y,1}$. Thus
  no possible bijection exists between $V_x$ and $V_y$ which preserves all
  edges. Thus $G_x$ is not isomorphic to $G_y$, so $(M(x), M(y))\notin GI$.

  Therefore $(x, y)\in R_{bc} \iff (M(x), M(y))\in GI$, so $R_{bc}\kr GI$.
\end{proof}

\begin{theorem}\label{thm:req_kr_dirgi}$R_{eq} \kr DirGI$\end{theorem}
\begin{proof}
  Construct machine $M\in\FP$ on input $w\in\sigmastar$:\\
  \begin{algorithm}[H]
    $V_w\gets\{v_1, v_2, \ldots, v_{|w|}\}$\; 
    $E_w\gets\{(v_1, v_2), (v_2, v_3), \ldots, (v_{|w|-1},
    v_{|w|})\}$\tcp*[f]{directed edges}\;
    \For{$i=1$ \KwTo $|w|$}{
      \If{$w_i=1$}{
        add vertex $v'_i$ to $V_w$\;
        add directed edge $(v_i, v'_i)$ to $E_w$\;
      }
    }
    \Return{$G_w=(V_w, E_w)$}
  \end{algorithm}
  Notice that this machine constructs a ``spine'' of vertices, with an extra
  vertex $v'_i$ and directed edge $(v_i, v'_i)$ adjacent to the spine whenever
  $w_i$ is a one, $\forall i\in\{1,2,\ldots,|w|\}$.

  Suppose $(x, y)\in R_{eq}$, so $x=y$. Then $M(x)$ and $M(y)$ produce the same
  graph, so $G_x$ is isomorphic to $G_y$ by the identity mapping. Therefore
  $(M(x), M(y))\in DirGI$.
  
  %% TODO is this sufficient proof?

  Suppose $(x, y)\notin R_{eq}$, so $x\neq y$. Suppose $|x|=|y|=n$. Run $M$ on
  input $x$ to yield $G_x=(V_x, E_x)$, and run $M$ on input $y$ to yield
  $G_y=(V_y, E_y)$. Since the graphs are directed, the ``spine'' created by the
  vertices $\{v_1, v_2, \ldots, v_n\}$ and the edges $\{(v_1, v_2), (v_2, v_3),
  \ldots, (v_{n-1}, v_n)\}$ must correspond in both $G_x$ and $G_y$. Let $i$ be
  the index of the first bit at which $x$ and $y$ differ. Suppose without loss
  of generality that $x_i=1$ and $y_i=0$. Then $v'_i\in V_x$ and $(v_i,
  v'_i)\in E_x$, but $v'_i\notin V_y$ so $(v_i, v'_i)\notin E_y$. Assume with
  the intention of producing a contradiction that a bijection exists between
  $V_x$ and $V_y$ which satisfies the conditions for a graph isomorphism. Since
  vertices along the ``spine'' of the $V_x$ must map to vertices along the
  ``spine'' of $V_y$, and specifically $v_i$ in $V_x$ must map to $v_i$ in
  $V_y$, $(v_i, v'_i)\in E_x$ implies $(v_i, v'_i)\in E_y$. But $(v_i,
  v'_i)\notin E_y$ because $y_i=0$. This is a contradiction. Therefore no such
  mapping exists, so $G_x$ is not isomorphic to $G_y$, and hence $(M(x),
  M(y))\notin DirGI$.

  Therefore $(x, y)\in R_{eq} \iff (M(x), M(y))\in DirGI$, so $R_{eq}\kr
  DirGI$.
\end{proof}

\begin{corollary}$R_{eq}\kr GI$\end{corollary}
\begin{proof}Follows directly from Theorem \ref{thm:req_kr_dirgi} and Corollary \ref{cor:gi_ke_dirgi}.\end{proof}

\begin{theorem}\label{thm:reqi_kr_dirgi}$R_{eqi}\kr DirGI$\end{theorem}
\begin{proof}
  Construct machine $M\in \FP$ on input $w\in\sigmastar$:\\
  \begin{algorithm}[H]
    $V_w\gets\{v_1, v_2, \ldots, v_{|w|}, v_{zero}, v_{one}\}$\;
    $E_w\gets\{(v_1, v_2), (v_2, v_3), \ldots, (v_{|w|-1},
    v_{|w|})\}$\tcp*[f]{directed edges}\;
    \For{$i=1$ \KwTo $|w|$}{
      \eIf{$w_i=1$}{
        add directed edge $(v_i, v_{one})$ to $E_w$
      }{
        add directed edge $(v_i, v_{zero})$ to $E_w$
      }
    }
    \Return{$G_w=(V_w, E_w)$}
  \end{algorithm}
  Notice that this machine, as in the machine in the proof of Theorem
  \ref{thm:req_kr_dirgi}, creates a ``spine'' representing each bit of word
  $w$.

  Suppose $(x, y)\in R_{eqi}$, so either $x=y$ or $x=\bar{y}$.

  In the case that $x=y$, $M(x)$ and $M(y)$ output exactly the same graph, so
  $G_x$ is isomorphic to $G_y$, and hence $(M(x), M(y))\in DirGI$.

  In the case that $x=\bar{y}$, define $\phi:V_x\to V_y$ by
  \begin{displaymath}
    \phi(v)=
    \begin{cases}
      v & \plain{if} v = v_i, \plain{for some} i\in\{1, 2, \ldots, |x|\}\\
      v_{zero} & \plain{if} v = v_{one}\\
      v_{one} & \plain{if} v = v_{zero}
    \end{cases}
  \end{displaymath}
  for all $v\in V_x$. Notice that $\phi$ maps each $v_i\in V_x$ to the
  corresponding $v_i\in V_y$, and maps $v_{zero}\in V_x$ to $v_{one}\in V_y$
  and $v_{one}\in V_x$ to $v_{zero}\in V_y$. Then $\forall
  i\in\{1,2,\ldots,|x|\}$, $x_i=0 \iff y=1$, so $(v_i, v_{zero})\in E_x \iff
  (\phi(v_i), \phi(v_{zero})) = (v_i, v_{one})\in E_y$. Similarly, $x_i=1 \iff
  y=0$, so $(v_i, v_{one})\in E_x \iff (\phi(v_i), \phi(v_{one}))\in E_y \iff
  (v_i, v_{zero})\in E_y$. The rest of the edges in $E_x$ map directly to the
  corresponding edges in $E_y$ by $(v_{i-1}, v_i) \mapsto (\phi(v_{i-1}),
  \phi(v_i))=(v_{i-1}, v_i)$, $\forall i\in\{2,3,\ldots,|x|\}$. Therefore,
  $\phi$ describes an isomorphism between $G_x$ and $G_y$, so $(M(x), M(y))\in
  DirGI$.

  Suppose $(x, y)\notin R_{eqi}$, so $x\neq y$ and $x\neq\bar{y}$. Thus
  $\exists i,j\in\{1,2,\ldots,|x|\}, i\neq j,$ such that $x_i=y_i\land x_j\neq
  y_j$. Suppose without loss of generality that $x_i=y_i=0$ and $0=x_j\neq
  y_j=1$. Now $x_i=y_i=0$ implies $(v_i, v_{zero})\in E_x$ and $(v_i,
  v_{zero})\in E_y$. Also, $x_j=0$ implies $(v_j, v_{zero})\in E_x$ and $y_j=1$
  implies $(v_j, v_{one})\in E_y$. Assume, with the goal of producing a
  contradiction, that there exists a bijection, $\phi:V_x\to V_y$ such that
  $(u,v)\in E_x\iff(\phi(u),\phi(v))\in E_y$, $\forall u,v\in V_x$. Since
  $\phi$ must map vertices on the ``spine'' of $G_x$ to corresponding vertices
  in $G_y$, and since $(v_i, v_{zero})\in E_x$, then $(\phi(v_i),
  \phi(v_{zero}))=(v_i, \phi(v_{zero}))$ must be in $E_y$. The only edge of
  this form in $E_y$ is $(v_i, v_{zero})$ so $\phi(v_{zero})=v_{zero}$. Since
  $(v_j, v_{zero})\in E_x$, $(\phi(v_j), \phi(v_{zero}))=(v_j, v_{zero})\in
  E_y$. But the only edge in $E_y$ with source vertex $v_j$ is, by
  construction, $(v_j, v_{one})$. This is a contradiction. Hence no such
  bijection exists, so $G_x$ is not isomorphic to $G_y$, and $(M(x),
  M(y))\notin DirGI$.

  Therefore $(x, y)\in R_{eqi} \iff (M(x), M(y))\in DirGI$, so $R_{eqi}\kr
  DirGI$.
\end{proof}

\begin{corollary}$R_{eqi}\kr GI$\end{corollary}
\begin{proof}Follows directly from Theorem \ref{thm:reqi_kr_dirgi} and 
  Corollary \ref{cor:gi_ke_dirgi}.\end{proof}

\begin{theorem}\label{thm:ra_kr_dirgi}$\forall a\in\sigmastar R_a \kr DirGI$
\end{theorem}
\begin{proof}
  Let $a\in\sigmastar$. Construct machine $M\in\FP$ on input $w\in\sigmastar$
  (let $n=|w|$):\\
  \begin{algorithm}[H]
    $U_w\gets\{v_1, v_2, \ldots, v_n\}$\;
    $U'_w\gets\{v'_1, v'_2, \ldots, v'_n\}$\;
    $F_w\gets\{(v_1, v_2), (v_2, v_3), \ldots, (v_{n-1},
    v_n)\}$\tcp*[f]{directed edges}\;
    $F'_w\gets\{(v'_1, v'_2), (v'_2, v'_3), \ldots, (v'_{n-1},
    v'_n)\}$\tcp*[f]{directed edges}\;
    \For{$i=1$ \KwTo $n$}{
      \If{$w_i=1$}{
	add vertex $u_i$ to $U_w$\;
	add directed edge $(v_i, u_i)$ to $F_w$\;
      }
      \If{$w_i\oplus a_i=1$}{
	add vertex $u'_i$ to $U'_w$\;
	add directed edge $(v'_i, u'_i)$ to $F'_w$\;
      }
    }
    $V_w\gets U_w\cup U'_w$\;
    $E_w\gets F_w\cup F'_w$\;
    \Return{$G_w=(V_w, E_w)$}
  \end{algorithm}
  Notice that for each input string $w$, this machine produces a graph which is
  the disjoint union of two distinct graphs, one representing the bits of $w$
  and the other representing the bits of $w\oplus a$.
  
  Suppose $(x,y)\in R_a$, so either $x=y$ or $x\oplus y=a$.
  
  In the case that $x=y$, $M(x)$ and $M(y)$ output exactly the same graph, so
  $G_x$ is isomorphic to $G_y$, and hence $(M(x), M(y))\in DirGI$.
  
  Consider the case in which $x\neq y$, but $x\oplus y=a$. Define $\phi:V_x\to
  V_y$ by
  \begin{displaymath}
    \phi(v)=
    \begin{cases}
      v'_i & \plain{if} v=v_i \plain{for some} i\in\{1, 2, \ldots, n\}\\
      v_i & \plain{if} v=v'_i \plain{for some} i\in\{1, 2, \ldots, n\}\\
      u'_i & \plain{if} u=u_i \plain{for some} i\in\{1, 2, \ldots, n\}\\
      u_i & \plain{if} u=u'_i \plain{for some} i\in\{1, 2, \ldots, n\}\\
    \end{cases}
  \end{displaymath}
  If no $u_i$ exists or no $u'_i$ exists for some $i\in\{1, 2, \ldots, n\}$
  then we don't define $\phi$ for those values not in the domain. $\phi$ maps
  each $v_i\in V_x$ to $v'_i\in V_y$, so $(v_i, v_{i+1})\in E_x \iff
  (\phi(v_i), \phi(v_{i+1}))=(v'_i, v'_{i+1})\in E_y, \forall i\in\{1, 2,
  \ldots, n-1\}$, and $\phi$ maps each $v'_i\in V_x$ to $v_i\in V_y$, so
  $(v'_i, v'_{i+1})\in E_x \iff (\phi(v'_i), \phi(v'_{i+1}))=(v_i, v_{i+1})\in
  E_y, \forall i\in\{1, 2, \ldots, n-1\}$.

  Now for each $i\in\{1, 2, \ldots, n\}$ for which $u_i$ exists and is in
  $V_x$, $(v_i, u_i)\in E_x$. This occurs if and only if $x_i=1$, and since by
  hypothesis $x\oplus y=a \iff y=x\oplus a$, $y_i=x_i\oplus a_i$. In the case
  that $a_i=0$, then $y_i=x_i\oplus a_i= 1\oplus 0=1$. Since $y_i=1$ and
  $y_i\oplus a_i=1\oplus 0=1$, $M(y)$ produces graph $G_y$ containing vertex
  $u'_i\in V_y$ and directed edge $(v'_i, u'_i)\in E_y$. Now $\phi(u_i)=u'_i$
  is well-defined, and $(v_i, u_i)\in E_x\iff (\phi(v_i), \phi(u_i))=(v'_i,
  u'_i)\in E_y$. In the case that $a_i=1$, then $y_i$
  
  %% TODO complete this proof
\end{proof}

\begin{corollary}$R_a\kr GI$\end{corollary}
\begin{proof}Follows directly from Theorem \ref{thm:ra_kr_dirgi} and 
  Corollary \ref{cor:gi_ke_dirgi}.\end{proof}
