\begin{theorem}\label{thm:rpar-gi}$R_{par}\kr GI$\end{theorem}
\begin{proof}
  Construct $M\in \FP$ on input $w\in\sigmastar$:
  \begin{enumerate}
  \item Initialize set of vertices $V_w=\{v_{p}, v_{o}\}$ and set of
    (undirected) edges $E_w=\{\}$.
  \item For $i=1$ to $|w|$:
    \begin{enumerate}
    \item If $w_i=1$ and $(v_{o}, v_{p})\notin E_w$, add undirected edge
      $(v_{o}, v_{p})$ to $E_w$.
    \item Else if $w_i=1$ and $(v_{o}, v_{p})\in E_w$, remove $(v_{o}, v_{p})$
      from $E_w$.
    \end{enumerate}
  \item Output $G_w=(V_w,E_w)$.
  \end{enumerate}

  Suppose $(x, y)\in R_{par}$, so either $x$ and $y$ both have even parity or
  $x$ and $y$ both have odd parity. 

  If $x$ and $y$ both have even parity, $x$ contains $2k$ ones and $y$ contains
  $2l$ ones, for some $k,l\in\mathbb{N}$. Since $2k$ is even, machine $M$ on
  input $x$ adds then removes the edge $(v_o, v_p)$ to and from $E_x$ an equal
  number of times. Similarly for $M$ on input $y$. Therefore $M(x)$ outputs
  $G_x=(V_x, E_x)$, where $V_x=\{v_o, v_p\}$ and $E_x=\{\}$, and $M(y)$ outputs
  $G_y=(V_y, E_y)$, where $V_y=\{v_o, v_p\}$ and $E_y=\{\}$. Then $G_x$ is
  isomorphic to $G_y$ by the identity function, $I:V_x\to V_y$, defined by
  $I(v)=v, \forall v\in V_x$.

  If $x$ and $y$ both have odd parity, $x$ contains $2k+1$ ones and $y$
  contains $2l+1$ ones, for some $k,l\in\mathbb{N}$. Since $2k+1$ is odd,
  machine $M$ on input $x$ adds edge $(v_o, v_p)$ to $E_x$ one more time than
  it removes the edge. Similarly for $M$ on input $y$. Therefore $M(x)$ outputs
  $G_x=(V_x, E_x)$, where $V_x=\{v_o, v_p\}$ and $E_x=\{(v_o, v_p)\}$, and
  $M(y)$ outputs $G_y=(V_y, E_y)$, where $V_y=\{v_o, v_p\}$ and $E_y=\{(v_o,
  v_p)\}$. Then $G_x$ is isomorphic to $G_y$ by the identity function,
  $I:V_x\to V_y$, defined by $I(v)=v, \forall v\in V_x$.

  Suppose $(x, y)\notin R_{par}$, so without loss of generality, $x$ has even
  parity and $y$ has odd parity. Then $x$ contains $2k$ ones and $y$ contains
  $2l+1$ ones, for some $k,l\in\mathbb{N}$. Since $2k$ is even, machine $M$ on
  input $x$ adds then removes the edge $(v_o, v_p)$ to and from $E_x$ an equal
  number of times. Since $2l+1$ is odd, machine $M$ on input $y$ adds edge
  $(v_o, v_p)$ to $E_y$ one more time than it removes the edge. Therefore
  $M(x)$ outputs $G_x=(V_x, E_x)$, where $V_x=\{v_o, v_p\}$ and $E_x=\{\}$, and
  $M(y)$ outputs $G_y=(V_y, E_y)$, where $V_y=\{v_o, v_p\}$ and $E_y=\{(v_o,
  v_p)\}$. Since $(v_o, v_p)\in E_y$ but $(v_o, v_p)\notin E_x$, so no
  bijection exists between $V_x$ and $V_y$ which preserves edges. Therefore,
  $G_x$ is not isomorphic to $G_y$.

  Therefore $(x, y)\in R_{par} \iff (M(x), M(y)) \in GI$, so $R_{par} \kr GI$.
\end{proof}

\begin{theorem}\label{thm:rbc-gi}$R_{bc}\kr GI$\end{theorem}
\begin{proof}
  Construct $M\in \FP$ on input $w \in \sigmastar$:
  \begin{enumerate}
  \item Initialize set of vertices $V_w=\{v_{zero}, v_{one,0}, v_{one,1},
    v_{one,2}\}$ and set of (undirected) edges $E_w=\{(v_{one,0}, v_{one,1}),
    (v_{one,1}, v_{one,2}), (v_{one,2}, v_{one,0})\}$ (so at this point, the
    graph $(V_w, E_w)$ consists of a single vertex connected to no other
    vertices and a triangle consisting of vertices $v_{one,0}$, $v_{one,1}$,
    and $v_{one,2}$).
  \item for $i=1$ to $|w|$:
    \begin{enumerate}
    \item Add $v_i$ to $V_w$.
    \item If $w_i = 1$, add undirected edge $(v_i, v_{one,0})$ to $E_w$.
    \item Else if $w_i = 0$, add undirected edge $(v_i, v_{zero})$ to $E_w$.
    \end{enumerate}
  \item Output $G_w=(V_w, E_w)$.
  \end{enumerate}
  
  Suppose $(x, y)\in R_{bc}$, so $x$ and $y$ have the same number of ones, say
  $k\in\mathbb{N}$. Assume $|x|=|y|=n$, so both $x$ and $y$ have $n-k$
  zeros. Define $E_{w,1}=\{(v_i, v_{one,0})|v_i = 1\}$ and $E_{w,0}=\{(v_i,
  v_{zero})|v_i = 0\}$, so $E_x = E_{x,1}\cup E_{x,0}$ and $E_y = E_{y,1} \cup
  E_{y,0}$ by construction. Define $V_{w,b}=\{v_i|x_i=b\}$, so $V_x=V_{x,1}
  \cup V_{x,0}$ and $V_y=V_{y,1} \cup V_{y,0}$. Note that
  $|V_{x,1}|=|V_{y,1}|=k$ and $|V_{x,0}|=|V_{y,0}|=n-k$. Since
  $|V_{x,1}|=|V_{y,1}|=k$, there exists a bijection between them, call it
  $\phi_1:V_{x,1}\to V_{y,1}$. Similarly, since $|V_{x,0}|=|V_{y,0}|=k$, there
  exists a bijection between them, call it $\phi_0:V_{x,0}\to V_{y,0}$. Define
  $\phi:V_x\to V_y$ by 
  \begin{displaymath}
    \phi(v) = 
    \begin{cases}
      \phi_0(v) & \plain{if} v = v_i \plain{and} x_i = 1, \plain{for some} i\in\{1,\ldots,|x|\}\\
      \phi_1(v) & \plain{if} v = v_i \plain{and} x_i = 0, \plain{for some} i\in\{1,\ldots,|x|\}\\
      v & \plain{if} v \in \{v_{zero}, v_{one,0}, v_{one,1}, v_{one,2}\}
    \end{cases}
  \end{displaymath}
  for all $v\in V_x$. Notice that each $v_i$ ``corresponds'' to a
  single $x_i$, because each $x_i$ can be either a one or a zero,
  exclusively.

  Since the only edges in $E_x$ are the edges $(v_i, v_{one,0})$ when $x_i=1$
  and $(v_i, v_{zero})$ when $x_i=0$, then $(v_i, v_{one,0})\in V_x
  \iff (\phi(v_i), \phi(v_{one,0}))=(\phi_1(v_i), v_{one,0})\in
  V_y$, and $(v_i, v_{zero})\in V_x \iff (\phi(v_i), \phi(v_{zero}))
  = (\phi_0(v_i), v_{zero})\in V_y$. Therefore $\phi$ describes a graph
  isomorphism, so $G_1$ is isomorphic to $G_2$.
  
  Suppose $(x, y)\notin R_{bc}$, so $x$ and $y$ have a different number of
  ones. Let $k$ be the number of ones in $x$, and $l$ be the number of ones in
  $y$, with $k\neq l$. Suppose without loss of generality that $k>l$. Define
  $E_{w,0}$ and $E_{w,1}$ as above. Now $|E_{x,1}|=k$ and $|E_{y,1}|=l$. Since
  $k>l$, $E_{x,1}$ has at least one more edge adjacent to the triangle created
  by the vertices $\{v_{one,0},v_{one,1},v_{one,2}\}$ than does $E_{y,1}$. Thus
  no possible bijection exists between $V_x$ and $V_y$ which preserves all
  edges. Thus $G_x$ is not isomorphic to $G_y$, so $(M(x), M(y))\notin GI$.

  Therefore $(x, y)\in R_{bc} \iff (M(x), M(y))\in GI$, so $R_{bc}\kr GI$.
\end{proof}

\begin{theorem}\label{thm:req_kr_gi}$R_{eq} \kr GI$\end{theorem}
\begin{proof}The proof is similar as the proofs of Theorems \ref{thm:rpar-gi}
  and \ref{thm:rbc-gi}, but uses a directed graph instead of an undirected
  graph. Since undirected graph isomorphism is polynomial time equivalent to
  directed graph isomorphism, the proof is complete.\end{proof}
\begin{proof}
  Construct machine $M\in\FP$ on input $w\in\sigmastar$:
  \begin{enumerate}
  \item Initialize set of vertices $V_w=\{v_1, v_2, \ldots, v_{|w|}\}$
    and set of directed edges $E_w=\{(v_1, v_2),(v_2,
    v_3),\ldots,(v_{|w|-1}, v_{|w|})\}$.
  \item For $i=1$ to $|w|$:
    \begin{enumerate}
    \item If $w_i=1$, add vertex $v'_i$ to $V_w$ and directed edge $(v_i,
      v'_i)$ to $E_w$.
    \end{enumerate}
  \item Output $G_w=(V_w, E_w)$.
  \end{enumerate}
  This machine constructs a ``spine'' of vertices, with an extra vertex $v'_i$
  and directed edge $(v_i, v'_i)$ adjacent to the spine whenever $w_i$ is a
  one, $\forall i\in\{1,2,\ldots,|w|\}$.

  Suppose $(x, y)\in R_{eq}$, so $x=y$. Then $M(x)$ and $M(y)$ obviously
  produce the same graph, so $(M(x), M(y))\in GI$.
  
  %% TODO is this sufficient proof?

  Suppose $(x, y)\notin R_{eq}$, so $x\neq y$. Suppose $|x|=|y|=n$. Run $M$ on
  input $x$ to yield $G_x=(V_x, E_x)$, and run $M$ on input $y$ to yield
  $G_y=(V_y, E_y)$. Since the graphs are directed, the ``spine'' created by the
  vertices $\{v_1, v_2, \ldots, v_n\}$ and the edges $\{(v_1, v_2), (v_2, v_3),
  \ldots, (v_{n-1}, v_n)\}$ must correspond in both $G_x$ and $G_y$. Let $i$ be
  the index of the first bit at which $x$ and $y$ differ. Suppose without loss
  of generality that $x_i=1$ and $y_i=0$. Then $v'_i\in V_x$ and $(v_i,
  v'_i)\in E_x$, but $v'_i\notin V_y$ so $(v_i, v'_i)\notin E_y$. Since
  vertices along the ``spine'' of the $V_x$ must map to vertices along the
  ``spine'' of $V_y$, and specifically $v_i$ in $V_x$ must map to $v_i$ in
  $V_y$, $(v_i, v'_i)\in E_x$ implies $(v_i, v'_i)\in E_y$. But no such mapping
  can exist, because $(v_i, v'_i)\notin E_y$. Therefore $G_x$ is not isomorphic
  to $G_y$, so $(M(x), M(y))\notin GI$.

  Therefore $(x, y)\in R_{eq} \iff (M(x), M(y))\in GI$, so $R_{eq}\kr GI$.  
\end{proof}

\begin{theorem}$R_{eqi}\kr GI$\end{theorem}
\begin{proof}
  Construct machine $M\in \FP$ on input $w\in\sigmastar$:
  \begin{enumerate}
  \item Initialize set of vertices $V_w=\{v_{zero}, v_{one}, v_1, v_2, \ldots,
    v_{|w|}\}$ and set of directed edges $E_w=\{(v_1, v_2), (v_2, v_3), \ldots,
    (v_{|w|-1}, v_{|w|})\}$.
  \item For $i=1$ to $|w|$:
    \begin{enumerate}
    \item If $w_i=1$ add directed edge $(v_i, v_{one})$ to $E_w$.
    \item If $w_i=0$ add directed edge $(v_i, v_{zero})$ to $E_w$.
    \end{enumerate}
  \item Output $G_w=(V_w, E_w)$.
  \end{enumerate}
  Notice that this machine, as in the machine in the proof of Theorem
  \ref{thm:req_kr_gi}, creates a ``spine'' representing each bit of word
  $w$.

  Suppose $(x, y)\in R_{eqi}$, so either $x=y$ or $x=\bar{y}$.

  If $x=y$, $M(x)$ and $M(y)$ output exactly the same graph, so $G_x$ is
  isomorphic to $G_y$, and hence $(M(x), M(y))\in GI$.

  If $x=\bar{y}$, define $\phi:V_x\to V_y$ by
  \begin{displaymath}
    \phi(v)=
    \begin{cases}
      v & \plain{if} v = v_i\\
      v_{zero} & \plain{if} v = v_{one}\\
      v_{one} & \plain{if} v = v_{zero}
    \end{cases}
  \end{displaymath}
  for all $v\in V_x$. Notice that $\phi$ maps each $v_i\in V_x$ to the
  corresponding $v_i\in V_y$, and maps $v_{zero}\in V_x$ to $v_{one}\in V_y$
  and $v_{one}\in V_x$ to $v_{zero}\in V_y$. Then $\forall
  i\in\{1,2,\ldots,|x|\}$, $x_i=0 \iff y=1$, so $(v_i, v_{zero})\in E_x \iff
  (\phi(v_i), \phi(v_{zero}))\in E_y \iff (v_i, v_{one})\in E_y$. Similarly,
  $x_i=1 \iff y=0$, so $(v_i, v_{one})\in E_x \iff (\phi(v_i),
  \phi(v_{one}))\in E_y \iff (v_i, v_{zero})\in E_y$. The rest of the edges in
  $E_x$ map directly to the corresponding edges in $E_y$ by $(v_{i-1}, v_i)
  \mapsto (\phi(v_{i-1}), \phi(v_i))=(v_{i-1}, v_i)$, $\forall
  i\in\{2,3,\ldots,|x|\}$. Therefore, $\phi$ describes an isomorphism between
  $G_x$ and $G_y$, so $(M(x), M(y))\in GI$.

  Suppose $(x, y)\notin R_{eqi}$, so $x\neq y$ and $x\neq\bar{y}$. Thus
  $\exists i,j\in\{1,2,\ldots,|x|\}, i\neq j,$ such that $x_i=y_i\land x_j\neq
  y_j$. Suppose without loss of generality that $x_i=y_i=0$ and $0=x_j\neq
  y_j=1$. Now $x_i=y_i=0$ implies $(v_i, v_{zero})\in E_x$ and $(v_i,
  v_{zero})\in E_y$. Also, $x_j=0$ implies $(v_j, v_{zero})\in E_x$ and $y_j=1$
  implies $(v_j, v_{one})\in E_y$. Assume, with the goal of producing a
  contradiction, that there exists a bijection, $\phi:V_x\to V_y$ such that
  $(u,v)\in E_x\iff(\phi(u),\phi(v))\in E_y$. Since $\phi$ must map vertices
  on the ``spine'' of $G_x$ to corresponding vertices in $G_y$, and since
  $(v_i, v_{zero})\in E_x$, then $(\phi(v_i), \phi(v_{zero}))=(v_i,
  \phi(v_{zero}))$ must be in $E_y$. The only edge of this form in $E_y$ is
  $(v_i, v_{zero})$ so $\phi(v_{zero})=v_{zero}$. Since $(v_j, v_{zero})\in
  E_x$, $(\phi(v_j), \phi(v_{zero}))=(v_j, v_{zero})\in E_y$. But the only edge
  in $E_y$ with source vertex $v_j$ is, by construction, $(v_j, v_{one})$. This
  is a contradiction. Hence no such bijection exists, so $G_x$ is not
  isomorphic to $G_y$, and $(M(x), M(y))\notin GI$.

  Therefore $(x, y)\in R_{eqi} \iff (M(x), M(y))\in GI$, so $R_{eqi}\kr GI$.
\end{proof}
