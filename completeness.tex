%% \begin{lemma}\label{lem:rl_eq_rel}Let $L\in\NP$, and let $N$ be the
%%   non-deterministic polynomial-time machine that decides $L$.  Then
%%   $R_L=\{(x,y)|x=y \plain{or} N(x\oplus y) \plain{accepts}\}$ is an equivalence
%%   relation in \NPEq.\end{lemma}
%% \begin{proof}To show that $R_L$ is an equivalence relation, we must show that
%%   $R_L$ is reflexive, symmetric, and transitive.
  
%%   Since $x=x$, $(x,x)\in R_L$, so $R_L$ is reflexive.

%%   To show $R_L$ is symmetric, consider $(x,y)\in R_L$. In the case that $x=y$,
%%   then $y=x$ because the equality relation is symmetric, so $(y,x)\in R_L$. In
%%   the case that $N(x\oplus y)$ accepts, then $V(y\oplus x)$ accepts, because
%%   $\oplus$ is commutative, so $(y,x)\in R_L$.

%%   To show $R_L$ is transitive, let $(x,y),(y,z)\in R_L$. In the case that $x=y$
%%   and $y=z$, then $x=z$ by the transitivity of the equality relation, so
%%   $(x,z)\in R_L$. In the case that $x=y$ and $N(y\oplus z)$ accepts, then
%%   $N(x\oplus z)$ accepts, so $(x,z)\in R_L$. In the case that $N( x\oplus y)$
%%   and $y=z$, then $N(x\oplus z)$ accepts, so $(x,z)\in R_L$. In the case that
%%   $N(x\oplus y)$ accepts and $N(y\oplus z)$ accepts and $x\neq y$ and $y\neq
%%   z$, if $x=z$ then $(x,z)\in R_L$. If $x\neq z$, let $c_1=x\oplus y$ and let
%%   $c_2=y\oplus z$. Then $c_1\oplus c_2=x\oplus y\oplus y\oplus z=x\oplus z$.

%%   TODO complete this proof
%%   %% TODO complete this proof
%% \end{proof}

%% proof for V(a, x xor y) in \PEq
%% \begin{lemma}\label{lem:rl_eq_rel}Let $L\in\NP$, and let $V$ be a verifier for
%%   $L$. Then $R_L=\{((a,x),(a,y))|x=y\plain{or}V(a,x\oplus y)\plain{accepts}\}$
%%   is an equivalence relation in \PEq.\end{lemma}
%% \begin{proof}Want to show that $R_L$ is reflexive, symmetric, and transitive.
  
%%   To show $R_L$ is reflexive, we show that $((a,x),(a,x))\in R_L$, because
%%   $x=x$.

%%   To show $R_L$ is symmetric, consider $((a,x),(a,y))\in R_L$. In the case that
%%   $x=y$, then $y=x$ because the equality relation is symmetric, so
%%   $((a,x),(a,y))\in R_L$. In the case that $V(a,x\oplus y)$ accepts, then
%%   $V(a,y\oplus x)$ accepts, because $\oplus$ is commutative, so
%%   $((a,y),(a,x))\in R_L$.

%%   To show $R_L$ is transitive, let $((a,x),(a,y)),((a,y),(a,z))\in R_L$. In the
%%   case that $x=y$ and $y=z$, then $x=z$ by the transitivity of the equality
%%   relation, so $((a,x),(a,z))\in R_L$. In the case that $x=y$ and $V(a, y\oplus
%%   z)$ accepts, then $V(a, x\oplus z)$ accepts, so $((a,x),(a,z))\in R_L$. In
%%   the case that $V(a, x\oplus y)$ and $y=z$, then $V(a, x\oplus z)$ accepts, so
%%   $((a,x), (a,y))\in R_L$. In the case that $V(a, x\oplus y)$ accepts and $V(a,
%%   y\oplus z)$ accepts and $x\neq y$ and $y\neq z$, if $x=z$ then
%%   $((a,x),(a,z))\in R_L$. If $x\neq z$, let $c_1=x\oplus y$ and let
%%   $c_2=y\oplus z$. Then $c_1\oplus c_2=x\oplus y\oplus y\oplus z=x\oplus z$.

%%   TODO complete this proof
%%   %% TODO complete this proof
%% \end{proof}

%% \begin{theorem}\label{thm:krc_npc}If a language $A$ is $\kr$-complete in \NPEq,
%%   then $A$ is $\mor$-complete in \NP. In other words, if $A$ is \NPEq-complete
%%   then $A$ is \NP-complete.\end{theorem}
%% \begin{proof}
%%   Let $L\in\NP$, so there exists a non-deterministic polynomial-time machine,
%%   $N$, which decides $L$, that is, $w\in L\iff N(w)$ accepts. Define
%%   $R_L=\{(x,y)|x=y \plain{or} N(x\oplus y) \plain{accepts}\}$. $R_L$ is an
%%   equivalence relation in $\NPEq$ by \autoref{lem:rl_eq_rel}. Since $A$ is
%%   $\kr$-complete in \NPEq, $R_L\kr A$, so $\exists f\in\FP:(x,y)\in R_L\iff
%%   (f(x), f(y))\in A$.

%%   Construct $M\in\FP$ on input $w$:\\
%%   \begin{algorithm}[H]
%%     $x\gets0^{|w|-1}1$\;
%%     $y\gets w\oplus x$\;
%%     \Return{$(f(x),f(y))$}
%%   \end{algorithm}

%%   Suppose $w\in L$, so $N(w)$ accepts. Since $x\oplus y=w$ by construction,
%%   $N(x\oplus y)$ accepts. Hence $(x,y)\in R_L$, and $(f(x),f(y))\in A$, since
%%   $A$ is $\kr$-complete in \NPEq. Since $M(w)=(f(x),f(y))$ by construction,
%%   $M(w)\in A$.
  
%%   Suppose $w\notin L$ so $N(w)$ does not accept. Since $x\oplus y=w$,
%%   $N(x\oplus y)$ does not accept, so $(x,y)\notin R_L$ and since $A$ is
%%   $\kr$-complete in \NPEq, $(f(x), f(y))\notin A$. Since $M(w)=(f(x),f(y))$ by
%%   construction, $M(w)\notin A$.

%%   Therefore $w\in L\iff M(w)\in A$, therefore $A$ is $\mor$-complete in \NP,
%%   or in other words, $A$ is \NP-complete.
%%   %% TODO double check this proof
%% \end{proof}

%% \begin{corollary}\label{cor:gi_complete}If $GI$ is not $\mor$-complete in
%%   \NP\,then $GI$ is not $\kr$-complete in \NPEq. In other words, if $GI$ is not
%%   \NP-complete then $GI$ is not \NPEq-complete.\end{corollary}
%% \begin{proof}
%%   This is a specific case of the contrapositive statement of
%%   \autoref{thm:krc_npc}.
%% \end{proof}

%% \begin{lemma}\label{lem:kr_mor}If $A\kr B$ then $A\mor B$.\end{lemma}
%% \begin{proof}Since $A\kr B$, $\exists f\in\FP: (x,y)\in A\iff(f(x), f(y))\in
%%   B$. Define $g\in FP$ by $g(x,y)=(f(x), f(y))$. Therefore $A\mor B$ by
%%   $g$.\end{proof}

%% \begin{lemma}\label{lem:np_closed}$\NP$ is closed under polynomial-time,
%%   many-one reductions. In other words, if $A\mor B$ and $B\in\NP$ then
%%   $A\in\NP$.\end{lemma}
%% \begin{proof}
%%   Since $B\in\NP$, there exists a non-deterministic polynomial time Turing
%%   machine which decides $B$; call it $N_B$. Since $A\mor B$, $\exists f\in\FP$,
%%   such that $w\in A\iff f(w)\in B$. Construct non-deterministic Turing machine
%%   $N_A$ on input $w$:\\
%%   \begin{algorithm}[H]
%%     Compute $f(w)$\;
%%     Run $N_B$ on input $f(w)$\;
%%     \lIf{$N_B(f(w))$ accepts}{\ACCEPT}\;
%%     \lElse{\REJECT}\;
%%   \end{algorithm}
%%   Since $f$ runs in polynomial time and $N_B$ runs in non-deterministic
%%   polynomial time, $N_A$ runs in non-deterministic polynomial time. Therefore
%%   $A\in\NP$.
%% \end{proof}

%% \begin{corollary}\label{cor:npeq_closed}$\NPEq$ is closed under polynomial time
%%   kernel reductions. In other words, if $R$ and $S$ are equivalence relations,
%%   such that $R\kr S$ and $S\in\NPEq$, then $R\in\NPEq$.\end{corollary}
%% \begin{proof}Suppose $S\in\NPEq$ and $R\kr S$. By \autoref{lem:kr_mor},
%%   $R\mor S$. Since $S\in\NPEq\subset\NP$ and $R\mor S$, then $R\in\NP$ by
%%   \autoref{lem:np_closed}. Since $R$ is an equivalence relation which can be
%%   decided in non-deterministic polynomial time, it is by definition in \NPEq.
%% \end{proof}

%% \begin{definition}\label{def:gi_classes}$\GIker$ is the class of problems
%%   which are polynomial-time equivalent under kernel reductions to $GI$. In
%%   other words, $\GIker=\{A|A\kequiv GI\}$.

%%   $\GIm$ is the class of problems which are polynomial-time, many-one
%%   equivalent to $GI$. In other words, $\GIm=\{A|A\moequiv GI\}$.

%%   $\GIt$ is the class of problems which are polynomial-time equivalent under
%%   Turing reductions to $GI$. In other words, $\GIt=\{A|A\tequiv GI\}$.
%% \end{definition}

%% \begin{theorem}\label{thm:giker_neq_npeq}If $GI$ is not \NP-complete then
%%   $\GIker\neq\NPEq$.\end{theorem}
%% \begin{proof}
%%   Suppose $GI$ is not \NP-complete. By \autoref{cor:gi_complete}, $GI$ is not
%%   \NPEq-complete. Assume $\GIker=\NPEq$. By \autoref{def:gi_classes}, $GI$ is
%%   complete under kernel reductions in $\GIker$. Hence $GI$ is complete under
%%   kernel reductions in \NPEq, or in other words, $GI$ is \NPEq-complete. This
%%   is a contradiction. Therefore $\GIker\neq\NPEq$.
%% \end{proof}

%% \begin{theorem}If $GI$ is not \NP-complete then
%%   $\GIker\subsetneq\NPEq$\end{theorem}
%% \begin{proof}Let $A\in\GIker$. By \autoref{cor:npeq_closed}, since $A\kr
%%   GI$ by definition of $\GIker$ and since $GI\in\NPEq$, $A\in\NPEq$. Hence
%%   $\GIker\subset\NPEq$. By \autoref{thm:giker_neq_npeq}, if $GI$ is not
%%   \NP-complete then $\GIker\neq\NPEq$. Therefore
%%   $\GIker\subsetneq\NPEq$.\end{proof}

\begin{definition}
  $R_{KC}=\{((G_1, k_1), (G_2, k_2))| G_1 \plain{is isomorphic to} G_2
  \plain{or} \\(G_1 \plain{has a clique of size} k_1 \plain{and} G_2 \plain{has
    a clique of size} k_2 \plain{and} k_1=k_2)\}$
\end{definition}

\begin{theorem}$R_{KC}$ is an equivalence relation.\end{theorem}
\begin{proof}To show that $R_{KC}$ is an equivalence relation, we must show
  that it is reflexive, symmetric, and transitive.

  Since $G$ is isomorphic to $G$ for all graphs $G$, $((G,k),(G,k))\in R_{KC}$,
  for all $k\in\mathbb{N}$, so $R_{KC}$ is reflexive.

  To show that $R_{KC}$ is symmetric, suppose $((G_1, k_1), (G_2, k_2))\in
  R_{KC}$. In the case that $G_1$ is isomorphic to $G_2$, then $G_2$ is
  isomorphic to $G_1$ because the isomorphism relation is symmetric, so
  $((G_2,k_2),(G_1,k_1))\in R_{KC}$. In the case that $G_1$ has a clique of
  size $k_1$ and $G_2$ has a clique of size $k_2$ and $k_1=k_2$, then
  $((G_2,k_2),(G_1,k_1))\in R_{KC}$ because the logical conjunction operation
  is commutative over propositions.

  To show that $R_{KC}$ is transitive, suppose $((G_1, k_1), (G_2, k_2))\in
  R_{KC}$ and $((G_2, k_2), (G_3, k_3))\in R_{KC}$. Since $k_1=k_2$ and
  $k_2=k_3$, then $k_1=k_3$ by the transitivity of the equality relation. There
  are four possible cases for the remaining properties.

  In the case that $G_1$ is isomorphic to $G_2$ and $G_2$ is isomorphic to
  $G_3$, then $G_1$ is isomorphic to $G_3$ so $((G_1, k_1), (G_3, k_3))\in
  R_{KC}$.

  In the case that $G_1$ is isomorphic to $G_2$, $G_2$ has a clique of size
  $k_2$, $G_3$ has a clique of size $k_3$, then $G_1$ has a clique of size
  $k_2=k_1$, so $((G_1, k_1), (G_3, k_3))\in R_{KC}$.

  In the case that $G_1$ has a clique of size $k_1$, $G_2$ has a clique of size
  $k_2$, and $G_2$ is isomorphic to $G_3$, then $G_3$ has a clique
  of size $k_2=k_3$, so $((G_1, k_1), (G_3, k_3))\in R_{KC}$.

  In the case that $G_1$ has a clique of size $k_1$, $G_2$ has a clique of size
  $k_2$, and $G_3$ has a clique of size $k_3$, then $((G_1, k_1), (G_3,
  k_3))\in R_{KC}$.

  Therefore $R_{KC}$ is reflexive, symmetric, and transitive, hence it is an
  equivalence relation.
\end{proof}

\begin{definition}$CLIQUE=\{(G,k)|G \plain{has a clique of size}
  k\}$\end{definition}

\begin{lemma}$CLIQUE$ is \NP-complete.\end{lemma}
\begin{proof}
  The proof is a polynomial time reduction from $3SAT$, which is \NP-complete.
\end{proof}

%% TODO rewrite algorithms using \KwIn and \KwOut
\begin{lemma}\label{lem:rkc_np}$R_{KC}\in\NP$\end{lemma}
\begin{proof}
  Since $GI\in\NP$, it has a deterministic polynomial time verifier, $M_1$,
  which accepts on input $(G_1, G_2, c)$, where $c$ is the isomorphism from
  vertices of $G_1$ to vertices of $G_2$.

  Since $CLIQUE\in\NP$, it has a deterministic polynomial time verifier, $M_2$,
  which accepts on input $(G, k, c)$, where $c$ is the set of vertices in $G$
  which comprise a clique of size $k$.

  To show that $R_{KC}\in\NP$, we construct a deterministic polynomial time
  verifier $M$ for $R_{KC}$. On input $(((G_1, k), (G_2, k)), c)$:\\
  \begin{algorithm}[H]
    \If{$c$ \textnormal{is the encoding of a mapping}}{
      Run $M_1$ on input $(G_1, G_2, c)$\;
      If $M_1$ accepts, \ACCEPT; otherwise \REJECT\;
    }
    \If{$c=(c_1, c_2)$ \textnormal{is the encoding of two cliques}}{
      Run $M_2$ on input $(G_1, k, c_1)$\;
      Run $M_2$ on input $(G_2, k, c_2)$\;
      If $M_2$ accepts on both inputs, \ACCEPT; otherwise \REJECT\;
    }
  \end{algorithm}
  
  Machine $M$ is a verifier for $R_{KC}$, so $R_{KC}\in\NP$.
\end{proof}

\begin{corollary}\label{cor:rkc_npeq}$R_{KC}\in\NPEq$\end{corollary}
\begin{proof}
  Since $R_{KC}\in\NP$ by \autoref{lem:rkc_np} and $R_{KC}$ is an equivalence
  problem, then by \autoref{def:peq}, $R_{KC}\in\NPEq$.
\end{proof}

\begin{theorem}\label{thm:rkc_npc}$R_{KC}$ is \NP-complete.\end{theorem}
\begin{proof}
  Since $R_{KC}\in\NP$ by \autoref{lem:rkc_np}, we need only show that $R_{KC}$
  is \NP-hard. To do this, we construct a polynomial time many-one reduction
  from $CLIQUE$, which is \NP-complete, to $R_{KC}$.

  Construct machine $M$ on input $(G, k)$ which outputs $((G, k), (K_k, k))$,
  where $K_k$ is the complete graph with $k$ vertices.

  Suppose $(G,k)\in CLIQUE$, so $G$ has a clique of size $k$. Then
  $M((G,k))=((G,k), (K_k, k))\in R_{KC}$, because $G$ has a clique of size
  $k$ by hypothesis and $K_k$ has a clique of size $k$ by construction.
  
  Suppose $(G,k)\notin CLIQUE$, so $G$ does not have a clique of size $k$ and
  $M((G,k))=((G,k), (K_k, k))$. If $G$ were isomorphic to $K_k$, then it would
  have a clique of size $k$, specifically the set of all its vertices, but this
  is a contradiction with the hypothesis so no such isomorphism
  exists. Although $K_k$ certainly has a clique of size $k$, specifically the
  set of all its vertices, $G$ does not have a clique of size $k$ by
  hypothesis, so $((G, k), (K_k, k))\notin R_{KC}$.
  
  Therefore $(G,k)\in CLIQUE\iff M((G,k))\in R_{KC}$, so $CLIQUE\mor R_{KC}$,
  and hence $R_{KC}$ is \NP-complete.
\end{proof}

\begin{theorem}$\NPC\cap\NPEq\neq\emptyset$\end{theorem}
\begin{proof}
  By \autoref{cor:rkc_npeq}, $R_{KC}\in\NPEq$, and by \autoref{thm:rkc_npc},
  $R_{KC}\in\NPC$, so $R_{KC}\in\NPEq\cap\NPC$.
\end{proof}

\begin{theorem}$GI\kr R_{KC}$\end{theorem}
\begin{proof}
  Construct machine $f\in\FP$ defined for all graphs $G=(V,E)$ by
  $f(G)=(G,|V|+1)$.

  Suppose $(G_1, G_2)\in GI$, so $G_1$ is isomorphic to $G_2$. This implies
  $|V_1|=|V_2|$ and thus $|V_1|+1=|V_2|+1$. Now $f(G_1)=(G_1, |V_1|+1)$ and
  $f(G_2)=(G_2, |V_2|+1)$. Since $G_1$ is isomorphic to $G_2$ and
  $|V_1|+1=|V_2|+1$, then $((G_1, |V_1|+1),(G_2, |V_2|+1))=(f(G_1), f(G_2))\in
  R_{KC}$.
  
  Suppose $(G_1, G_2)\notin GI$, so $G_1$ is not isomorphic to $G_2$. Since
  $G_1$ cannot have a clique of size $|V_1|+1$, and further, since $G_2$ cannot
  have a clique of size $|V_2|+1$, then $((G_1, |V_1|+1), (G_2,
  |V_2|+1))=(f(G_1), f(G_2))\notin R_{KC}$.

  Therefore $(G_1, G_2)\in GI\iff (f(G_1), f(G_2))\in R_{KC}$, so $GI\kr
  R_{KC}$.
\end{proof}


%% TODO rewrite this section

Now the question is where does $\GIm$ live? There exist exotic problems which
are not equivalence problems, but which are polynomial-time many-one equivalent
to the graph isomorphism problem. For example, Kozen's M-tree clique problem,
the problem of recognizing a legitimate regular deck, and the problem of
recognizing a self-complementary graph are \emph{not} equivalence problems, but
they are indeed polynomial-time many-one equivalent to the graph isomorphism
problem. Since they are not equivalence relations, they certainly cannot be
kernel reduced to graph isomorphism, so $\GIker\subsetneq\GIm$.
%% TODO references

The next question is are there equivalence relations which many-one reduce to
graph isomorphism but which do not kernel reduce to graph isomorphism?
Informally, the question asks if there is some reduction relations such that
the reduction \emph{requires} information about both elements in each pair
(because a many-one reduction gets access to both elements $x$ and $y$ when
$(x,y)\in R$ whereas a kernel reduction gets access only to one at a time). In
fact, the problem of recognizing a self-complementary graph, when expressed as
an equivalence relation, may be a candidate for such a problem.

Examining most of the other problems in the class $\GIm$ reveals that nearly
all other (known) reductions can be expressed as kernel reductions. Many of the
techniques involve either edge replacement or composition of graphs, which can
both be performed on either graph without knowledge of the other. However, the
self-complementary graph problem, which is $SCGI=\{G|G \plain{is isomorphic to}
\bar{G}\}$, does not seem to have this property. Note that this set can be
alternately defined as an equivalence relation, $SCGI=\{(G_1,
G_2)|G_1=\bar{G_2} or \}$.
