\begin{theorem}\label{thm:krc_npc}If a language $A$ is $\kr$-complete in \NPEq,
  then $A$ is $\mor$-complete in \NP. In other words, if $A$ is \NPEq-complete
  then $A$ is \NP-complete.\end{theorem}
\begin{proof}
  %% TODO complete this proof
\end{proof}

\begin{corollary}\label{cor:gi_complete}If $GI$ is not $\mor$-complete in
  \NP\,then $GI$ is not $\kr$-complete in \NPEq. In other words, if $GI$ is not
  \NP-complete then $GI$ is not \NPEq-complete.\end{corollary}
\begin{proof}
  This is a specific case of the contrapositive statement of Theorem
  \ref{thm:krc_npc}.
\end{proof}

\begin{definition}\label{def:gi_classes}$\GIker$ is the class of problems
  which are polynomial-time equivalent under kernel reductions to $GI$. In
  other words, $\GIker=\{A|A\kequiv GI\}$.

  $\GIm$ is the class of problems which are polynomial-time, many-one
  equivalent to $GI$. In other words, $\GIm=\{A|A\moequiv GI\}$.

  $\GIt$ is the class of problems which are polynomial-time equivalent under
  Turing reductions to $GI$. In other words, $\GIt=\{A|A\tequiv GI\}$.
\end{definition}

\begin{lemma}\label{lem:kr_mor}If $A\kr B$ then $A\mor B$.\end{lemma}
\begin{proof}Since $A\kr B$, $\exists f\in\FP: (x,y)\in A\iff(f(x), f(y))\in
  B$. Define $g\in FP$ by $g(x,y)=(f(x), f(y))$. Therefore $A\mor B$ by
  $g$.\end{proof}

\begin{lemma}\label{lem:np_closed}$\NP$ is closed under polynomial-time,
  many-one reductions. In other words, if $A\mor B$ and $B\in\NP$ then
  $A\in\NP$.\end{lemma}
\begin{proof}
  Since $B\in\NP$, there exists a non-deterministic polynomial time Turing
  machine which decides $B$; call it $N_B$. Since $A\mor B$, $\exists f\in\FP$,
  such that $w\in A\iff f(w)\in B$. Construct non-deterministic Turing machine
  $N_A$ on input $w$:\\
  \begin{algorithm}[H]
    Compute $f(w)$\;
    Run $N_B$ on input $f(w)$\;
    \eIf{$N_B(f(w))$ accepts}{
      \ACCEPT
    }{
      \REJECT
    }
  \end{algorithm}
  Since $f$ runs in polynomial time and $N_B$ runs in non-deterministic
  polynomial time, $N_A$ runs in non-deterministic polynomial time. Therefore
  $A\in\NP$.
\end{proof}

\begin{corollary}\label{cor:npeq_closed}$\NPEq$ is closed under polynomial time
  kernel reductions. In other words, if $R\kr S$ and $S\in\NPEq$ then
  $R\in\NPEq$.\end{corollary}
\begin{proof}Follows directly from Lemmas \ref{lem:kr_mor} and
  \ref{lem:np_closed}.\end{proof}

\begin{theorem}\label{thm:giker_neq_npeq}If $GI$ is not \NP-complete then
  $\GIker\neq\NPEq$.\end{theorem}
\begin{proof}
  Suppose $GI$ is not \NP-complete. By Corollary \ref{cor:gi_complete}, $GI$ is
  not \NPEq-complete. Assume $\GIker=\NPEq$. By Definition
  \ref{def:gi_classes}, $GI$ is complete under kernel reductions in
  $\GIker$. Hence $GI$ is complete under kernel reductions in \NPEq, or in
  other words, $GI$ is \NPEq-complete. This is a contradiction. Therefore
  $\GIker\neq\NPEq$.
\end{proof}

\begin{theorem}If $GI$ is not \NP-complete then
  $\GIker\subsetneq\NPEq$\end{theorem}
\begin{proof}Let $A\in\GIker$. By Corollary \ref{cor:npeq_closed}, since $A\kr
  GI$ by definition of $\GIker$ and since $GI\in\NPEq$, $A\in\NPEq$. Hence
  $GIker\subset\NPEq$. By Theorem \ref{thm:giker_neq_npeq}, if $GI$ is not
  \NP-complete then $\GIker\neq\NPEq$. Therefore
  $\GIker\subsetneq\NPEq$.\end{proof}
