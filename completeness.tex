\begin{definition}
  \hangindent=1.2in
  $R_{KC}=\{((G_1, k_1), (G_2, k_2))| k_1=k_2 \plain{and} (G_1\cong G_2
  \plain{or}\\ (G_1 \plain{has a clique of size} k_1 \plain{and} G_2 \plain{has
  a clique of size} k_2))\}$
\end{definition}

\begin{theorem}$R_{KC}$ is an equivalence relation.\end{theorem}
\begin{proof}To show that $R_{KC}$ is an equivalence relation, we must show
  that it is reflexive, symmetric, and transitive.

  Since $G$ is isomorphic to $G$ for all graphs $G$, $((G,k),(G,k))\in R_{KC}$,
  for all $k\in\mathbb{N}$, so $R_{KC}$ is reflexive.

  To show that $R_{KC}$ is symmetric, suppose $((G_1, k_1), (G_2, k_2))\in
  R_{KC}$. In the case that $G_1$ is isomorphic to $G_2$, then $G_2$ is
  isomorphic to $G_1$ because the isomorphism relation is symmetric, so
  $((G_2,k_2),(G_1,k_1))\in R_{KC}$. In the case that $G_1$ has a clique of
  size $k_1$ and $G_2$ has a clique of size $k_2$ and $k_1=k_2$, then
  $((G_2,k_2),(G_1,k_1))\in R_{KC}$ because the logical conjunction operation
  is commutative over propositions.

  To show that $R_{KC}$ is transitive, suppose $((G_1, k_1), (G_2, k_2))\in
  R_{KC}$ and $((G_2, k_2), (G_3, k_3))\in R_{KC}$. Since $k_1=k_2$ and
  $k_2=k_3$, then $k_1=k_3$ by the transitivity of the equality relation. There
  are four possible cases for the remaining properties.

  In the case that $G_1$ is isomorphic to $G_2$ and $G_2$ is isomorphic to
  $G_3$, then $G_1$ is isomorphic to $G_3$ so $((G_1, k_1), (G_3, k_3))\in
  R_{KC}$.

  In the case that $G_1$ is isomorphic to $G_2$, $G_2$ has a clique of size
  $k_2$, $G_3$ has a clique of size $k_3$, then $G_1$ has a clique of size
  $k_2=k_1$, so $((G_1, k_1), (G_3, k_3))\in R_{KC}$.

  In the case that $G_1$ has a clique of size $k_1$, $G_2$ has a clique of size
  $k_2$, and $G_2$ is isomorphic to $G_3$, then $G_3$ has a clique
  of size $k_2=k_3$, so $((G_1, k_1), (G_3, k_3))\in R_{KC}$.

  In the case that $G_1$ has a clique of size $k_1$, $G_2$ has a clique of size
  $k_2$, and $G_3$ has a clique of size $k_3$, then $((G_1, k_1), (G_3,
  k_3))\in R_{KC}$.

  Therefore $R_{KC}$ is reflexive, symmetric, and transitive, hence it is an
  equivalence relation.
\end{proof}

\begin{definition}$CLIQUE=\{(G,k)|G \plain{has a clique of size}
  k\}$\end{definition}

\begin{lemma}$CLIQUE$ is \NP-complete.\end{lemma}
\begin{proof}
  The proof is a polynomial time reduction from $3SAT$, which is \NP-complete.
\end{proof}

%% TODO rewrite algorithms using \KwIn and \KwOut
\begin{lemma}\label{lem:rkc_np}$R_{KC}\in\NP$\end{lemma}
\begin{proof}
  Since $GI\in\NP$, it has a deterministic polynomial time verifier, $M_1$,
  which accepts on input $(G_1, G_2, c)$, where $c$ is the isomorphism from
  vertices of $G_1$ to vertices of $G_2$.

  Since $CLIQUE\in\NP$, it has a deterministic polynomial time verifier, $M_2$,
  which accepts on input $(G, k, c)$, where $c$ is the set of vertices in $G$
  which comprise a clique of size $k$.

  To show that $R_{KC}\in\NP$, we construct a deterministic polynomial time
  verifier $M$ for $R_{KC}$. On input $(((G_1, k_1), (G_2, k_2)), c)$:\\
  \begin{algorithm}[H]
    If $k_1\neq k_2$, \REJECT\;
    \If{$c$ \textnormal{is the encoding of a mapping}}{
      Run $M_1$ on input $(G_1, G_2, c)$\;
      If $M_1$ accepts, \ACCEPT; otherwise \REJECT\;
    }
    \If{$c=(c_1, c_2)$ \textnormal{is the encoding of two cliques}}{
      Run $M_2$ on input $(G_1, k_1, c_1)$\;
      Run $M_2$ on input $(G_2, k_2, c_2)$\;
      If $M_2$ accepts on both inputs, \ACCEPT; otherwise \REJECT\;
    }
  \end{algorithm}
  
  Machine $M$ is a verifier for $R_{KC}$, so $R_{KC}\in\NP$.
\end{proof}

\begin{corollary}\label{cor:rkc_npeq}$R_{KC}\in\NPEq$\end{corollary}
\begin{proof}
  Since $R_{KC}\in\NP$ by \autoref{lem:rkc_np} and $R_{KC}$ is an equivalence
  problem, then by \autoref{def:peq}, $R_{KC}\in\NPEq$.
\end{proof}

\begin{theorem}\label{thm:rkc_npc}$R_{KC}$ is \NP-complete.\end{theorem}
\begin{proof}
  Since $R_{KC}\in\NP$ by \autoref{lem:rkc_np}, we need only show that $R_{KC}$
  is \NP-hard. To do this, we construct a polynomial time many-one reduction
  from $CLIQUE$, which is \NP-complete, to $R_{KC}$.

  Construct machine $M\in\FP$ on input $(G, k)$ which outputs $((G, k), (K_k,
  k))$, where $K_k$ is the complete graph with $k$ vertices.

  Suppose $(G,k)\in CLIQUE$, so $G$ has a clique of size $k$. Then
  $M((G,k))=((G,k), (K_k, k))\in R_{KC}$, because $G$ has a clique of size
  $k$ by hypothesis and $K_k$ has a clique of size $k$ by construction,
  specifically, the set of all vertices in $K_k$.
  
  Suppose $(G,k)\notin CLIQUE$, so $G$ does not have a clique of size $k$ and
  $M((G,k))=((G,k), (K_k, k))$. If $G$ were isomorphic to $K_k$, then it would
  have a clique of size $k$, specifically the set of all its vertices, but this
  is a contradiction with the hypothesis so no such isomorphism
  exists. Although $K_k$ certainly has a clique of size $k$, specifically the
  set of all its vertices, $G$ does not have a clique of size $k$ by
  hypothesis, so $((G, k), (K_k, k))\notin R_{KC}$.
  
  Therefore $(G,k)\in CLIQUE\iff M((G,k))\in R_{KC}$, so $CLIQUE\mor R_{KC}$,
  and hence $R_{KC}$ is \NP-complete.
\end{proof}

\begin{lemma}\label{lem:kr_mor}If $A\kr B$ then $A\mor B$.\end{lemma}
\begin{proof}
  Since $A\kr B$, $\exists f\in\FP: (x,y)\in A\iff(f(x), f(y))\in B$. Define
  $g\in FP$ by $g(x,y)=(f(x), f(y))$. Therefore $A\mor B$ by $g$.
\end{proof}

\begin{theorem}If a language $A$ is \NPEq-complete, then $A$ is
  \NP-complete.\end{theorem}
\begin{proof}
  If $A$ is \NPEq-complete then $R_{KC}\kr A$, since $R_{KC}\in\NPEq$ by
  \autoref{cor:rkc_npeq}. By \autoref{lem:kr_mor}, $R_{KC}\kr A\implies
  R_{KC}\mor A$. Since $R_{KC}$ is \NP-complete by \autoref{thm:rkc_npc}, then
  $A$ is \NP-complete.
\end{proof}

\begin{theorem}$GI\kr R_{KC}$\end{theorem}
\begin{proof}
  Construct machine $f\in\FP$ defined for all graphs $G=(V,E)$ by
  $f(G)=(G,|V|+1)$.

  Let $G_1=(V_1, E_1)$ and $G_2=(V_2, E_2)$, and suppose $(G_1, G_2)\in GI$, so
  $G_1$ is isomorphic to $G_2$. This implies $|V_1|=|V_2|$ and thus
  $|V_1|+1=|V_2|+1$. Now $f(G_1)=(G_1, |V_1|+1)$ and $f(G_2)=(G_2,
  |V_2|+1)$. Since $G_1$ is isomorphic to $G_2$ and $|V_1|+1=|V_2|+1$, then
  $((G_1, |V_1|+1),(G_2, |V_2|+1))=(f(G_1), f(G_2))\in R_{KC}$.
  
  Suppose $(G_1, G_2)\notin GI$, so $G_1$ is not isomorphic to $G_2$. Since
  $G_1$ cannot have a clique of size $|V_1|+1$, and further, since $G_2$ cannot
  have a clique of size $|V_2|+1$, then $((G_1, |V_1|+1), (G_2,
  |V_2|+1))=(f(G_1), f(G_2))\notin R_{KC}$.

  Therefore $(G_1, G_2)\in GI\iff (f(G_1), f(G_2))\in R_{KC}$, so $GI\kr
  R_{KC}$.
\end{proof}

%% TODO references
Here is a list of problems which kernel reduce to the graph isomorphism
problem:
\begin{itemize}
\item
\end{itemize}

Here is a list of problems which many-one reduce to graph isomorphism, but
which do not have an immediately obvious kernel reduction:
\begin{itemize}
\item
\end{itemize}

Here is a list of problems which are not equivalence problems, but which still
many-one reduce to the graph isomorphism problem:
\begin{itemize}
\item Kozen's M-tree clique problem
\item recognition of a self-complementary graph
\item self-complementary graph isomorphism (cannot be asked as an equivalnence
  problem; reflexivity does not hold)
\item recognition of a legitimate regular deck
\end{itemize}
