\begin{theorem}\label{thm:krc_npc}If a language $A$ is $\kr$-complete in \NPEq,
  then $A$ is $\mor$-complete in \NP. In other words, if $A$ is \NPEq-complete
  then $A$ is \NP-complete.\end{theorem}
\begin{proof}
  %% TODO complete this proof
\end{proof}

\begin{corollary}If $GI$ is not $\mor$-complete in \NP\,then $GI$ is not
  $\kr$-complete in \NPEq. In other words, if $GI$ is not \NP-complete then
  $GI$ is not \NPEq-complete.\end{corollary}
\begin{proof}
  This is a specific case of the contrapositive statement of Theorem
  \ref{thm:krc_npc}.
\end{proof}

\begin{definition}\label{def:gi_class}\GI\,is the class of problems which are
  polynomial-time, many-one equivalent to $GI$.\end{definition}

\begin{lemma}$GI$ is $\mor$-complete in \GI.\end{lemma}
\begin{proof}Follows directly from Definition \ref{def:gi_class}.\end{proof}

\begin{theorem}If $GI$ is not \NP-complete and \NPEq\,under kernel reductions
  equals \GI\,under many-one reductions then there exists a language $B$ in
  \NPEq\,such that $B\nkr GI$. In other words $\NPEq\backslash\GI\neq\emptyset$.\end{theorem}
\begin{proof}
  %% TODO complete this proof
\end{proof}
