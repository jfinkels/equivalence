\begin{theorem}\label{thm:krc_npc}If a language $A$ is $\kr$-complete in \NPEq,
  then $A$ is $\mor$-complete in \NP. In other words, if $A$ is \NPEq-complete
  then $A$ is \NP-complete.\end{theorem}
\begin{proof}
  %% TODO complete this proof
\end{proof}

\begin{corollary}\label{cor:gi_complete}If $GI$ is not $\mor$-complete in
  \NP\,then $GI$ is not $\kr$-complete in \NPEq. In other words, if $GI$ is not
  \NP-complete then $GI$ is not \NPEq-complete.\end{corollary}
\begin{proof}
  This is a specific case of the contrapositive statement of Theorem
  \ref{thm:krc_npc}.
\end{proof}

\begin{definition}\label{def:gi_classes}$\GIker$ is the class of problems
  which are polynomial-time equivalent under kernel reductions to $GI$. In
  other words, $\GIker=\{A|A\kequiv GI\}$.

  $\GIm$ is the class of problems which are polynomial-time, many-one
  equivalent to $GI$. In other words, $\GIm=\{A|A\moequiv GI\}$.

  $\GIt$ is the class of problems which are polynomial-time equivalent under
  Turing reductions to $GI$. In other words, $\GIt=\{A|A\tequiv GI\}$.
\end{definition}

\begin{lemma}$\GIker\subset\NPEq$\end{lemma}
\begin{proof}Let $A\in\GIker$, so $A\kequiv GI$.\end{proof}

\begin{theorem}If $GI$ is not \NP-complete then $\GIker\neq\NPEq$.\end{theorem}
\begin{proof}
  Suppose $GI$ is not \NP-complete. By Corollary \ref{cor:gi_complete}, $GI$ is
  not \NPEq-complete. Assume $\GIker=\NPEq$. By Definition
  \ref{def:gi_classes}, $GI$ is complete under kernel reductions in
  $\GIker$. Hence $GI$ is complete under kernel reductions in \NPEq, or in
  other words, $GI$ is \NPEq-complete. This is a contradiction. Therefore
  $\GIker\neq\NPEq$.
\end{proof}

%% TODO is it true that PEq is a subset of NPEq?

\begin{theorem}$\GIker\subseteq\GIm\subseteq\NPEq$\end{theorem}
\begin{proof}
  %% TODO complete this proof
\end{proof}
