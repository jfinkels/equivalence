%% basicfacts.tex - basic facts about kernel reductions
%%
%% Copyright 2010, 2011, 2012, 2014 Jeffrey Finkelstein.
%%
%% This LaTeX markup document is made available under the terms of the Creative
%% Commons Attribution-ShareAlike 4.0 International License,
%% https://creativecommons.org/licenses/by-sa/4.0/.
\section{Basic facts about kernel reductions}
\label{sec:basicfacts}

In this section we provide some basic facts about kernel reductions.

\begin{proposition}\label{prop:compose}
  If $R\kr S$ and $S\kr T$ then $R\kr T$. In other words, polynomial time kernel reductions compose.
\end{proposition}
\begin{proof}
  Let $f$ and $g$ be the polynomial time kernel reductions from $R$ to $S$ and from $S$ to $T$, respectively.
  Then $g\circ f$ computes a polynomial time kernel reduction from $R$ to $T$.
  It is polynomial time computable because polynomial time computable functions compose, and $\pair{x}{y}\in R$ if and only if $\pair{f(x)}{f(y)}\in S$ if and only if $\pair{g(f(x))}{g(f(y))}\in T$.
  Therefore $R\kr T$.
\end{proof}

The next two propositions characerize reducibility among equivalence relations with respect to the number of equivalence classes in each equivalence relation.
The following proposition was first stated as a fact in \autocite{fg11}; the proof is provided here for completeness.

\begin{proposition}\label{prop:noreduction}
  Let $R$ and $S$ be equivalence relations on $\Sigma^*$.
  Suppose $R$ has $n$ equivalence classes and $S$ has $m$ equivalence classes.
  If $n>m$ then $R\nkrnt S$ (that is, $R$ does not kernel reduce to $S$, regardless of any time bound on the function computing the reduction).
\end{proposition}
\begin{proof}
  Assume that $R\krnt S$.
  Then there exists a computable function $f$ such that $\forall x,y$, $\pair{x}{y}\in R\iff \pair{f(x)}{f(y)}\in S$.

  Since $R$ has $n$ non-empty equivalence classes which form a partition of $\Sigma^*$, then $\exists r_1,\ldots,r_n\in\Sigma^*$ such that $R=[r_1]_R\cup\cdots\cup[r_n]_R$.
  Since each element of $R$ is in exactly one equivalence class, $\forall i,j\leq n$, $i=j\iff\pair{r_i}{r_j}\in R\iff\pair{f(r_i)}{f(r_j)}\in S$.
  Therefore the image of each $r_i$ is in some equivalence class in $S$.
  Also, $\forall i,j\leq n$, $i\neq j\iff \pair{r_i}{r_j}\notin R\iff \pair{f(r_i)}{f(r_j)}\notin S$.
  Therefore, the image of each $r_i$ does not relate to the image of any other $r_j$, for $i\neq j$, and $i,j\leq n$.
  Therefore each of the equivalence classes $[f(r_1)]_S,\ldots,[f(r_n)]_S$ is disjoint, so $S$ has at least $n$ equivalence classes.
  But $n>m$.
  This is a contradiction with the hypothesis that $S$ has $m$ equivalence classes.

  Therefore $R\nkrnt S$.
\end{proof}

\begin{proposition}\label{prop:tworeduction}
  Let $R$ and $S$ be equivalence relations on $\Sigma^*$, with $R\in\PEq$.
  If $S$ has at least two equivalence classes, then $R\mor S$.
\end{proposition}
\begin{proof}
  Let $M$ be the deterministic polynomial time Turing machine which decides $R$.
  Let $s_1$ and $s_2$ be representatives of two equivalence classes in $S$ (elements of different equivalence classes do not relate in $S$).
  The many-one reduction from $R$ to $S$ proceeds as follows on input $\pair{x}{y}$: if $M$ accepts $\pair{x}{y}$, output $\pair{s_1}{s_1}$.
  Otherwise output $\pair{s_1}{s_2}$.

  Since $M$ runs in polynomial time, and since the lengths of $s_1$ and $s_2$ do not depend on the lengths of $x$ or $y$, this reduction can be computed in polynomial time.

  If $\pair{x}{y}\in R$ then the reduction outputs $\pair{s_1}{s_1}$, which is in $S$.
  If $\pair{x}{y}\notin R$, then the reduction outputs $\pair{s_1}{s_2}$, which is not in $S$.
  Therefore this is a correct polynomial time many-one reduction from $R$ to $S$.
\end{proof}

%% Propositions \ref{prop:noreduction} and \ref{prop:tworeduction} provide a simple proof of \autocite[Remark~5.2]{bcffm}.

%% \begin{theorem}[{\autocite[Remark~5.2]{bcffm}}]
%%   There exists an infinite sequence of equivalence relations $R_i$, each in \PEq, such that
%%   \begin{displaymath}
%%     \cdots\nkr R_4\nkr R_3\nkr R_2
%%   \end{displaymath}
%%   but
%%   \begin{displaymath}
%%     \cdots\moe R_4\moe R_3\moe R_2.
%%   \end{displaymath}
%%   In other words, polynomial time kernel reductions and polynomial time many-one reductions are different in \PEq.
%% \end{theorem}
%% \begin{proof}
%%   Let $R_i$ be defined as $R_i=\lb\pair{x}{y} \st x\equiv y\pmod{i}\rb$ for all $i\geq 2$.
%%   Equivalence of two integers modulo $i$ can be decided in polynomial time, so $R_i\in\PEq$.
%%   Also, $R_i$ has $i$ equivalence classes.

%%   Since $i+1>i$, by \autoref{prop:noreduction}, $R_{i+1}\nkr R_i$.
%%   But by \autoref{prop:tworeduction}, $R_{i+1}\moe R_i$ for all $i\geq 2$.
%%   This provides the sequence of equivalence relations described in the statement of the theorem.
%% \end{proof}

%% \begin{corollary}\label{cor:finite}
%%   \mbox{}
%%   \begin{enumerate}
%%     \renewcommand{\labelenumi}{\roman{enumi}.}
%%   \item If an equivalence relation has a finite number of equivalence classes, then it is not \PEq-complete.
%%   \item If an equivalence relation has a finite number of equivalence classes, then it is not \NPEq-complete.
%%   \end{enumerate}
%% \end{corollary}
%% \begin{proof}
%%   Suppose $R$ is an equivalence relation.
%%   Assume with the intention of producing a contradiction that $R$ is \PEq-complete.
%%   $R_{eq}=\{\pair{x}{y}|x=y\}$ has an infinite number of equivalence classes (specifically, one for each binary string), and $R$ has a finite number of equivalence classes, by assumption.
%%   By \autoref{prop:numbers}, $R\nkr R_{eq}$.
%%   But since $R_{eq}\in\PEq$, then $R_{eq}\kr R$.
%%   This is a contradiction.
%%   Therefore $R$ is not \PEq-complete.

%%   The proof is the same in \NPEq, because $R_{eq}\in\PEq\subseteq\NPEq$.
%% \end{proof}

The following proposition is presented without proof, because its proof is nearly the same as the proof of the corresponding statement for polynomial time many-one reductions in \NP.

\begin{proposition}\label{prop:closed_under_kr}
  $\NPEq$ is closed under polynomial time kernel reductions, that is, if $S\in\NPEq$ and $R\kr S$ then $R\in\NPEq$.
\end{proposition}

The next lemma states that kernel reductions must preserve ``related-ness'' of pairs of elements by mapping equivalence classes in the domain to equivalence classes in the codomain.

\begin{lemma}\label{lem:image}
  Let $R$ and $S$ be equivalence relations on $U$, and let $w\in U$.
  Suppose $R\krnt S$, by some computable function $f$.
  Then $f([w]_R)\subseteq [f(w)]_S$.
  In other words, the image of an equivalence class of $R$ is a subset of an equivalence class of $S$.
\end{lemma}
\begin{proof}
  Since $w\in [w]_R$, it follows that $f(w)\in f([w]_R)$.
  Let $x\in f([w]_R)$.
  Then $(x, f(w))\in S$, so $x\in [f(w)]_S$.
  Therefore $f([w]_R)\subseteq [f(w)]_S$.
\end{proof}
