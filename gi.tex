%%%%
%% gi.tex
%%
%% Copyright 2011 Jeffrey Finkelstein
%%
%% Except where otherwise noted, this work is made available under the terms of
%% the Creative Commons Attribution-ShareAlike 3.0 license,
%% http://creativecommons.org/licenses/by-sa/3.0/.
%%
%% You are free:
%%    * to Share — to copy, distribute and transmit the work
%%    * to Remix — to adapt the work
%% Under the following conditions:
%%    * Attribution — You must attribute the work in the manner specified by
%%    the author or licensor (but not in any way that suggests that they
%%    endorse you or your use of the work).
%%    * Share Alike — If you alter, transform, or build upon this work, you may
%%    distribute the resulting work only under the same, similar or a 
%%    compatible license.
%%    * For any reuse or distribution, you must make clear to others the 
%%    license terms of this work. The best way to do this is with a link to the
%%    web page http://creativecommons.org/licenses/by-sa/3.0/.
%%    * Any of the above conditions can be waived if you get permission from
%%    the copyright holder.
%%    * Nothing in this license impairs or restricts the author's moral rights.
%%%%
\section{Using the graph isomorphism as a subproblem}
\label{sec:gi}

Let $\Pi$ be an arbitrary graph property (which holds for all isomorphic graphs if it holds for any one of them).
Let $L_\Pi$ be the language on graphs $G$ defined by $\{G\,|\,\Pi(G)$ is true$\}$.
%%Say that $\Pi$ is \defn{hard for $\GI$} if $\GI\mor L_\Pi$.
Say that $\Pi$ is an \defn{\NP-complete property} if $L_\Pi$ is \NP-complete.
Define $A(\Pi)$ by
\begin{displaymath}
  A(\Pi) = \{\pair{G}{H} | G\cong H \plain{or} (\Pi(G) \plain{and} \Pi(H))\}
\end{displaymath}
By checking the three required properties of an equivalence relation, we find the following.
\begin{proposition}
  For all graph properties $\Pi$, $A(\Pi)$ is an equivalence relation.
\end{proposition}

\begin{theorem}
  If $\Pi$ is an \NP-complete property and there exists a graph $H$ with property $\Pi$, then $A(\Pi)$ is an \NP-complete equivalence relation.
\end{theorem}
\begin{proof}
  The previous proposition shows that $A(\Pi)$ is an equivalence relation, so it remains to show that it is \NP-complete.
  The reduction is from $L_\Pi$, and the mapping is given by $G\mapsto\pair{G}{H}$.
  This function is computable in polynomial time (the size of the graph $H$ is constant with respect to $G$).

  Suppose $G\in L_\Pi$, then $\Pi(G)$ and $\Pi(H)$ are both true, so $\pair{G}{H}\in A(\Pi)$.
  Suppose now that $G\notin L_\Pi$, so it certainly must not be the case that $\Pi(G)$ and $\Pi(H)$ are both true.
  However, neither can $G\cong H$ be true, since otherwise $\Pi(G)$ would be true (since the graph property $\Pi$ is true on all graphs which are isomorphic).
  Thus $\pair{G}{H}\notin A(\Pi)$.
  We conclude that $L_\Pi\mor A(\Pi)$, and so it is an \NP-complete equivalence relation.
\end{proof}
