\documentclass[final]{beamer}
\mode<presentation>{\usetheme{I6pd}}

\usepackage[english]{babel}
\usepackage[orientation=landscape,size=a0]{beamerposter}

\usepackage{color}
\usepackage{complexity}

%\usefonttheme{professionalfonts}

\newcommand{\emphblue}[1]{\emph{\textcolor{blue}{#1}}}
\newcommand{\sigmastar}{\Sigma^*}

\title{On the computational complexity of equivalence relations}
\author{Jeffrey Finkelstein}
\institute{Tufts University}
\date{\today}

\begin{document}
\begin{frame}{} 
  %  \vfill
  \begin{columns}[t]
    \begin{column}{.3\linewidth}

      \begin{block}{\LARGE What is an equivalence relation?}
        \Large
        \begin{itemize}
        \item Let $\sigmastar=\{0,1\}^*$. Then
          $R\subseteq\sigmastar\times\sigmastar$ is an \emphblue{equivalence
          relation} if the following properties hold for all
          $x,y,z\in\sigmastar$:
          \begin{list}
            \Large
          \item i. (\emphblue{reflexivity}) $(x,x)\in R$
          \item ii. (\emphblue{symmetry}) $(x,y)\in R\implies (y,x)\in R$
          \item iii. (\emphblue{transitivity}) if $(x,y)\in R$ and $(y,z)\in
            R$, then $(x,z)\in R$
          \end{list}
          If $(x,y)\in R$, we use the notation $x\sim y$ and say \emph{$x$
            relates to $y$}.
        \item $\PEq=\{R\subseteq\sigmastar\times\sigmastar|R$ is an equivalence
          relation for which membership can be \emphblue{decided} by a
          deterministic Turing machine in polynomial time$\}$
        \item $\NPEq=\{R\subseteq\sigmastar\times\sigmastar|R$ is an
          equivalence relation for which membership can be \emphblue{verified}
          by a deterministic Turing machine in polynomial time$\}$
        \end{itemize}
      \end{block}

      \begin{block}{\LARGE Equivalence relations in $\PEq$ and $\NPEq$}
        \Large
        \begin{itemize}
          \setlength{\itemsep}{20pt}
        \item in \PEq:
          \begin{itemize}\Large
          \item \emphblue{the equality relation}: are two strings equal?
          \item \emphblue{same parity}: do two strings have the same parity (even
            or odd number of ones)?
          \item \emphblue{same bitcount}: do two strings have the same number of
            ones?
          \item \emphblue{equal or complement}: are two strings either equal or
            bitwise complements?
          \item \emphblue{tree isomorphism}: are two trees the same up to a
            relabeling of vertices?
          \end{itemize}
        \item in $\NPEq$ (not known to be in \PEq):
          \begin{itemize}\Large
          \item \emphblue{graph isomorphism}: are two graphs the same up to a
            relabeling of vertices?
          \item \emphblue{context-free grammar isomorphism}: do two context
            free grammars produce the same language?
          \item \emphblue{algebraic structure isomorphism}: given two algebraic
            structures, are they the same up to relabeling of elements?
          %\item these problems are actually all of equivalent complexity!
          \end{itemize}
        \item really hard (not known to be in \NPEq):
          \begin{itemize}\Large
          \item \emphblue{boolean formula isomorphism}: are two boolean
            formulas the same up to a permutation of input variables?
          \end{itemize}
        \end{itemize}
      \end{block}

      \begin{block}{\LARGE Why are equivalence relations important?}
        \begin{itemize}
          \Large
        \item graph isomorphism is a candidate for $\NP\backslash\P$ (a million
          dollars!)
        \item as a tool for studying computational complexity
        \item applications in determining equivalence of structures in
          practical computer science and engineering
        \end{itemize}
      \end{block}
    \end{column}

    \begin{column}{.3\linewidth}
      \begin{block}{\large Title}
        Hey there
      \end{block}
    \end{column}

    \begin{column}{.3\linewidth}
      \begin{block}{\large Title}
        Hey there, again!
      \end{block}
    \end{column}
  \end{columns}
  %\vfill
\end{frame}
\end{document}
