\documentclass[final]{beamer}
\mode<presentation>{\usetheme{I6pd}}

\usepackage[english]{babel}
\usepackage[orientation=landscape,size=a0]{beamerposter}

\usepackage{complexity}

\newcommand{\sigmastar}{\Sigma^*}

\title{On the computational complexity of equivalence relations}
\author{Jeffrey Finkelstein}
\institute{Tufts University}
\date{\today}

\begin{document}
\begin{frame}{} 
  %  \vfill
  \begin{columns}[t]
    \begin{column}{.3\linewidth}

      \begin{block}{\LARGE What is an equivalence relation?}
        \Large
        \begin{itemize}
        \item Let $\sigmastar=\{0,1\}^*$. Then
          $R\subseteq\sigmastar\times\sigmastar$ is an \emph{equivalence
          relation} if the following properties hold:
          \begin{list}
            \Large
          \item i. (reflexivity) $\forall x\in\sigmastar$, $(x,x)\in R$
          \item ii. (symmetry) $\forall x,y\in\sigmastar$, $(x,y)\in R\implies
            (y,x)\in R$
          \item iii. (transitivity) $\forall x,y,z\in\sigmastar$, if $(x,y)\in
            R$ and $(y,z)\in R$, then $(x,z)\in R$
          \end{list}
          If $(x,y)\in R$, we use the notation $x\sim y$ and say \emph{$x$
            relates to $y$}.
        \item $\PEq=\{R\subseteq\sigmastar\times\sigmastar|R$ is an
          equivalence relation for which membership can be decided by a
          deterministic Turing machine in polynomial time$\}$
        \item $\NPEq=\{R\subseteq\sigmastar\times\sigmastar|R$ is an
          equivalence relation for which membership can be decided by a
          nondeterministic Turing machine in polynomial time$\}$
        \end{itemize}
      \end{block}

      \begin{block}{\LARGE Examples of equivalence relation in $\PEq$ and
          $\NPEq$}
        \Large
        \begin{itemize}
        \item in \PEq:
          \begin{itemize}\Large
          \item \emph{the equality relation}: $x\sim y$ if $x=y$
          \item \emph{same parity}: $x\sim y$ if both have an even or odd
            number of ones
          \item \emph{same bitcount}: $x\sim y$ if both have the same number of
            ones
          \item \emph{same threshold}: $\forall k\in\mathbb{N}$, $x\sim y$ if
            both have greater than $k$ ones
          \item \emph{equal or inverse}: $x\sim y$ if $x=y$ or $x=\bar{y}$, the
            bitwise complement of $y$
          \end{itemize}
        \item in $\NPEq$ (not known to be in \PEq):
          \begin{itemize}\Large
          \item \emph{graph isomorphism}: given two graphs, is there a
            permutation of the vertices of one which preserves edges in the
            other?
          \end{itemize}
        \end{itemize}
      \end{block}

      \begin{block}{\LARGE Why are equivalence relations important?}
        \begin{itemize}
        \item many applications throughout computer science
          
        \end{itemize}
      \end{block}
    \end{column}

    \begin{column}{.3\linewidth}
      \begin{block}{\large Title}
        Hey there
      \end{block}
    \end{column}

    \begin{column}{.3\linewidth}
      \begin{block}{\large Title}
        Hey there, again!
      \end{block}
    \end{column}
  \end{columns}
  %\vfill
\end{frame}
\end{document}
