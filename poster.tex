\documentclass[final]{beamer}
\mode<presentation>{\usetheme{I6pd}}

\usepackage[english]{babel}
\usepackage[orientation=landscape,size=a0]{beamerposter}

\newcommand{\sigmastar}{\Sigma^*}

\title{On the computational complexity of equivalence relations}
\author{Jeffrey Finkelstein}
\institute{Tufts University}
\date{\today}

\begin{document}
\begin{frame}{} 
  %  \vfill
  \begin{columns}[t]
    \begin{column}{.3\linewidth}
      \begin{block}{\LARGE What is an equivalence relation?}
        \Large 
        \begin{itemize}
        \item Let $\sigmastar=\{0,1\}^*$. Then
          $R\subseteq\sigmastar\times\sigmastar$ is an \emph{equivalence
          relation} if the following properties hold:
          \begin{enumerate}
          \item (reflexivity) $\forall x\in\sigmastar$, $(x,x)\in R$
          \item (symmetry) $\forall x,y\in\sigmastar$, $(x,y)\in R\implies
            (y,x)\in R$
          \item (transitivity) $\forall x,y,z\in\sigmastar$, if $(x,y)\in R$ and
            $(y,z)\in R$, then $(x,z)\in R$
          \end{enumerate}
          If $(x,y)\in R$, we use the notation $x\sim y$ and say \emph{$x$
            relates to $y$}.
          \item \PEq
        
      \end{block}
      \begin{block}{Why are equivalence relations important?}
        Many problems in theoretical and practical computer science are in fact
        problems 
      \end{block}
    \end{column}
    \begin{column}{.3\linewidth}
      \begin{block}{\large Title}
        Hey there
      \end{block}
    \end{column}
    \begin{column}{.3\linewidth}
      \begin{block}{\large Title}
        Hey there, again!
      \end{block}
    \end{column}
  \end{columns}
  %\vfill
\end{frame}
\end{document}
