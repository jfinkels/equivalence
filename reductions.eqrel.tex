\begin{theorem}$R_{par}\kr R_{bc}$\end{theorem}
\begin{proof}
  Construct $M\in \FP$ on input $w\in\sigmastar$:\\
  \begin{algorithm}[H]
    \For{$i=1$ \KwTo $|w|-1$}{
      \If{$w_i=1$}{
        \For{$j=i+1$ \KwTo $|w|$}{
          \If{$w_j=1$}{
            write 0 to both $w_i$ and $w_j$\;
            break\;
          }
        }
      }
    }
  \end{algorithm}
  Notice that this is the machine which finds pairs of ones and writes zeros in
  their place, one pair at a time.

  Suppose $(x, y)\in R_{par}$, so either $x$ and $y$ both have even parity or
  $x$ and $y$ both have odd parity.
  
  If $x$ and $y$ both have even parity, $x$ contains $2k$ ones and $y$ contains
  $2l$ ones, for some $k,l\in\mathbb{N}$. $M(x)$ and $M(y)$ both output the
  string $0^{|x|}$, and since both $M(x)$ and $M(y)$ have a bitcount of zero,
  $(M(x), M(y))\in R_{bc}$.

  If $x$ and $y$ both have odd parity, $x$ contains $2k+1$ ones and $y$
  contains $2l+1$ ones, for some $k,l\in\mathbb{N}$. $M(x)$ and $M(y)$ both
  output a string containing a single one, so both $M(x)$ and $M(y)$ have a
  bitcount of one, $(M(x), M(y))\in R_{bc}$.

  Suppose $(x, y)\notin R_{par}$, so without loss of generality, $x$ has even
  parity and $y$ has odd parity. Then $x$ contains $2k$ ones and $y$ contains
  $2l+1$ ones, for some $k,l\in\mathbb{N}$. Thus $M(x)$ outputs the string
  $0^{|x|}$ and $M(y)$ outputs the string containing a single one. Since the
  bitcount of $M(x)$ is zero and the bitcount of $M(y)$ is one,
  $(M(x), M(y))\notin R_{bc}$.

  Therefore $(x, y)\in R_{par} \iff (M(x), M(y))\in R_{bc}$, so
  $R_{par} \kr R_{bc}$.
\end{proof}

\begin{theorem}$R_{bc}\kr R_{eq}$\end{theorem}
\begin{proof}
  Construct $M\in \FP$ on input $w\in\sigmastar$ which sorts the bits of $w$
  using an efficient (polynomial-time) sorting algorithm. Notice that if
  $|w|=n$ and $w$ contains $k$ ones, this machine outputs the string
  $0^{n-k}1^k$.

  Suppose $(x, y)\in R_{bc}$, so $x$ and $y$ have the same number of ones, say
  $k$. Thus $M(x)=M(y)=0^{n-k}1^k$, so $(M(x), M(y))\in R_{eq}$.
  
  Suppose $(x, y)\notin R_{bc}$, so $x$ and $y$ have a different number of
  ones. Suppose $x$ has $k$ ones and $y$ has $l$ ones, for some
  $k,l\in\mathbb{N}$. Assume without loss of generality that $k>l$. Then
  $M(x)=0^{n-k}1^{k}$ and $M(y)=0^{n-l}1^{l}$, so $M(x)\neq M(y)$. Thus
  $(M(x), M(y))\notin R_{eq}$.

  Therefore $(x, y)\in R_{bc} \iff (M(x), M(y))\in R_{eq}$, so $R_{bc}\kr
  R_{eq}$.
\end{proof}
